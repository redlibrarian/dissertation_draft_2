\begin{quote}
Many of the key themes of the radical right - law and order, the need for social discipline and authority in the face of a conspiracy by the enemies of the state, the onset of social anarchy, the 'enemy within', the dilution of British stock by alien black elements - are well articulated before the full dimensions of the recession are revealed. They emerge in relation to the radical movements and political polarisation of the 1960s, for which '1968' must stand as a convenient, though inadequate notation. \autocite[176]{Hall2017b}
\end{quote}

\section*{The Crisis of the Late-1960s}

The two contrapuntal voices I want to focus on in this chapter are the concrete political processes of the late 1960s on the one hand and the developments in political theory on the other. The claim here is that Rawls' theory of justice appeared at just the right time to revitalize liberal thought faced with the crisis of the 1960s, and that ambiguities or tensions within the theory of justice provided an impetus for further theoretical developments, namely liberalism/libertarianism - focused on individual identity and rights - and communitarianism - focused on collective or communal self-definition. 

In each area - concrete politics and political theory - there is a productive tension between the individual and the collective that demands resolution (though with the anti-dialectical feeling of the time, this resolution is always only local; no final resolution is ever reached or even reachable). In real political events, the repression of individual and collective identity in the name of the post-war compromise led to two different reactions: the hyper-individual self-expression of the hippie counterculture and the expression of collective rights and values on the part of the various social justice movements, anti-colonial uprisings, etc. On the terrain of political theory, this tension is inscribed within Rawls' theory as an attempt to balance individual equality with the "principle of difference". The tension in Rawls' theory provided the impetus for libertarian and communitarian versions of liberalism over the course of the 1970s and 1980s. In the next chapter we will see how this tension - between the individual and the collective - also sparked changes within Intellectual Freedom, creating a double-bind for librarianship in which it is still caught.

In a nutshell, I want to argue that the "resurgence of the people" at the end of the 1960s released social and political energies based on demands for individual rights and freedoms as well as for new forms of collective, communal obligations. The counterculture of the 1960s based on individual freedom was part of a larger explosion of demands for civil rights, decolonization, and identity-based solidarity. What appears contradictory - the combination of individual expression on one hand and collective identity/belonging on the other - is important to note here\footnote{The figure of Bob Dylan might stand as the quintessential expression of this contradiction: the hyperindividualist, anarchic Romantic poet who nevertheless became known for protest songs in the name of social justice.}. Both tendencies were responses to the repression of difference in the context of homogeneous "mass society" of the postwar period\footnote{As Horkheimer and Adorno pointed out in the \textit{Dialectic of Enlightenment} (1944), "the principle of individuality was contradictory from the outset". In the postwar period, "individuals are tolerated only as far as their wholehearted identity with the universal is beyond question" \parencite[124-125]{Horkheimer2002}.}.

David Harvey, in his \textit{Brief History of Neoliberalism}, explains that these energies made possible the manufacture of consent necessary for the success of the neoliberal project over the following decade \parencite{Harvey2005}. Political theory both responded to the late-1960s crisis and provided theoretical legitimation to neoliberalism itself. The tension between individual and collective is exacerbated within neoliberalism as labour and consumption are hyperindividualized (in the forms of the entrepreneur\footnote{See, for example, Foucault's discussion of the neoliberal worker as "entrepreneur of the self" (\textit{entrepreneur de soi}) concerned with self-investment as a way to satisfy individual desire. Foucault describes the "entrepreneur of himself [as] being for himself his own capital, being for himself his own producer, being for himself the source of earnings... we should think of consumption as an enterprise activity by which the individual, precisely on the basis of the capital at his disposal, will produce something that will be his own satisfaction" \parencite[226]{Foucault2008}.} and "disposable culture") while factory logic is expanded into society as a whole (i.e. operaismo's "social factory", to which we will turn below).

It was not only liberalism that had to respond to the challenge of the 1960s, however; radical political theory was also faced with a breakdown or transformation of post-war institutions (the state, the Communist Party, trade unions, etc.). The left sought to engage with social forces which no longer looked to those institutions for protection and the furthering of their interests. We will look at the Italian response to this crisis in the form of \textit{operaismo} (workerism) and autonomism below.

The various civil rights, anti-colonial, and Marxist insurrectionary movements of the 1960s inspired Indigenous sovereignty and Quebecois nationalist movements in Canada. The Front de libération du Québec was formed sometime prior to 1963, for example, and should be considered part of the European movement that produced the Red Brigades (1966) and the Red Army Faction (1970) as well as taking inspiration from national-liberation movements in the Congo, Angola, Kenya, and elsewhere. Similarly, these movements spurred Indigenous resistance to an assimilationiast government policy proposed by Pierre Trudeau's Liberals in 1969. Quebecois nationalism came to a head with the 1970 October Crisis, during which Trudeau invoked the War Measures Act and declared Martial Law. The particularly Canadian version of communitarianism - Charles Taylor's and James Tully's politics of (cultural) recognition - are, on the one hand, good faith attempts to come to grips with the realities of Canadian politics and, on the other hand, theoretical justifications for Canadian state policy in the 1970s and 80s. 

The "new kind of revolution" of 1968 set the stage for all of these developments. The upheavals of 1968 were new because they did not constitute the long-awaited socialist revolt based on vanguard parties and the seizure of state power. Rather, they were an explosion of popular demands for collective civil rights, decolonization, and individual expression. This explosion marked the beginning of the neoliberal transition and provided the means for neoliberal economists and politicians to manufacture the consent necessary for that change\footnote{This is not to argue that civil rights activists, anti-colonialists, or gender and sexual minorities brought neoliberalism on themselves, but that this cultural movement combined with the late 1960s crisis of profitability to give capital a justification for neoliberal restructuring. In his discussion of the transition "from Fordism to flexible accumulation", Harvey writes that "the period from 1965 to 1973 was one in which the inability of Fordism and Keynesianism to contain the inherent contradictions of capitalism became more and more apparent" \parencite[141-142]{Harvey1990}.}. Stuart Hall called the strategy adopted to deal with the crisis (and exemplified in Thatcherism) "authoritarian populism"\parencite{Hall2021h}\footnote{This was the conclusion drawn by Hall, at al., in their 1978 analysis of the aftermath of the late-1960s crisis\parencite{Hall2013a}: that the fear of social anarchy (especially after 1968) "served to win for the authoritarian closure the gloss of popular consent"\parencite[283]{Hall2021h}.} This transition, which in Canada transformed the dominant orientation from a colonial, assimilationist society to multinational, polyethnic, settler-colonial one\footnote{This passage is perhaps most succinctly marked by Quebec's "Quiet Revolution" against the Catholic Church and the anglophone elite from 1960 to about 1970.}, required a theoretical justification, which it found in the different emphases on individualism or collectivity in liberal theory itself. 

In the next section we will look more closely at 1968's "new kind of revolution" before looking at the specific changes in Canadian constitutional politics.

\section*{1968: A New Kind of Revolution}

If, as historian Eric Hobsbawm suggests, 1968 never looked like it could or would be the revolution predicted by socialists, it nonetheless constituted a different sort of revolution. Hobsbawm argued that while the student revolts in 1968 and 1969 were virulent, the students alone could never lead a mass movement against capitalism \parencite[298-299]{Hobsbawm1994}. What Hobsbawm missed, however, is that the students were not alone. Not only were they allied with workers in the 1968 revolts \parencite{Feenberg2001}, but the worker-student revolts were themselves part of a much greater social upheaval, a "resurgence of the people" following the period of capitalist expansion after the Second World War. Twenty years of social peace and individual repression in the name of improved standards of living did not mean, as Hobsbawm claimed, that "revolution was the last thing in the minds of proletarian masses" \parencite[299]{Hobsbawm1994}. Rather, the revolutionary impulse adopted what we would think of today as an intersectional approach \parencite[7]{Taylor2017}.

This is not to suggest that what occurred in 1968 was in fact the orthodox Marxist revolution against class exploitation. The Communist Parties and the trade unions associated with them had become suspect or been rejected outright in many parts of Europe after the Soviet invasion of Hungary in 1956 \parencite{Hall2017a}, though the Maoist tendency continued to be popular both in Paris and in countries of the developing world looking to cast off the chains of colonialism. In the capitalist metropoles, the new revolutionary wave took on a new guise: polyvalent, pluralist, extra-parliamentary, focused on needs, desires, and civil rights. In a sense, this revolutionary wave was "postmodern", rejecting the dominance of the Comintern, the Communist Party, even the Hegelian dialectic itself in favour of the multitude of voices at play with each other. 

In a talk delivered in 1966, Jacques Derrida argued that in the modernist conception of structure, "the function of [the] centre was not only to orient, balance, and organize the structure - one cannot in fact conceive of an unorganized structure - but above all to make sure that the organizing principle of the structure would limit what we might call the \textit{play} of the structure" \parencite[352]{Derrida1978}. The difference between the new kind of revolution and the one orthodox Marxists like Hobsbawm predicted was precisely the rejection of a centre - the liberal state, the Establishment, the Communist Party, the patriarchy - in favour of play. The Situationist International (1957-1972), with its roots in Dada and Surrealism, is a clear expression of this prioritization of play in the revolutionary movement (Guy Debord's \textit{The Society of the Spectacle} appeared in 1967). We will return to the relationship between postmodernism and neoliberalism below.

Beginning in the mid-1950s, then, geopolitical and social changes led to the development of a new kind of revolutionary consciousness. The Algerian War of Independence (1954-1962) gave way to the Vietnam War (1955-1975). The Campaign for Nuclear Disarmament (CND) marches began in 1958, introducing Gerald Holton's peace sign, which would be become perhaps \textit{the} symbol of the hippie counterculture a decade later. The Selma-to-Montgomery marches if 1965 may stand for the whole American Civil Rights movement. The \textit{Mouvement de libération des femmes} was founded in 1968 by Antoinette Fouque, and in the same year feminists staged a protest against the Miss America pageant: two key events in the development of second-wave feminism. In 1969, the "Stonewall Riots" marked a turning point in what was then called the gay liberation movement, though as historical Susan Stryker points out, the riots were part of a larger struggle over sexual orientation and gender identity:

\begin{quote}
Gay, transgender, and gender-variant people had been engaging in violent protest and direction action against social oppression for at least a decade by that time. Stonewall stands out as the biggest and most consequential example of a kind of event that was becoming increasingly common rather than as a unique occurrence. By 1969, as a result of many years of social upheaval and political agitation, large numbers of people who were socially marginalized because of their sexual orientation or gender identity, especially younger people who were part of the Baby Boom generation, were drawn into the idea of "gay revolution" and were primed for any event that would set such a movement off. The Stonewall Riots provided that very spark, and they inspired the formation of Gay Liberation Front cells in big cities, progressive towns, and college campuses all across the United States. \parencite[82]{Stryker2017}
\end{quote}


While the arrival of neoliberalism is often dated to the late 1970s, with the electoral victories of Thatcher (1979) and Reagan (1980), these electoral victories are better seen as the culmination of processes that began to be visible in the 1960s. Cultural theorist Stuart Hall argued that between 1964 and 1968 the world turned \parencite[149]{Hall2017}. The Civil Rights movement in the US gave strength to Black struggles in the UK in the context of a broader and deeper resistance to the postwar compromise between capital and labour that relied on assimilation and repression to keep the social peace. Hall saw this period as one in which "the great consensus of the 1950s" was challenged, when the state and the ruling classes began to understand that what appeared merely as anti-establishment childishness (we can think of the Beatles or Monty Python here) or a fad for "permissiveness" was in fact "something worse than that - something close to an organized and active conspiracy against the social order" \parencite[149-150]{Hall2017}.

One particularly influential locus of this conflict was the workplace where,  Grégoire Chamayou notes, capital tried to overcome the crisis of profitability (i.e. extract more and more surplus value) by intensifying labour discipline, Fordist automation, surveillance and control. 

\begin{quote}
when confronted with acts of worker indiscipline, the management could find no better solution than to respond by intensifying the disciplinary regime that this indiscipline had rejected in the first place, then fanning it to such an extent that it was radicalized and turned into open revolt. [...] Faced with the crisis of disciplinary governability, a new art of governing labour would need to be invented. \parencite[13]{Chamayou2021}
\end{quote}

This "new art" was neoliberal post-Fordism and the social factory. 
\bigskip
\newline
David Harvey argues that it was the repression of individual and collective needs and desires in the name of postwar reconstruction and prosperity that led to the explosion of new social and political demands. That such prosperity would be rejected by workers and the oppressed came as a shock. As Grégoire Chamayou writes, "we need to try and imagine the immense and painful surprise represented by the movements of the 1960s for those who were firmly convinced of the withering away of social conflict in the 'consumer society'" \parencite[14]{Chamayou2021}  The neoliberal project, based on free-market fundamentalism\footnote{Quinn Slobodian has recently argued that we should not over-emphasize the free market within neoliberalism. "If we place too much emphasis on the category of market fundamentalism, we will fail to notice that the real focus of neoliberal proposals is not on the market per se but on redesigning states, laws, and other institutions to protect the market" \parencite[6]{Slobodian2018}.} and individual consumer choice, recuperated the energies released in 1968 and 1969 to further its own political and economic project in the early 1970s \parencite[10]{Harvey2005}. 

Between 1968 and the end of the Bretton Woods accords in 1973, capital sought to reject outright the new social demands (what Boltanski and Chiarello call the "first exit strategy" from the crisis of 1968 \parencite[177]{Boltanski2005}), while after 1975 capital adopted a new strategy to disarm its critics. On the one hand, the labour-capital relationship was repaired by granting some autonomy to workers, autonomy previously restricted to the managerial strata. On the other hand, the intellectual demands of marginalized identities was defused through the granting of similar benefits:

\begin{quote}
Many of those who had been voicing [a social] form of criticism at the time of the 1968 crisis had become satisfied with the changes that had taken place in the organization of work and, more broadly, in society. The incorporation of many components of [intellectual] criticism into the new spirit of capitalism had deprived earlier critics of the achievements of the so-called liberation movement. \parencite[178]{Boltanski2005}
\end{quote}

Grégoire Chamayou characterizes this process as attempting to stimulate the participation of workers through a "strategy of commitment" versus an older "strategy of control" \parencite[16]{Chamayou2021}. 

Worker revolt was not the only locus of unrest arising from the "resurgence of the people". In Canada, Indigenous sovereignty movements and Quebecois nationalism posed more of a threat to the existing constitutional order, and prompted a change in federal government policy as well as a philosophical justification for it. We will look at this process next before turning to the response of political theory itself. 

\section*{The Crisis of Canadian Constitutionalism}

The conflict between universal individual equality and collective belonging played out in Canadian politics in new ways from the end of the 1960s. Writing in 1946, Canadian historian Frank Underhill wrote that while "Canada is caught up in [the] modern crisis of liberalism" \parencite[5]{Underhill1961} of the immediate postwar period, 

\begin{quote}
in the world of ideas we do not yet play a full part. We are still colonial. Our thinking is still derivative... For our intellectual capital we are still dependent upon a continuous flow of imports from London, New York, and Paris, not to mention Moscow and Rome. \parencite[6]{Underhill1961}
\end{quote}

Underhill referred to the 1950s variously as a period of "calm" and of "stagnation", writing in 1956 that "We in Canada cannot much longer remain aloof from the deeper intellectual currents of the twentieth century. It is high time for our younger academic liberals to start something" \parencite[242]{Underhill1961}.  Underhill looked to a younger generation to stimulate and revitalize Canadian politics, and by the 1950s this was beginning to occur. 

In 1949, the young Pierre Elliot Trudeau became a leading figure in Quebec's "Quiet Revolution" against the Catholic Church, the Anglophone business class, and the stagnant political establishment in Quebec. The slogan of the Quiet Revolution - "maîtres chez nous" (masters in our own house) - was the motto of the Quebec Liberals in the 1960 election, marking the victory of the progressive forces Underhill predicted. But within Quebec, the sentiment of self-government led to the development of separatist forces culminating in the October Crisis of 1970. Trudeau himself entered federal politics in 1965 with a programme of liberal universal egalitarianism. By the mid-1960s, then, the stagnation of Canadian political and social life had been overcome, but only by exposing the tensions between liberal individualism and the shared-identity of various Canadian regions and cultures.

Besides Quebecois sovereignty, Indigenous activism also received fresh impetus in the late-1960s. The National Indian Brotherhood, formed in 1968, represented a new sense of pan-Indigenous awareness in Canada and indeed across North America. In the US, the American Indian Movement was also founded in 1968 as part of a "new wave" of Indigenous activism \parencite[3]{Nichols2020} which alongside Quebec nationalism should be understood as part of a global struggle for sovereignty, such as the post-colonial struggles in Asia and Africa in the 1960s. 

Under pressure from Indigenous and Quebecois resurgence, liberal governments like Trudeau's sought to increase the universalism of post-war social policy (i.e. the social liberalism of the welfare state). In his 1968 book, \textit{Federalism and the French Canadians}, Trudeau merely expressed a liberal orthodoxy when he wrote that "the state... must seek the general welfare of all its citizens regardless of sex, colour, race, religious beliefs, or ethnic origin" \parencite[4]{Trudeau1968}. In formulating the influential "Just Society" programme, Trudeau "rejected the notion that any group could be accorded a position separate from the rest of the population and was convinced that removing the legislated difference between Indigenous and other Canadians [i.e. the Indian Act] could cure Canada's 'Indian Problem'" \parencite[49-50]{Nickel2019}

Framing the solution to the problem as one of \textit{removing} a legislated difference exemplifies Berlin's concept of "negative liberty". But it also places Trudeau squarely in the classical liberal tradition of the equality of individuals. Underhill noted that as late as the 1950s, "Canada, in fact, is still living mentally and spiritually in the nineteenth century. Our problems of liberalism and democracy are mostly nineteenth-century problems" \parencite[230]{Underhill1961}. Underhill argues that on the centenary of the publication of Mill's \textit{On Liberty} in 1959, "it will not be in Britain or the United States that critics will most easily find cases to illustrate what was worrying Mill and Tocqueville. The typical community in which collective mediocrity and democratic uniformity reign supreme and unquestioned will be our own English-speaking Canada" \parencite[231]{Underhill1961}.

With the "Just Society" programme, Trudeau attempted to reinscribe democratic uniformity in Canadian politics, in the face of challenges made by Quebecois and Indigenous nationalists. The response to Indigenous activism, for example, was the 1969 "White Paper" produced by then Minister of Indian Affairs Jean Chrétien. The White Paper proposed scrapping the Indian Act, eliminating "Indian Status", and fully assimilating Indigenous peoples into settler culture and society. However, Trudeau and Chrétien misunderstood "Indigenous realities and how liberal concepts of individualism, freedom, and equality ran counter to Indigenous peoples' history, collective rights, and self-identification" \parencite[50]{Nickel2019}\footnote{Canadian individualism is thoroughly bourgeois. As Underhill remarks, in the early 1900s, "at last Canada was a nation, for there was so much prosperity to distribute that every section could be satisfied and every individual could become a capitalist" \parencite[40]{Underhill1961}. Indigenous peoples and Quebecois francophones were excluded from this prosperity, and by mid-century the only path liberals like Trudeau could see was to assimilate them into bourgeois notions of nation and individuality.}.

The White Paper policy was seen by Indigenous peoples as a form of "cultural genocide". This term sparked enormous controversy when it was used to describe the residential school system in the final report of the Truth and Reconciliation Commission \parencite[1]{TruthandReconciliationCommissionofCanada2015}, but it was first used in 1969 by Harold Cardinal to describe the effects of the white paper policy. In his response to the federal position, \textit{The Unjust Society}, Cardinal wrote that:

\begin{quote}
The new Indian policy promulgated by Prime Minister Pierre Elliott
Trudeau’s government... is a thinly disguised programme of
extermination through assimilation. For the Indian to survive, says
the government in effect, he must become a good little brown white
man. The Americans to the south of us used to have a saying:
”the only good Indian is a dead Indian.” The [White Paper] doctrine
would amend this but slightly to ”The only good Indian is a
non-Indian”. \parencite[1]{Cardinal1969}
\end{quote}

In Sarah Nickel's view, reaction to the White Paper in combination with Indigenous political action (including the formation of new associations and organizations) created space for the introduction of Indigenous discourse to the Canadian political landscape\parencite[52]{Nickel2019}. According to Glen Coulthard, a resurgence of "Indigenous anticolonial activism" in the early 1970s forced the Canadian government to abandon its politics of assimilationist cultural genocide and adopt "a seemingly more conciliatory set of discourses and institutional practices that emphasize [Indigenous] recognition and accomodation" \parencite[6]{Coulthard2014}. Legal struggles, such as the landmark Calder (1973) and Delgamuukw (1997) decisions on the recognition of land and treaty rights, were important elements in the process that led in a direct line from the conflict between liberal egalitarianism and national identity to the politics of recognition formalized by Taylor and Tully in the early- to mid-1990s \footnote{Indigenous land rights, for example in Tully, only occur within the framework of recognition by a sovereign figure, rather than as the pre-existing self-determination of Indigenous peoples themselves. Paradoxically Indigenous self-determination can only take effect by being recognized by a transcendentally sovereign figure; as we will see, the same is true of the right to free speech, free expression, and Intellectual Freedom.}.

Quebecois sovereignty, which came to a head at the same time as the Indigenous resurgence identified by Coulthard, also contributed to this process. The nationalist Front de libération du Québec (FLQ), formed in the early 1960s, carried out a number of attacks between 1963 and 1970, including the bombing of the Montreal Stock Exchange in 1969. In October 1970, the group kidnapped British Trade Commissioner James Cross and subsequently kidnapped and killed Quebec labour minister Pierre Laporte. Trudeau's response to the October Crisis was to trigger the War Measures Act and declare martial law. The October Crisis subsequently caused a shift in the Quebecois sovereigntist movement, as sovereigntists repudiated violence and focused on legal measures to advance the nationalist cause, such as the passing of the Charter of the French Language ("Bill 101") in 1977\footnote{It is interesting to compare the Canadian experience with the Italian, where the "hot autumn" of 1969 led to the "years of lead" in the 1970s, until the assassination of Prime Minister Aldo Moro in 1978 forced the Italian state into a final repression of left-wing militant activity.}.

The Quebecois struggle for legal recognition reached a high point with the referendum on independence in 1980. The referendum failed but it underlined the importance of Quebecois recognition for Trudeau's project of national unity. The patriation of the constitution from Great Britain took place in 1982, amending the British North America Act and supplementing it with a Canadian Charter of Rights and Freedoms (modeled on the UN Declaration of 1948). This Charter followed the same liberal principles we have seen Trudeau adopt before. However, Trudeau allowed that the constitutional protection of language and education rights would recognize, at least to a limited extent, Quebec's identity as a "distinct society" within the Canadian federation, and would therefore undermine Quebecois nationalism. But this departure from strict equality was only a tactical retreat in Trudeau's larger strategy of universalism. One commentator has written that for Trudeau, "a bill of rights would fulfil one goal of the original Canadian constitution - to allow all its citizens to 'consider the whole of Canada their country and field of endeavour'" \parencite[262]{Weinrib1998}. Indeed, when subsequent Prime Ministers sought to appease Quebec by strengthening its distinct status in the constitution, Trudeau sought to derail the process.

Quebec was not satisfied with the level of recognition included in the new
Canadian constitution. As of 2021, Quebec has still not approved the 1982 Constitution Act (Quebecois consent was not a requirement for the passing of the
Act, an and attempt by Quebec to veto the new constitution was rejected by
the supreme court). In 1987, a round of constitutional negotiations was opened
by Prime Minister Brian Mulroney with the express intention of getting Quebec
to endorse the 1982 constitution (the Meech Lake Accords). Recognition of
Quebec as a "distinct society" within the Canadian federation proved divisive,
with Trudeau then denouncing Quebec’s claim to a separate identity. As Ian
MacDonald has written in his account of Quebecois politics at the time, ”the
distinct society and limits to the federal spending power would prove to be the
most difficult points with opponents of any form of recognition of Quebec’s
distinctiveness within Canada and proponents of a centralist vision of federalism”
\parencite[250]{MacDonald2002}.

In its 1985 election platform, the Quebec Liberal Party asked for distinct
society recognition to be placed in the preamble to the Constitution. Trudeau,
among others, was dissatisfied with this proposal\footnote{Although Trudeau had retired from politics in 1984, he led attacks on Quebec’s demands and
the Mulroney government’s response to those demands\parencite[268-272]{MacDonald2002}.}. One major concern over a constitutionally-enshrined distinct status for Quebec was the problem of "asymmetrical
federalism" (Kymlicka 2005, p. 278) which has plagued Canadian politics
since confederation. Will Kymlicka sums up the problem of Quebec’s distinct
society status, writing:

\begin{quote}
Most English Canadians overwhelmingly reject the idea of "special status" for Quebec. To grant special rights to one province on the grounds that it is nationality-based, they argue, is somehow to denigrate the other provinces, and to create two classes of citizens. \parencite[279]{Kymlicka2005}\footnote{The Quebecois of course, like Indigenous peoples, have always felt that there were two classes of citizens in Canada.  The characterization of English-French relations as "two solitudes" comes from the title of Hugh Maclennan's 1945 novel.}
\end{quote}

Women's groups, Indigenous groups, and other minority groups also criticized the recognition of Quebec's status as a distinct society, fearing that such a status could allow Quebec to disregard elements of the Charter of Rights and Freedoms\footnote{Indeed, the "notwithstanding clause" of the Canadian Charter, which allows Parliament or provincial legislatures to override sections of the Charter, has often been used in this way. The Quebec language laws are the most famous example of the use of the notwithstanding clause, but it has also been used to enforce a ban on religious face-coverings. In Alberta, the province has (unsuccessfully) tried to use the clause to oppose same-sex marriage and to limit compensation for forced sterilization.}. In the end the Meech Lake accords were defeated due to a lack of provincial consensus when Elijah Harper, member of Manitoba Legislative Assembly, voted against the amendments due to the lack of Indigenous representation in the constitutional process.

In both the Indigenous and Quebecois contexts, "recognition" became a core part of Canadian legal and constitutional procedures in the 1970s and 1980s. The Calder and Delagmuukw decisions, among others, recognized Indigenous land claims while the recognition of Quebec as a distinct society is at the heart of Canadian constitutional debates. It is unsurprising then that the "politics of recognition" should be a key topic in Canadian political philosophy. Emerging as a specific tendency out of the communitarian challenge to liberalism, the politics of recognition seeks to moderate the core liberal ideas of universal egalitarianism, individual rights, objective proceduralism, as well as state and social agnosticism towards a substantive vision of the good.

\section*{The Constitutional Context} 

As we saw in the last chapter, the politics of recognition began as a pragmatic response by the Canadian government to the challenges posed by Indigenous sovereignty claims and Quebecois nationalism. The recognition of Indigenous rights, especially land rights, and Quebec's identity as a distinct society, especially around language rights, were formalized in specific legal and juridical recognition (the Calder land rights case of 1973, for example, and Quebec's Loi 101 or "Charte de la langue française" of 1977). The Canadian constitution, patriated from Britain in 1982, included elements of the practical strategy of recognition, particularly as it concerned Quebec.

After a decade or so of pragmatic application prior to the patriation of the Constituion came a decade of attempts at constitutional reform which sought to define the proper recognition of minority and multicultural groups. In many ways, recognition began to be fought out at the political level through this constitutional discussions. Tully points out that various shared-identity groups were concerned that protection for some cultural groups in the constitution could come at the expense of other group rights \autocite[12]{Tully1995} (See chapter X, page Y for Trudeau's role in raising alarms about the Meech Lake accords).

The 1987 Meech Lake Accord, which sought to recognize Quebec's distinctiveness and decentralize the federal government, failed in part due to the lack of Indigenous representation in the constitutional process; the 1992 Charlottetown Accord, which again tried to enshrine Quebecois recognition while also supporting a limited notion of Indigenous self-government \autocite[13]{Kymlicka1996}, was narrowly rejected in a national referendum. The failure of two rounds of constitutional amendment which sought, in a limited way at least, to address Indigenous and Quebecois political issues, opened the door to increased Indigenous resistence and Quebecois nationalism in the 1990s. 

In the lead up to the Charlottetown Accord, the Canadian government argued for a Canadian exceptionalism that balanced individual and collective rights:

\begin{quote}
From its beginnings - in democracy, freedom and the rule of law - Canada has developed its own unique way of governing, its own special relationship between citizen and the state. Whether out of genius or necessity, the architects of Canada provided a framework which has allowed us to build a country on the basis of what appear increasingly to be universal values - freedom, equality, compassion and community - in a distinctly Canadian way. \autocite[2]{GovernmentofCanada1991}
\end{quote}

This image of Canada erases generations of colonial oppression of Indigenous peoples, the treatment of Quebecois as second-class citizens (for example in the conscription crises of 1917 and 1944), and the internment of Japanese Canadians beginning in 1942. The image is a representation of Canada designed to ensure unity (i.e. liberal universalism) in preparation for constitutional reform.

The justification for this reform, the government argued, was not only the historical changes that had taken place since 1867 but the challenges to Canadian citizenship posed by Indigenous peoples and Quebecois nationalism\footnote{To a lesser extent, the other rights-groups of the "resurgence of the people" - women, LGBTQ people, and post-colonial immigrants were part of this attempt to enshrine recognition within the Canadian constitutional order.}, challenges the universal citizenship supported by Trudeau had been unable to deal with. 

The government explicitly framed Indigenous and Quebecois demands in terms of recognition: "Aboriginal Canadians are frustrated by a Constitution that does not fully recognize their special place in the Canadian society... Canada must also address Quebec's desire for recognition of its distinct nature..." \autocite[vi]{GovernmentofCanada1991}. The constitutional sought to incorporate recognition of collective rights, balanced (and limited) by liberalism's individual egalitarianism:

\begin{quote}
Being Canadian does not require that we all be alike. Around a core set of shared values, Canadian citizenship accommodates a respect of diversity that enriches us all... In the Canadian experience, it has not been enough to protect only universal individual rights. Here, the Constitution and ordinary laws also protect other rights accorded to individuals as members of certain communities. This accommodation of both types of rights makes our constitution unique and reflects the Canadian value of equality that accommodates difference. \autocite[1-3]{GovernmentofCanada1991}
\end{quote}

Published on the eve of constitutional reform requiring national agreement, there is an element of propaganda to the government's insistence on Canada's unique commitment to communitarian freedom. Nonetheless this document provides a useful illustration of the hegemonic perspective on Canadian government, a post-Trudeau departure from universality not only maintained by the federal government itself but one it expected to resonate with a majority of Canadians. That the Charlottetown Accord failed to pass in a national referendum indicates that the government's representation of "Canada" failed to convince the majority.

In any event, the image presented by the government was a false one. In the early- and mid-1990s, two crises highlighted the continued dissatisfaction of Indigenous peoples and Quebec in ways that went beyond the recognition of difference, and undermined the federal government's picture of Canada as a place of tolerance and accommodation. The Kanehsatà:ke resistance in 1990 and the second Quebec referendum on independence (1995) surround the theoretical work on the politics of recognition just as much as the existing question of constitutional reform.

\section*{Kanehsatà:ke and the Quebec Referendum}

The politics of recognition thus began as a pragmatic response by the Canadian government to the challenges posed by Indigenous sovereignty and Quebecois nationalism. The recognition of Indigenous rights (especially land rights) and Quebec's identity as a distinct society (especially language rights) were formalized in specific legal and juridical recognition (the Calder land rights case of 1973, for example, and Quebec's Loi 101 of 1977). 

The political philosophy Charles Taylor called "the politics of recognition" was formalized after a decade of attempts to "properly" define the recognition of minority and multicultural groups through the Canadian constitutional process after 1982. While Taylor's contribution was more philosophical, James Tully's contribution to the politics of recognition specifically addressed the constitutional question. Will Kymlicka, writing at the same time, defended the "liberal" side of the liberal-communitarian debate, though he too tried to balance individual and collective rights by showing that "many (but not all) of the demands of ethnic and national groups are consistent with liberal principles of individual freedom and social justice" \parencite[193]{Kymlicka1996}.

By the time Taylor and Tully were writing about recognition, however, Canadian politics had seen a resurgence of both Indigenous sovereignty demands and Quebecois nationalism. The patriation of the constitution and the development of the Canadian Charter of Rights and Freedoms was intended to balance, recognize, and therefore resolve the tensions inherent in a polyethnic, multicultural, and multinational colonial federation, but the early 1990s it was clear that this was a false promise. Kymlicka notes in 1996 that the failed Charlottetown Accords would have recognized Indigenous peoples' right to self-government and Quebec's distinct status as the only culturally and linguistically French culture in North America \parencite[13]{Kymlicka1996}. But while Kymlicka does not agree with the communitarian response to liberal individualism, he nevertheless recognizes that "it is increasingly accepted in many countries that some forms of cultural difference can only be accommodated through special legal or constitutional measures, above and beyond the common rights of citizenship" \parencite[26]{Kymlicka1996}.

In the lead up to the Charlottetown Accords, the Canadian government argued for a Canadian exceptionalism that balanced individual and collective rights:

\begin{quote}
From its beginnings - in democracy, freedom and the rule of law - Canada has developed its own unique way of governing, its own special relationship between citizen and the state. Whether out of genius or necessity, the architects of Canada provided a framework which has allowed us to build a country on the basis of what appear increasingly to be universal values - freedom, equality, compassion and community - in a distinctly Canadian way. \parencite[2]{GovernmentofCanada1991}
\end{quote}

The justification for constitutional reform, the Government argued, was not only the historical changes which had taken place since 1867 but the challenges to Canadian citizenship posed by Indigenous peoples and Quebec, challenges which the universal individual citizenship supported by Trudeau had been unable to deal with. The Government frames Indigenous and Quebecois demands in terms of recognition: "Aboriginal Canadians are frustrated by a Constitution that does not fully recognize their special place in the Canadian society... Canada must also address Quebec's desire for recognition of its distinct nature..." \parencite[vi]{GovernmentofCanada1991}. Will Kymlicka notes that "underlying much liberal opposition to the demands of ethnic and national minorities is a very practical concern for the stability of liberal states"\parencite[192]{Kymlicka1996} and it is significant that the Government's politics of recognition must take for granted both a "Canada" and a Constitution.

Recognition can only be the accommodation of difference within these constraints: 

\begin{quote}
Being Canadian does not require that we all be alike. Around a core set of shared values, Canadian citizenship accommodates a respect of diversity that enriches us all... In the Canadian experience, it has not been enough to protect only universal individual rights. Here, the Constitution and ordinary laws also protect other rights accorded to individuals as members of certain communities. This accommodation of both types of rights makes our constitution unique and reflects the Canadian value of equality that accommodates difference. \parencite[1-3]{GovernmentofCanada1991}
\end{quote}

Published on the eve of constitutional reforms requiring national agreement, there is an element of propaganda to the government's insistence on Canada's unique commitment to communitarian freedom. Nonetheless this document provides a useful illustration of the hegemonic perspective on Canadian government, a post-Trudeau departure from universality not only maintained by the federal government itself but one it expected to resonate with a majority of Canadians. 

But the image presented in this document was a false one. In the early 1990s, two crises highlighted the continued dissatisfaction of Indigenous peoples and Quebec in ways that went beyond the recognition of difference, and undermined the federal government's picture of Canada as a place of tolerance and accommodation. The Kanehsatà:ke resistance in 1990 and the second Quebec referendum on independence surround the theoretical work on the politics of recognition just as much as the immediate question of Constitutional reform.

Part of the traditional territories of the Kanien'kéha:ka (Mohawk) nation, Kanehsatà:ke in what is currently Quebec has been occupied by French settlers since the early 18th century and is the subject of a longstanding land claim against the Canadian government. In 1990, the town of Oka, Quebec, which surrounds Kanehsatà:ke, approved the expansion of a local golf course without consulting the Kanien'kéha:ka, to whom that land was sacred. Despite formal protest, the development pushed ahead until the Kanien'kéha:ka erected a barricade to prevent further development. After several injunctions and an attempt by the Kanien'kéha:ka to gain a moratorium on development, the Quebec provincial police force, joined by members of the Canadian Army and the Montreal police, stormed the barricade, resulting in the death of a police officer and provoking a 78-day standoff. On September 26, 1990, the Kanien'kéha:ka surrendered and the golf course expansion was halted. The Canadian government purchased the land, but it has not yet been returned to the Kanien'kéha:ka and the prior land claim is still unresolved (see \cite[116]{Coulthard2014}).

After the failed referendum in 1980, Quebecois separatism continued to build steam. Quebec engaged in two rounds of Constitutional discussions (the Meech Lake and Charlottetown Accords) but neither round successfully managed to enshrine Quebecois recognition as a distinct society. By the mid-1990s, separatism was once again a powerful force in Quebecois politics. In a second referendum in 1995, the separatist side lost by the slimmest of margins (49.4\% voted in favour of sovereignty, 50.6\% in favour of remaining in the confederation). This referendum provoked a great deal of soul-searching in Canada and Quebec, and while the separatist movement lost steam after 1995, the problems of Quebec's cultural distinctiveness remain important considerations in Canadian politics.

In the aftermath of the Quebec referendum Taylor linked the question of citizenship with the protection of democracy: "In democratic countries, citizenship is more important than cultural identity, precisely because it is an essential component in preserving democracy" \parencite[248]{Ancelovici1998}. Like other post-Rawls liberals, Taylor sees democracy in procedural terms (as a democratic process), remarking that "Being citizens of the same country means we share something in common, that we are linked by a basic identity and are able to coexist and form a democratic electorate". 

This proceduralism - common to both liberalism and communitarianism - marks liberalism's origins as the ideology of capitalism and the domination of nature through instrumental reason \parencite{Popowich2020c}. The "unity of individuals" of the electorate which lies at the heart of the liberal-communitarian debate is a disagreement over what amount of homogeneity is required for the procedural administration of society. Communitarians like Taylor argue that the total quantification of human social relations - derived from utilitarianism - should be rejected in favour of the protection of forms of "strong evaluation" required for real human agency \parencite[17]{Taylor1985b}. Neoliberalism, on the other hand, has paved the way for the real "procedural republic" of algorithmic \parencite{Parisi2015}, surveillance \parencite{Zuboff2019}\footnote{While the phrase "surveillance capitalism" has been popularised by Shoshana Zuboff in a defense of liberal principles, the term originated in an article by ecological Marxist John Bellamy Foster and media critic Robert W. McChesney in 2014 \parencite{Foster2014}.}, and platform capitalism \parencite{Srnicek2016}. Legislation, bills of rights, constitutions, even the Charter of Rights and Freedoms plays into this proceduralism by attempting to circumvent debate and struggle through the creation of sovereign, transhuman laws (see \parencite[84]{Hall2021g}. Constitutions, in this sense, can be understood as pre-computational algorithms for the government of societies.

It is in this procedural sense that proclamations of values, or the upholding of principles like free speech or Intellectual Freedom are useful to the maintenance of hegemony. The value-free "thin theory of the good" that informs the capitalist society of negative liberty portrays free speech/free expression and other goods as substantive rights, but their real value is in the creation of a procedural code which seeks to peacefully disarm and circumvent direct action, struggle, and disagreement. In Jacques Rancière's formulation, even charters of rights and freedoms become instruments of "police" as an alternative to real politics: in a divided society, "there is only the order of domination or the disorder of revolt" \parencite[12]{Ranciere1999}. What links liberals like Dworkin with communitarians like Taylor and Tully, and separates them from any real radical politics, is exactly the preference for the order of domination over the disorder of revolt \parencite[6]{Ranciere1999}. The basic unit of liberal proceduralism is the individual with equal rights and the agency to enter into contract. In the next section, we will see how Taylor's critique of modern individualism does not go far enough to displace this individualism from its central position within political theory.

In the next chapter we will look more closely at the individualistic basis of Taylor and Tully's communitarian politics. 