\begin{quote}
The hope of every ideology is to naturalize itself out of History into Nature... \autocite[8]{Hall1988}
\end{quote}

\section*{Introduction}

In this chapter I want to investigate traces: of social contract individualism  on communitarianism, of Canadian politics on Canadian political of recognition, and on the social and political conjuncture into which individuals are born that leave an inventory of traces in their lives and their subjectivity... 

In the last chapter, we traced Western political history from the crises of the late 1960s through to the development of neoliberalism in the 1970s, and Canadian politics from the resurgence of Indigenous and Quebecois activism in the 1970s to the 1990 Kanehsatà:ke resistance of  and the second Quebec referendum of 1995. During this period, a pragmatic politics of recognition was adopted by the Canadian federal government in its dealings with Indigenous peoples and Quebec, and informed the constitutional process of the 80s and the two failed amendment attempts in 1987 and 1992. 

In this chapter, I will show how the politics of recognition, far from being a pure political theory, is in fact deeply marked by concrete Canadian political history of the 1970s, 1980s, and 1990s.  Interventions in the politics of recognition by Charles Taylor and James Tully appeared in the first half of the 1990s, a period when Kanehsatà:ke and the Quebec referendum exposed deep faults within Canadian politics, particularly with the liberal universal egalitarianism espoused by Pierre Trudeau. Trudeau had led the patriation of the Constitution and the development of the Charter of Rights in 1982, but tried to derail the Meech Lake Accord in 1987, because in his view it granted too much to Quebec's sense of difference. 

The constitutional process of the 1980s and early 90s had tried to resolve the contradiction in Canadian multinational federalism, and the failure of that resolution - both in terms of the Accords themselves, and the challenges posed in Kanehsatà:ke and Quebec  - provided the context for Taylor's and Tully's attempts to develop a political philosophy as a way past the aporiae of Canadian politics.

While the politics of recognition had been a practical response to Indigenous activism and Quebecois nationalism in the 1970s, in the 1990s Taylor, Tully, and others attempted to legitimate it as a political philosophy. The politics of recognition thereby serves a real political and ideological function in the shoring up of communitarian liberalism as the dominant Canadian approach to questions of individualism, community, rights, and the social order, and in trying to ensure the continued stability of the Canadian state itself.

The politics of recognition is not drastically in need of more critique. Since its inception recognition dominated the "politics of indigeneity" as well as related questions of multiculturalism and national subgroups within federal states. However, Robert Nichols has noted that recognition "has come under such sustained critique for many decades"\autocite[188n59]{Nichols2020} that more theoretical critique is unnecessary. 

From the perspective of this thesis, two important critiques are those of Glen Sean Coulthard \autocite{Coulthard2014}, who argued that Taylor's politics of recognition did not go far enough to overturn the settler-colonial politics of dispossession and oppression, and of Nancy Fraser \autocite{Fraser2003}, who argued that recognition meant nothing without real material redistribution. Both of these ideas will reappear throughout this chapter. My own critique is focused on the legacy of social contract individualism in this communitarian variant of liberalism, and the political and ideological role this legacy plays when read against real political developments.

What I want to focus on is a novel way of reading Taylor and Tully as interventions in concrete crises of Canadian politics. In hindsight, Taylor's "'Multiculturalism' and the Politics of Recognition" (1992/1994) and Tully's \textit{Strange Multiplicities} can be read as cultural representations of concrete political problems. Such a reading conforms to Said's contrapuntal methodology which, following Gramsci, sees "the filiation between forms of cultural expression and political action" and that subsequently understands "cultural production as a mode of political struggle" \autocite[434]{Harootunian2005}. It also conforms to Hall's method conjunctural analysis. A given conjuncture, Hall argues, also following Gramsci, gains a particular texture from the fact that political crises often last for decades, in which time 

\begin{quote}
the political forces which are struggling to conserve and defend the existing structure itself are making efforts to cure them within certain limits, and to overcome them. These incessant and persistent efforts (since no social formation will ever admit that it has been superseded) form the terrain of the “conjunctural”, and it is upon this terrain that the forces of opposition organise. \autocite[178]{Gramsci1971}
\end{quote}

I read Taylor and Tully, then, as participating in this conjunctural struggle, as engaging in cultural production as political intervention against the backdrop of Canadian constitutionalism, Indigenous resistance, and expressions of Quebecois nationhood. Taylor and Tully are not, therefore, engaged in "pure" political philosophy, as their texts would suggest, but are in fact "struggling to conserve and defend the existing structure itself... within certain limits". These limits, as I will show below, are the limits of liberal individualism itself. 

[Chapter Outline]

\section*{Rawls' Response to the Late 1960s Crisis}

As we have seen the effects of 1968 resonated throughout the world of radical left political theory, making space for new theoretical and practical strategies appropriate to what would become the neoliberal conjuncture. The crises of the late 1960s challenged liberalism as well, and we can trace one such challenge in John Rawls' position on the Vietnam War and his attempt to deal with the issues it raised in his \textit{Theory of Justice}.

In the wake of social energies unleashed in the late 1960s, particularly the civil disobedience of Vietnam War protest, Rawls sought to reconcile the individualism of social contract theory with the recognition of social bonds, relations, and obligations. The origin of such recognition is often ascribed to Hegel \parencite[Chap. 2]{Honneth1995}\footnote{In an article comparing the communitarian critique of liberalism and the social theory of Marx and Hegel, Sean Sayers notes that while "much liberal social thought starts from the assumption that the individual is an atomic entity, 'unencumbered' by any necessary social relations... [b]oth Marx and Hegel reject this approach" \parencite[85]{Sayers2007}.}, but Rawls draws on Hegel only to mention that Hegel supported a departure from strict equality to further social and political good \parencite[264-265]{Rawls1999}. However, two later adherents of communitarianism - specifically in the form of the "politics of recognition" - rely directly on Hegel's "inherently nonindividualistic" \parencite[46]{Geuss2005} social philosophy.

To perform the contrapuntal reading suggested by Said, we have to connect the debates within liberalism with the actual social and political processes we traced in the last chapter\footnote{Here I follow Raymond Geuss' prescription for realism in political theory, that is, philosophy "centred on the study of historically instantiated forms of collective human action" \parencite[22]{Geuss2008}.} Thus, despite the often abstract and de-historicized framing of liberal political theory, we must recognize in the work of subsequent liberal theorists not merely a response to Rawls but to the individualistic, consumerist, neoliberal culture that was being developed in the 1970s (what Tom Wolfe called the "me decade"\cite{Wolfe1976}). 

Rawls' theory of justice as fairness tried to balance the equal and universal distribution of individual rights with a "failsafe" mechanism in the form of the difference principle. The difference principle argues that a departure from strict equality is justified if it benefits the least-advantaged members of society, thus recognizing that social relations play a role in individual flourishing. In other words, Rawls attempted to balance the competing or conflicting demands of collective and individual justice that marked the late-1960s "resurgence of the people".  However, Rawls' attempt at balance inscribed an ambiguity at the heart of his theory which opened the doors for the split between "liberalism" and "communitarianism".  

Democracy, for Rawls, "is arrived at by combining the principle of fair equality of opportunity with the difference principle" \autocite[65]{Rawls1999}. The difference principle recognizes that social and economic inequalities exist, and that "the social order is not to establish and secure the more attractive prospects of those better off unless doing so is to the advantage of those less fortunate" \autocite[65]{Rawls1999}. Rawls thus tries to reconcile equality (of opportunity) with structural difference: individuals may have equal opportunities to succeed, as in social contract theory,  classical liberalism, and utilitarianism but they do not have equal starting points - not everything is left up to individual agency and power.

The ambiguity consists precisely in the relation of individual to society in Rawls theory. By mitigating the pure individualism of classical liberalism and utilitarianism, Rawls opens the door to the communitarian critique of liberalism itself. After Rawls, individualists like Ronald Dworkin and libertarians like Robert Nozick sought to reinforce strict and universal individual equality while communitarians like Charles Taylor and James Tully investigated the ways departures from universal equality could be used to achieve a more just and democratic society. However, both Taylor and Tully remain bounded by the liberal horizon, in which the difference principle is subservient to the principle of equality of opportunity; this prevented them from taking what I will argue is a necessary further step in order to achieve real social justice.

The ambiguity in Rawls' theory of justice also led to Dworkin's formulation of two theories of equality - important to an understanding of the politics of recognition - in his 1978 essay on liberalism. For Dworkin, drawing on Isaiah Berlin's conceptualization of "negative liberty" \autocite{Berlin1969}\footnote{For a critique of negative liberty, see \autocite{Taylor1985a}.}, the individualist/proceduralist form of equality required that states remain agnostic as to any particular conception of the good (cf. Rawls' "thin theory of the good" \parencite[347-348]{Rawls1999}) and guarantee individual rights through the equal application of neutral or objective procedures. A second form of equality - which Dworkin rejects, but which forms the basis of Taylor and Tully's politics of recognition - saw states commit to some collective goals or goods and implement policies to achieve those goals \parencite{Dworkin1978}. For Dworkin, as for Berlin, this "positive" concept of liberty and equality risked leading to totalitarianism. 

A re-evaluation of the relationship between individual and society has made urgent by the crisis of the late-1960s, which radically called into question the individual-society assumptions of the welfare state. The "resurgence of the people" of 1968 brought questions of communal belonging and identity politics to the fore in the form of civil rights, gay rights, women's rights, anti-colonialism, and other movements. 

Earlier, I argued that these expressions of communal belonging were a response to the universality of liberal egalitarianism in the postwar period. But another, radically individualistic tendency, also developed in the 1960s: a thoroughgoing individualism inherited from Romanticism. The anarchism of Robert A. Heinlein and the existentialism of John Fowles can stand as expressions of the search for anarchic individual freedom in the 1960s. The hippie movement was paradoxical in that it was a movement of radically free, unencumbered individuals who nevertheless shared a common aesthetic and set of values. Charles Manson is the end-result of this contradictory tendency in the 1960s: the radical individualist outsider/artist attacking the conformity of postwar society by leading a cult of almost interchangeable hippie children. The contradiction is expressed most clearly in the fact of children fleeing the stifling conformity of postwar suburban America searching for their own individuality and finding it in the hippie commune \autocite[page?]{Melnick2018}.

Rawls' theory of justice was an attempt, in part, to resolve these contradiction between individualism and communal belonging\footnote{For an discussion of Rawls' ambiguous position between individualism and communitarianism, see \autocite[239-269]{Forrester2019}. Forrester notes that "interpretivist communitarian originated in the same Wittgensteinian moment as Rawls' theory" despite the fact that Rawls' considered communitarian ideas as "tending towards conservatism" \autocite[258]{Forrester2019}.}. However, one of the main ideological planks of the neoliberal turn contemporaneous with Rawls' work was a return to the radical individualism of the Enlightenment. 

While individualism and individual rights were balanced in Rawls by a sense of social responsibility and collective belonging, his work provoked a polarization of debate between radical individualists and libertarians (like Ronald Dworkin and Robert Nozick) and communitarians like Taylor. While these debates were going on, however, neoliberal economics and politics were pursuing a radically new prioritization of the "possessive individual", reconceived now as an "entrepreneur of the self" \autocite[226]{Foucault2008}. Education and self-development (\textit{Bildung}) began to be seen as individual investments necessary for success in the new neoliberal conjuncture, rather than social investments leading to social amelioration. Thus, while the debate between libertarians and communitarians was taking place in the 1970s and 1980s, neoliberalism itself was "grounded in the 'free, possessive individual' with the state case as tyrannical and oppressive" \autocite[318]{Hall2011}.

On the terrsin of theory, the radical individualism of Dworkin and Nozick opened space for the communitarian version of liberalism. Forrester writes that "self-styled communitarians began to claim that liberalism misunderstood the social nature of the self and that the liberal subject created a hostile and alienating social order". What was new in the communitarian position was that "it was now joined with the interpretive turn against naturalism and liberal neutrality, and a localist, democratic of of Rawls" \autocite[252]{Forrester2019}.  This critique, however, was only partial, since there was a "family resemblance" between communitarianism and Rawls' theory of justice, and "communitarian critics left in place key aspects of the Rawlsian vision" \autocite[258]{Forrester2019}.

\section*{The Pre-Social, Natural Individual}

In the Canadian context, the tension between the universal individual egalitarianism of liberal thought (which Trudeau wanted to enshrine in the Constitution of 1982) and communitarian recognition of communal bonds and obligation played out in concrete political events, such as Kanehsatà:ke and the Quebec referendum, as well as within the politics of recognition itself. It is important to understand that communitarianism always recognizes some limits given to it by liberal individualism, which explains Trudeau's antagonism towards the post-patriation constitutional accords. Communitarian never goes as far, for example, as radical theory in adopting a position of social construction, let alone the denial of individual subjectivity as such, which can be found in some postmodern social theories. Indeed, the unwillingness or inability to engage with social construction in any form is one of the chief critiques of Intellectual Freedom we will look at in the next chapter. 

If "recognition" means seeing something that was already there\footnote{This is Judith Butler's starting point in \textit{Gender Trouble}, as when she writes that "on the one hand, \textit{representation} serves as the operational term within a political process that seeks to extent visibility and legitimacy to women as political subjects; on the other hand, representation is the normative function of a language which is said either to reveal or to distort what is assumed to be true about the category of women" \autocite[2]{Butler1990}.} The politics of recognition can never go as far, for example, as Judith Butler's constructed and performative conception of subjectivity, can never fully abandon the social contract roots of liberal individualism. For example, Tully argues that modern constitutionalism arose out of the the recognition of the "equality of independent, self-governing national states and the equality of individual citizens" \autocite[15]{Tully1995} as if states and individuals were natural facts about the world that simply need to be seen and recognized for what they are. Butler's critique of a liberal "common sense" insistence on natural (binary) sex applies equally to Tully's social ontology: 

\begin{quote}
The prevailing assumption of the ontological integrity of the subject before the law might be understood as the contemporary trace of the state of nature hypothesis, that foundationalist fable constitutive of the juridical structures of classical liberalism. The performative invocation of a nonhistorical "before" becomes the foundational premise that guarantees a presocial ontology of persons who freely consent to be governed and, thereby, constitute the legitimacy of the social contract. \autocite[4]{Butler1990}
\end{quote}

That social contract individualism forms the basis of modern liberal politics - including recognition - was supported not just by Rawls but by Habermas, who wrote in his response to Taylor's essay on recognition, that "modern constitutions how their existence to a conception found in modern natural law according to which citizens came together voluntarily to form a legal community of free and equal consociates" \autocite[107]{Habermas1994}. 

What Butler calls liberalism's "presocial ontology" is critiqued by Marx in the 1857 "Introduction", in which he writes that the utterly free individual was a socio-historical result rather than a natural starting point; individual subjectivity was constructed, not natural, and it was therefore something that had to be \textit{produced}, not just \textit{recognized} \autocite[84]{Marx1973}[21]{Hall2021f}. Individual equality is thus an ideological representation within a particular (bourgeois, liberal) signifying system\footnote{For ideology and representation in signifying systems, see \autocite{Hall2013}.}.

The politics of recognition in Taylor and Tully, then, attempts to moderate but not abandon the foundationalism of social contract individuals. Both argue for the recognition of cultural difference and the adoption of some form of collective rights, but fall short of adopting a full blown theory of social construction. Their individualism reaffirms the contractarian origins of recognition and places hard limits on collective action and social solidarity while maintaining an ostensible commitment to social justice\footnote{In his essay on "Atomism", Taylor critiques Robert Nozick's libertarianism by a different reconciliation of individual and social good than that offered by recognition. Taylor argues that even if autarkic individuals \textit{do} form the basis of a political ontology, the society which provides the best conditions for the \textit{Bildung} of those individuals must be the best. Therefore even libertarians must recognize a social horizon for their ultraindividualism. See \autocite{Taylor1985}.}.


\section*{Liberal Proceduralism: Citizenship and Equality}

In the aftermath of the Quebec referendum, Taylor linked the question of citizenship with the protection of democracy: "in democratic countries, citizenship is more important than cultural identity, precisely because it is an essential component in preserving democracy" (Taylor quoted in \autocite[248]{Ancelovici1998}). Like other post-Rawls' liberals, Taylor sees democracy in procedural terms (as a democratic process), remarking that "being citizens of the same country means we share something in common, that we are linked by a basic identity and are able to coexist and \textit{form a democratic electorate}" (emphasis added).

This claim may need some justification, as Taylor is often associated with the interpretivist perspective against the procedural. Katrina Forrester writes that "critics suggested that liberal philosophers' commitment to consensus, and their focus on distributional decisions and procedure rather than democratic control, implied a tacit acceptance of technocracy and inequality. Objectivity and impersonal modes of knowledge came to be associated with antidemocratic expertise. The language of interpretation and experience was joined to the critique of the liberal procedural state" \autocite[252]{Forrester2019}. Taylor falls into the camp of such critics, but he still insists on an objective sense of morality against which "strong interpretation" can take place.

For example, Taylor, along with Patrizia Nanz and Madeleine Beaubien Taylor recently wrote that "We believe that to restore responsible government we must reconstruct democracy from the bottom up. Only if we enhance and reinvigorate democracy at the base will the citizenry find clarity about what to ask for, or what future to envision for their community or region. Only then can local communities put pressure on their representatives in policy-making bodies to push for more courageous policies" \autocite[6]{Taylor2020}. Taylor is therefore a proponent of democratic participation, albeit limited by representational parliamentarianism. 

However, in \textit{Sources of the Self} Taylor argues that the "malaise of modernity" was occasioned by the turning away of objective sources of truth and morality towards an inner voice, leading to narcissism and relativism. There is an ambivalence, then, between individual citizenship on the one hand, and intersubjectivity and objective "strong evaluation" on the other hand. Far from indicating a contradiction within Taylor's work,however, I see this ambivalence as indicating the limit Taylor's thought. Taylor's communitarian politics is limited by an individualist social ontology, while his individualist ontology gestures towards intersubjective social relations (e.g the politics of recognition) without going as far as social construction. We will return to this point below.

The proceduralism common to both liberalism and communitarianism betrays liberalism's origins as the ideology of capitalism and the domination of nature through instrumental reason \autocite{Popowich2020}. The "unity of individuals" of the electorate that lies at the heart of the liberal-communitarian debate is a disagreement over what amount of homogeneity is required for the procedural administration of society: to put it in 21st century terms, the less difference there is between members of a society, the more amenable the society is to algorithmic command and control\footnote{The reason liberal individualism is foundational to proceduralism is that while it appears to be predicated on difference, liberal individualism - even libertarianism - presumes that all individuals are fundamentally the same (at least in the ways that matter). In \textit{The Racial Contract}, for example, Charles W. Mills notes that "whites will... take for granted... the appropriateness of concepts that \textit{derace} the polity, denying its actual racial structuring"  \autocite[95]{Mills1997}. Mills shows how the origins of this procedural, individualistic erasure of difference derives from classical liberal notions of equality (even as these preserve and extent oppressive structures of difference) \autocite[15-17]{Mills1997}.}. Debates within liberalism end up being about which differences are significant enough to cause procedural problems and which do not. 

Communitarians like Taylor argue that the total quantification of human social relations - derived from utilitarianism - should be rejected in favour of the protection of forms of "strong evaluation" required for real human agency \autocite[17]{Taylor1985}. Neoliberalism, on the other hand, has paved the way for the real "procedural republic" of algorithmic \autocite{Parisi2015}, surveillance \autocite{Zuboff2019}\footnote{While the phrase "surveillance capitalism" is associated with Zuboff's defense of liberal principles, it originated in a 2014 article by ecological Marxist John Bellamy Foster and media critic Robert W. McChesney \autocite{Foster2014}.}, and platform \autocite{Srnicek2016} capitalism. Legislation, bills of rights, constitutions, even the Charter of Rights and Freedoms plays into this proceduralism by attempting to circumvent debate and struggle through the creation of sovereign, transhuman laws (see \autocite[84]{Hall2021b}). Constitutions, in this sense, can be understood as pre-computational algorithms for the "governance" of societies\footnote{Hall writes that "the term 'governance' is itself another shifty New Labour concept: not a synonym for 'government' but the signifier of 'a new process of governing'" in which government is reduced to the application of economic logic" \autocite[305-6]{Hall2017c}.}.

Proceduralism is founded on radical individualism, and despite communitarianism's self-image, it remains buried in the heart of that project. It is in the procedural sense that proclamations of values, or the upholding of principles like free speech or Intellectual Freedom are useful to the maintenance of hegemony. The value-free "thin theory of the good" that informs the capitalist society of negative liberty portrays free speech/free expression and other goods as substantive rights, but their real value is in the creation of a procedural code by which any conflict can be resolved and which seeks to peacefully disarm and circumvent direct action, struggle, and disagreement. In Jacques Rancière's view, even charters of rights and freedoms become instruments of "police" as an alternative to real politics: in a divided society, he writes, "there is only the order of domination or the disorder of revolt" \autocite[12]{Ranciere1999}\footnote{In \textit{Hatred of Democracy}, Rancière argues that such order is brought about by the "'arithmetical' equality between equivalent units" \autocite[64]{Ranciere2006}; it is this arithmetical equality sought after by capitalist political economy that makes algorithmic governance possible.}.

In \textit{Multicultural Citizenship}, Will Kymlicka explained the practical political role played by liberal theory as having less to do with rights, individualism, or egalitarianism, but with order: "Underlying much liberal opposition to the demands of ethnic and national minorities is a very practical concern for the stability of liberal states" \autocite[192]{Kymlicka1996}. Recognition, then, can only be the accommodation of (some) difference within the constraints of liberal order founded on individualism\footnote{There are parallels here with Butler's argument in \textit{Gender Trouble} that traditional feminisms rely on an essentialist biological definition of sex as a foundation for any discussion of gender. Butler "troubles" the idea of a pre-social sex in the same way that radical politics troubles the idea of a pre-social individual. According to Janet Coleman it was the invocation of pre-social rights (e.g. to self-preservation) that "led to the widespread invocation by thinkers of a contract to establish society" \autocite[256]{Coleman1996} entered into by "non-social isolates who had to create politics as a convention" \autocite[10]{Coleman2000}.}. Indeed, while Kymlicka defends the liberal position against communitarianism, he notes that "many (but not all) of the demands of ethnic and national groups are consistent with liberal principles of individual freedom and social justice" \autocite[193]{Kymlicka1996}. 

By the time Taylor, Tully, and Kymlicka were writing about recognition and multiculturalism, however, Canadian politics had witnessed a resurgence of both Indigenous activism and Quebecois nationalism that threatened to destabilize the Canadian political order. The patriation of the constitution and the development of the Charter of Rights and Freedoms was intended to balance, recognize, and therefore defuse the tensions inherent in a polyethnic, multicultural, and multinational colonial federation. By the early 1990s it was clear that this was a false promise, and Indigenous social issues, Quebecois nationalism, and a host of other forms of inequality remain urgent problems of Canadian politics. When Kymlicka writes that "it is increasingly accepted in many countries that some forms of cultural difference can only be accommodated through special legal or constitutional measures, above and beyond the common rights of citizenship" \autocite[26]{Kymlicka1996}, the unspoken goal of this accommodation is to shore up and stabilize the liberal political and social order against such challenges. 

What links liberals like Dworkin with communitarians like Taylor and Tully, and divides all three from any real radical politics, is exactly the preference for the order of domination over the disorder of revolt. This preference places limits on the amount and kind of difference the liberal order can tolerate. The basic unit of liberal proceduralism is the individual with equal rights and the agency to enter into contract\footnote{Allegra de Laurentiis translates a phrase in Hegel's \textit{Philosophy of Right} as "person capable of property", making property ownership not just an activity but an intimate characteristic of liberal personhood \autocite[4]{deLaurentiis2005}.}. All other differences can either be tolerated as harmless or violently erased. Either way, a universal, homogeneous universality remains at the heart of the communitarian critique of liberalism. In the next section, we will see how Taylor's critique of modern individualism does not go far enough to displace universal equality from its central position within liberal theory. 

\section*{Taylor's Partial Critique of Individualism}

The politics of recognition arose in both its practical and theoretical forms, as we have seen, out of the need to take seriously the demands of collective identity and to balance them against individual rights. The "resurgence of the people" contained contradictory - sometimes paradoxical - expressions of individual self-definition and collective solidarity. These contradictions became more pressing as neoliberalism deepened its commitment to pure individualism while stamping out individual difference in the name of automation, productivity, and the reduction of risk.

The primary formulators of the politics of recognition, Axel Honneth and Charles Taylor, turned to Hegel to provide an alternative to the Kantian perspective and what Marx called the "Robinsonades" of liberal political philosophy (and in particular Rousseau's "Natural Man"). Charles Taylor connects Hegel's intersubjective position with the rise of a particular concept of human dignity. Hegel, in Taylor's view, "takes it as fundamental that we can flourish only to the extent that we are recognized" \parencite[5]{Taylor1994}\footnote{Bhambra and Holwmood (2021) however argue that Hegel's theory of recognition not only applied solely to a European subject insulated from actually-existing slavery, but that it was in fact used to justify slavery in colonized areas \autocite[46-50]{Bhambra2021}.}.

In \textit{The Struggle for Recognition}, Honneth describes the young Hegel questioning "the individual presuppositions of Kant's moral theory" then dominant, a questioning which eventually developed into "the conviction that, for the foundation of a philosophical science of society, it would first be necessary to break the grip that atomistic preconceptions had on the whole tradition of modern natural law" \autocite[11]{Honneth1995}.
	
Gadamer, in \textit{Truth and Method} (1960), sums up the Hegelian model of intersubjective individuality in terms very different from Kant's \textit{sapere aude}:

\begin{quote}
Every single individual who raises himself out of his natural being to the spiritual finds in the language, customs, and institutions of his people a pre-given body of material which, as in learning to speak, he has to make his own. Thus every individual is always engaged in the process of Bildung and in getting beyond his naturalness, inasmuch as the world into which he is growing is one that is humanly constituted through language and custom. \autocite[13]{Gadamer2013}
\end{quote}

Hegel, for Gadamer, is still beholden to an idea of naturalness, a pre-social individuality, but he goes further than Kant in recognizing that the process of intersubjective Bildung is unavoidable and inseparable from what we think of as individuality. Hall puts this idea succinctly when he remarks that through inculcation into pre-existing languages and cultures, children "become not simply biological individuals but cultural subjects" \autocite[8]{Hall2013}.

This is the essence of Taylor's partial critique of liberal individualism. For Taylor, individuals have an original identity which "needs and is vulnerable to the recognition given or withheld by significant others". Identity is thus a natural and original element of individuality which can nevertheless be "formed or malformed through the course of our contact" with others. 

Where I think Taylor weakens Hegel's position is that in Hegel the natural person is merely biological while in Taylor this becomes a pre-social individual. In Hegel, human individuality is \textit{produced} by an already-constituted human world of languages, culture, tradition, habit, etc. All that pre-exists it is the fact of biological individuation. But for Taylor, there is an additional pre-social "cultural subject" that can only ever be partially formed (or malformed) by social relationships. Taylor's critique of liberal atomism attempts to take on board \textit{some} of Hegel's intersubjectivity while still retaining the idea of a Kantian original identity at the heart of social and political relations. In this way, Taylor seeks to resolve the ambiguity at the heart of the Kantian view of individuality.

Taylor's view of recognition is founded on this resolution of the Kantian problem: the presence of an original identity means that for Taylor an individual approaches social and political life already being who they are\footnote{Again, this derives ultimately from the social contract idea of man in the state of nature}. For example, Taylor writes that:

\begin{quote}
The demand for recognition... is given urgency by the supposed links between recognition and identity, where this latter term designates something like a person's understanding of who they are, of their fundamental defining characteristics as a human being. The thesis is that our identity is \textit{partly} shaped by recognition or its absence, often by the misrecognition by others, and so a person or group of people can suffer real damage, real distortion, if the people or society around them mirror back to them a confining or demeaning or contemtible picture of themselves. Nonrecognition or misrecognition can inflict harm, can be a form of oppression imprisoning someone in a false, distorted, and reduced mode of being. \autocite[25, emphasis added]{Taylor1994}
\end{quote}

An individual, for Taylor, has "fundamental defining characteristics" which they must come to understand about themselves and which society must recognize. Identity is only partly formed by recognition, as opposed to the Hegelian view of the full social construction of identity. Taylor's challenge to atomism thus moderates, but does not discard, the originary individuality of the social contract and Kantian enlightenment.

The same criticism Butler makes of traditional feminism - that there is a pre-social, natural category of "woman" that is simply misrecognized or misrepresented by patriarchy - can be made of Taylor's social ontology: that there is a pre-social, natural individual whose reality is misrecognized. The political strategies entailed by both views is the same: to somehow "correct" representation or recognition in order for the natural, pre-social foundational subject to appear clearly and correctly. 

\bigskip

Taylor's early work focused on Hegel \autocite{Taylor1977b} and he drew on Hegel's master/slave dialectic in his thinking on intersubjective identity formation. Taylor was drawn to the critique of atomism in 1980s (the "greed decade") in the broader context of the liberal-communitarian debate. He perceived a "malaise of modernity" that he ascribed to a modern culture of corrosive individualism and narcissistic isolation \autocite[55-57]{Taylor1991}. In the monumental \textit{Sources of the Self} (1989) Taylor traced the philosophical lineage of this individualistic identity which he called the "disengaged self", cut off from traditional social sources of truth, authority, and legitimacy. In "Multiculturalism and the 'Politics of Recognition'" he attempted to account for multicultural challenges to liberal universality by proposing a less individualistic conception of identity as the basis for his communitarian liberalism. The ambiguity noted above can be clearly seen in Taylor's attempt to reconcile the recognition of individuality with common sources of "strong evaluation"; he found this reconciliation in the intersubjectivity of the politics of recognition.

Taylor contrasts his intersubjective, or dialogical, view with the monological, atomistic ontology that he argued arose over the course of the 18th and 19th centuries. The move toward inwardness and personal authenticity that Taylor ascribes to Rousseau and Herder made individuals the source of their own moral and social values, arising from their most authentic, independent, inward, and natural selves. Taylor claims that the dialogic or social nature of identity formation "has been rendered almost invisible by the overwhelminingly monological bent of mainstream modern philosophy" \autocite[32]{Taylor1994}\footnote{The term "mainstream" here automatically precludes Marxism, which is unfortunate, as not only has Hegel's social production of identity been a core concern of many Marxisms, but Taylor's choice of language as a synecdoche for the dialogic nature of identity was used by Marx in exactly the same sense in his critique of "Robinsonades" \autocite[84]{Marx1973}.

Taylor is dismissive of Marxism, writing that "the trouble with vulgar Marxisms is that, when they don't neglect [individual motivation] altogether and rely on some incomprehensible 'structural' determination which bypasses motivation altogether, their implicit picture of human motivation is unbelievably one-dimensional" \autocite[203]{Taylor1989}. The reference to structural determination is aimed at Althusser, but Taylor also rejects the rich, dialogical, even Hegelian humanist tendencies within Marxism. This supports my claim that Taylor's ontology only slightly moderates liberalism's total atomism while continuing to reject Hegelian social determination.}. Taylor rejects "historical explanation" (or "diachronic-causal story") \autocite[202-203]{Taylor1989}, seeing in the rise of modern individualism and 1980s culture of narcisissm nothing but the Hegelian development of ideas. In this way he obscures the tight linkages between political thought - including his own - and concrete, material social changes.

As in Kant, identity for Taylor is something original which we need the courage (authenticity) to enact in the world, but this authentic self relies on our intersubjective relationships to become fully formed in the world. Authenticity is then an intersubjective, social outcome; Taylor's view of identity attempts to reconcile individuality as origin and outcome, and he condemns modern "narcissism" as the latest form of liberalism's foundational atomism.

It is this original identity which must be respected and allowed to flourish through the process of recognition. Original identity can only be made "fully human" through social contact with other people: "we become fully human agents, capable of understanding ourselves, and hence of defining our identity, through our acquisitions of rich human languages of expression" \autocite[32]{Taylor1994}. Individual freedom (full human agency) comes out after the fact, through our intersubjective relationships and the acuisition of social languages that allow our pre-existing identity to be self-interpreted and recognized by others.

In this way, Taylor's philosophy begins with a social contract "natural individual" and ends with classical liberal individual freedom. Along the way our natural subjectivity is moderated and influenced through recognition and contact with others (i.e. through communitarianism), but the start and end points are purely liberal. In the end, Taylor protects the social contract view of individual identity and the liberal view individual freedom by accommodating social relationships in the middle of the process. This suggests that a rejection of social construction would have serious repercussions for any notion of freedom.

The influence of our significant others (through varieties of recognition) modifies, transforms, or deforms identity but does not constitute it. "My identity" is thus ontologically prior to (and separate from) "my dialogical relations with others". Thus, while Taylor challenges the atomism of modern society, he does not go as far as the Hegelian- or Spinozan-Marxist approach which sees individuals born into an already-existing set of social relationships and languages that \textit{produce} individuals. Taylor still subscribes to the dualism of the social contract and Mill's \textit{On Liberty} which see individuals and society ontologically distinct from each other. 

Taylor's dialogic approach to identity - and Tully's mediated constitutionalism - have been so influential in Canada because they share with classical and post-Rawlsian liberalism a presumption of individual identity, individual equality, and individual freedom while accommodating (or paying lip-service to) the communal perspectives of multicultural, multinational, polytethnic identity groups within a federal state. The politics of recognition thus plays an ideological role in legitimating and stabilizing Canadian politics against the destabilizing effects of "illiberal" groups such as Indigenous activists and Quebecois nationalists. 

The power imbalance between the Canadian state and any given identity group makes it easy for the federal government to "recognize and accommodate a range of group-specific claims without having to abandon their commitment to a core set of fundamental rights" \autocite[29]{Coulthard2014}. Taylor's politics of recognition provides a "slot" for group-specific claims in between core liberal ideas of natural individuality and individual rights and freedoms. However, as Coulthard points out, asymmetrical power relations, ignored by liberal and social contract individualism, undermine the process of mutual recognition as such, as witness the violent state repression at Kanehsatà:ke.  

Given the unequal power relationships between the Canadian state and minority communities, Taylor's presumption of a regime of equality falls apart. Coulthard remarks that these power dynamics make recognition something that is "ultimately 'granted' or 'accorded' a subaltern group or identity by a dominant group or identity" \autocite[31, 38]{Coulthard2014}. Stuart Hall makes the same point about rights in general in liberal philosophy, writing that liberalism ascribes to rights "a timeless universality, speaks of them as if they were 'given' rather than \textit{won} and as if they were given once and for all, rather than having to be constantly secured" \autocite[84]{Hall2021b}.

Rejecting the presumption of an equal society based on individual freedom requires us to see that colonized peoples (and other subaltern groups) can be discursively, symbolically, or legally \textit{recognized} while leaving the underlying power structures unchanged. Intellectual Freedom as an \textit{individual} right allows libraries to recognize individual Indigenous or trans people, for example, while rejecting any notion of structural racism or the libraries role in construction an anti-trans moral panic and the oppression of trans people. Libraries thus play the same legitimating ideological role as Taylor's politics of recognition, and draw on Taylor's philosophy (or rather, draw on the same ideological currents as Taylor's philosophy) to \textit{recognize} social inequity while doing nothing to ameliorate it. Nancy Fraser's critique of recognition can be summed up in her pronouncement that "justice today requires \textit{both} redistribution \textit{and} recognition" \autocite[26]{Fraser1999}; libraries perform the politics of recognition but stop at that.

The performance of recognition - recognition as a representational act - deepens the psychological effects of colonialism on colonized subjects, patrarchy on women, transphobia on trans people, as "the terms of recognition tent to remain in the possession of those in power to bestow on their inferiors in ways they deem appropriate" \autocite[39]{Coulthard2014}\footnote{We will see an example of this in the next chapter in Toronto Public Library's response to a trans library worker concerned about the platforming of a transphobic speaker.}. 

But in addition, recognition provides a mechanisms by which these subaltern groups can be \textit{represented} to the dominant society; the public act of recognition is always at the same time representation\footnote{Recognition and representation are intimately bound together as part of a single process: "Lordship was something publicly represented... representation pretended to make something invisible visible through the public presence of the person of the lord." \autocite[7]{Habermas1989}.}. Hegemonic values and interpretations become encoded in these representations even when, as in Tully, the represented groups are theoretically able to speak for themselves. Indeed, a constructivist social ontology would support the idea that there is no such thing as "speaking for oneself", there is only more or less contesting the languages in which we have been born and raised or in which we find ourselves after a geographic and cultural shift. In Hall's discussion of the media's role in the construction of racist discourse in the UK in the 1970s, he points out that the \textit{ways} recognition takes place constraints even those who are "speaking for themselves" to represent themselves and their views in terms set by the dominant society.

The Canadian approach is only a partial deviation from the universality and equality of liberal individualism. The prioritization of Rawls' difference principle in Canada institutes - in theory - a regime of cultural accommodation, toleration, and the mutual recognition of equals, enshrined for example in the Multiculturalism Act of 1985. This is the political culture which produced both the Charter of Rights and Freedoms and the Intellectual Freedom regime specific to Canada; a culture that attempts to balance equality and difference \textit{within} a stable overarching regime of liberal rights. 

Taylor is explicit in his defence of the communitarian balancing of identity and difference, of a commensurability between liberalism and a substantive commitment to social good. Referring specifically to Quebecois nationalism, Taylor writes:

\begin{quote}
A society can be organized around a definition of a good life, without this being seen as a depreciation of those who do not share this definition. [...] A society with strong collective goals can be liberal provided it is capable of respecting diversity, especially when dealing with those who do not share its common goals; and provided it can offer adequate safeguard for fundamental rights. \autocite[59]{Taylor1994}\footnote{Cf. Kymlicka above. For Canadian liberals and communitarians, there is no fundamental obstacle to the reconciliation of individual and communal rights, so long as some fundamental rights are considered sacrosanct. Kymlicka also writes that "group representation is not inherently illiberal or undemocratic. It is a plausible extension of our existing democratic traditions, and there may be some circumstances where it is the most appropriate way to ensure an adequate voice for minority interests and perspectives \autocite[151]{Kymlicka1996}. For Kymlicka, departures from the principle of equality must remain the exception rather than the rule.}.
\end{quote} 

This provision requires that even a polity like Quebec, where strong collective goals around identity and culture are reflected in public policy, maintain a set of inviolable individual rights equally and universally distributed. The substantive protection of Quebecois culture through legislation is an example, in Taylor's view, of a society with strong collective goals that remains liberal\footnote{Recent Quebecois legislation seen as Islamophobic by the rest of Canada are defended by some Quebecois as upholding a Catholic Quebecois identity, and therefore are no indication of structural racism within Quebec.}. However, as the Kanehsatà:ke resistance shows, there will always remain limits to liberal tolerance of collective action and a substantive or illiberal theory of the good\footnote{Indeed, this is the point of Popper's paradox of tolerance in \textit{The Open Society and Its Enemies}. Canadian society avoids the paradox of tolerance by being selective about what kind of illiberality they tolerate and what kinds must be stamped out.}. What I am arguing here is that this limit is often drawn precisely where individual vs. collective (or natural vs. socially constructed) identity meet, for the purposes of ideological construction and reproduction.

The fundamental rights a society would have to uphold in order to remain liberal while subscribing to a substantive (or thick) idea of the good are, for Taylor, the traditional individualistic ones  of possessive liberalism: "rights to life, liberty, due process, free speech, free practice of religion, and so on" \autocite[57]{Taylor1994}.

By moderating but not rejecting liberal individualism, Taylor's politics of recognition draws a distinction between what are (for him) properly political issues - rights and immunities which must be universally and equally applied - and what might be considered personal or cultural. We can think here of Trudeau's famous defense of the decriminalization of homosexuality in Canada in 1969: "there's no place for the state in the bedrooms of the nation".

But this distinction means, in fact, that only \textit{cultural} (or "lifestyle") differences - not, say, economic ones - can be accommodated by a communitarian liberal society. Taylor's attempt to reconcile individual liberty with intersubjective identity formation requires that he retain, if only a diminished sense, a foundational pre-social natural individuality. If Indigenous socio-economic life challenge the mode of production liberalism was designed to defend and support, transgender people challenge the foundationalism of a pre-social "authentic" self which, despite his criticism of how "authenticity" has become "narcissism" in the modern world, Taylor retains. Indigenous and trans people in Canada are model representatives of destabilizing, disruptive, and illiberal challenges to liberal, and are therefore well-positioned as demonized Others in Canadian social life. 

Indigenous ideas around land ownership, stewardship, and resource extraction are not cultural differences to be merely recognized, but are irreconcilable conflicts between traditional Indigenous relationships to the land and capitalist extractive ones. "Cultural" aspects of Indigeneity can be safely recognized, while land claims, pipeline protests, and other "non-cultural" expressions of Indigenous sovereignty must be violently repressed.

Transgender people, by "troubling" the concept, first of natural binary gender and then of binary sex itself (and therefore all the social, economic, and political divisions based on sex), challenge one of the few remaining foundationalisms of liberal social ontology. Trans people by the mere fact of their existence threaten the idea of a natural self that can be revealed by the law and recognized by society, and by extension threaten, as we have seen, the very notion of consent in liberal politics. Both Indigenous people and trans people can very easily be represented as existential threats to the natural (liberal) order of private property, wealth creation, the nuclear family and appropriate gender roles, thus participating directly in ideological forms of social and political control of the "silent majority" of Canadian citizens.

Taylor's politics of recognition protects the "fundamental defining characteristics" of settler-colonial capitalist society while allowing a safe, non-threatening measure of cultural difference to be recognized. It is the perfect political philosophy for a polity that seeks to differentiate itself from American fundamentalism - offering what Justin Trudeau has called the "sunny ways" of Canadian tolerance and prosperity - while continuing to pursue its own projects of capitalist development and imperialism (see for example \autocite{Shipley2018}).

Taylor identifies habeas corpus as one of the non-negotiable rights of liberal society. In the same way that a policy which denied the universal and equal application of habeas corpus could not be tolerated as an aspect of cultural difference or social distinctiveness\footnote{In fact, the application of habeas corpus in Canada as not as universally applied as Taylor suggests, particularly for disabled people. See for example "The Habeas Corpus of Justin Clark" \autocite{McPhedran2018}.}, so a policy which challenges the underlying economic requirements of racial capitalism, or the prerequisite consent of the social contract, is also dismissed as out of bounds in Taylor's politics. The limits of real progressive transformation are firmly drawn within the politics of recognition.


\section*{Tully's Critique of Constitutionalism}

The presumption of universal individual equality, as we saw in Coulthard's critique of Taylor, denies the reality of power imbalances in liberal society. The ostensible freedom and equality of every individual erases all questions of power and oppression, reducing inequality to procedural error easily fixed through technology or policy tweaking. Recognition on its own becomes an idealist substitute for a material reckoning with inequality and domination. Recognition becomes recuperated to the everyday politics of racial-capitalist domination, including Indigenous dispossession, racism, and sex- and gender-based oppression.

On the face of it, James Tully's application of the politics of recognition to constitutional questions should at least make these questions of power explicit. And while Tully does admit that "the basic laws and institutions of modern societies, and their authoritative traditions of interpretation, are \textit{unjust} \autocite[5]{Tully1995}, he does not see them as dehumanizing or oppressive precisely because he sees constituent groups as primordially free and equal before entering into a consensual constitutional relationship. He treats collectivities as a social contract theorist treats individuals, ignoring the fact that, as Hall put it in terms of UK race relations, such relations "are not... simply the relations of the immigrant community to the host community [but] are the mutual interrelations of both groups" \autocite[41]{Hall2021a}\footnote{Indeed, Tully makes the mistake Hall ascribes to some socialist critics of popular culture who posit an autonomous and authentic working-class popular culture whose participants are not taking in by capitalist commercial cultural production. Hall writes that the problem with his theory is "that it neglects the absolutely essential relations of cultural power - of domination and subordination - which is an intrinsic feature of cultural relations". Hall rather asserts "on the contrary that there is \textit{no} whole, authentic, autonomous 'popular culture' which lies outside the field of force of the relations of cultural power and domination" \autocite[186-187]{Hall2002}.}. 

Post-contact Indigenous peoples cannot be considered as free, independent social groups entering into constitutional discussions as equal partners: the colonialist legacy of oppression, dependency, and genocide cannot be ignored for the sake of constitutional purity. Consent is impossible in conditions of inequality, power imbalance, and ongoing oppression\footnote{While the treaties signed in the US, Canada, and New Zealand \textit{appeared} as relations between equal parties, Robert Nichols writes that "consent was legible only as assent to this system of self-extinguishment" \autocite[50]{Nichols2020}. Bhambra and Holmwood remark that "the consolidation of European rule over indigenous territories occurred through the making and breaking of treaties as much as through direct conquest and dispossession... the treaties that the United States made continued to be broken and remade" \autocite[67]{Bhambra2021}.}. 

There is a curious ambiguity in the way Tully describes contemporary constitutionalism, made even more striking when we remember the context of failed constitutional reform and Indigenous and Quebecois activism. On the one hand, "the constitution, which should be the expression of popular sovereignty, is an imperial yoke, galling the necks of the culturally diverse citizenry, causing them to dissent and resist", but the solution to this problem lies in "requiring constitutional amendment before they can consent" \autocite[5]{Tully1995}. The procedure of constitutional reform, for Tully, takes priority over dissent from and active resistance to an imperial yoke.

The heart of the ambiguity lies in the limit placed around acceptable cultural recognition. Injustice, in this view, arises solely from non- or misrecognition rather than, say, genocidal extermination. Justice can then only be accomplished through true (or corrected) recognition rather than real redress, redistribution, or transformation. We will see in the next chapter how this distinction maps onto the strict division between speech and action: speech, which is peaceful, is the only legitimate response to the violence of colonial occupation.

In Tully's constitutional theory, the only avenue to social change is more speech, more discourse, more procedure. Tully thus explicitly avows what was only implicit in Taylor's politics of recognition. By calling for a "politics of \textit{cultural} recognition", Tully makes cultural diversity, rather than real inequality, "a characteristic constitutional problem of our time" \autocite[2]{Tully1995}. Tully's focus on "intercultural demands" has the effect already noted of circumscribing legitimate targets of constitutional (or more broadly political) challenge and reform. Rather than dealing with questions of Indigenous land rights, Murdered and Missing Indigenous Women and Girls (MMIWG), residential schools, Islamophobia, or the real material grievances of francophone Quebecois, Tully's politics would limit intercultural demands to

\begin{quote}
schools and social services in one's first language, publicly supported TV, film and radio, affirmative action, and changes in the dominant curricula and national histories so that they respect and affirm other cultures, to the right to speak and act in culture-affirming ways in public institutions and spheres. \autocite[2]{Tully1995}.
\end{quote}

All of these are important, but by limiting diversity and recognition to these areas and practices, real routes to justice, truth, and reconciliation are cut off. Because Tully, like Taylor, wants to protect certain sacrosanct elements of the modern liberal state and its consitution, he places out of bounds real questions of redistribution, power, equality, even survival. Legitimate demands cannot be political or economic, but are limited to the sphere of culture and the demand for recognition becomes circumscribed and recuperated to the "peaceful" settler-capitalist status quo. In addition, as noted above, recognition itself becomes a performative or rhetorical smokescreen obscuring real intolerance and discrimination in Canadian society, not least against Indigenous and trans people.

When Tully writes that "the struggle of the Aboriginal peoples of the world, and especially those of the Americas, for cultural survival and recognition are a special example of the phenomenon of the politics of cultural recognition", the term "cultural survival" - while important - erases the issue of \textit{real} survival in the face of genocidal government policies, racism, or sex- and gender-based violence. 

In the previous chapter, I touched on the idea of "cultural genocide" as used by Harold Cardinal in his attack on the assimilationist 1969 White Paper and used again in the Truth and Reconcilation Report of 2005. In the wake of the discovery of unmarked graves of Indigenous children on the sites of the former Kamloops and Marieval residential schools and elsewhere in mid-2021, Indigenous people in Canada have called for an admission that the colonization of Turtle Island was not merely \textit{cultural} genocide but genocide \textit{tout court} \autocite{Qaqqaq2021}. In this context, Taylor and Tully's politics of (cultural) recognition appears less progressive than it perhaps did in the 1990s and more and more like a bromide for disaffected groups in Canada that nonetheless still defends the structures and traditions of white, settler-colonial power. 

Read against the real course of Canadian political history, Taylor and Tully's politics of recognition can be understood in terms of what Robert Nichols has called "recursive dispossession" \autocite[117]{Nichols2020}. This is the paradoxical effect of Western legal theory needing Indigenous peoples to have legitimate (i.e. Western legal) land title before they can be dispossessed of that land. Recursive dispossession grants Indigenous land title in colonial law at the moment that it strips that title away.

I am arguing here that Taylor and Tully's idealist prioritizing of recognition does on the theoretical level what Canadian politics was already doing on the concrete level: setting up and maintaining the recursive dispossession of self-determination by "recognizing" cultural difference at the same moment that inequality becomes institutionalized. For example, the Calder decision that acknowledged that Indigenous land claims pre-date colonial law did so precisely through recognition \textit{by} colonial law. It thus recognized Indigenous sovereignty at the same moment that it subordinated sovereignty to Canadian law\footnote{There are connections here to Antonio Negri's contention that the act of creating a constitution automatically constrains and limits the supposedly sovereign freedom of the democratic constituents themselves.}. 

Seeing recognition as part of the process of recursive dispossession allows us to understand how, when faced with "fierce material and ideological resistance beyond the narrow confines of staid academic debate" \autocite[110]{Nichols2020}, the politics of recognition is able to place that resistance out of bounds, making it impossible to move from recognition to redistribution or to an actually anti-oppressive, liberationist politics.

Admittedly, Tully sees cultural recognition as only the first step towards dealing with other pressing social, political, and economic problems \autocite[6]{Tully1995}, but this has the effect of postponing these questions until such time as constitutional recognition has satisfactorily occurred\footnote{In \textit{Confronting the Democratic Discourse of Librarianship}, I argue that a seemingly infinite number of lives and amount of time allow liberals to indefinitely postpone real social change. \autocite[293ff]{Popowich2019}.} In 1974, Hall described the UK media's "liberal-consensus assumption that we are all proceeding, slowly but inevitably, towards a racially-integrated society" as "for blacks, a highly problematic question" \autocite[53]{Hall2021c}.

The postponement of real social transformation until a perfect constitutional order is achieved reaffirms the priority of speech over action. Additionally, the history of (failed) Canadian constitutional reform suggests that this perfect constitutional order is not in any sense imminent. Meanwhile, anti-trans and anti-Indigenous violence remain rampant in Canadian society.

It is clear that what Tully values above all is social peace. In the conclusion to \textit{Strange Multiplicity}, he argues that the constitutionalism he proposes "offers a mediated peace":

\begin{quote}
A mediated peace \textit{is} a just peace: just because it is a constitutional settlement in accord with the three conventions of justice [mutual recognition, continuity, and consent] and peaceful because the constitution is accommodated to the diverse necks of those who agree to it. If this view of constitutionalism came to be accepted, the allegedly irreconcilable conflicts of the present would not have to be the tragic history of our future. \autocite[211]{Tully1995}.
\end{quote}

It is difficult to see how the constitution can realistically achieve this kind of mediated peace. Tully himself point out how the "presumed, culture-blind liberal constitutionalism" of the Charter of Rights and Freedoms "rather than uniting the citizens on a constitution that transcends cultural diversity... has fostered disunity. The province of Quebec, the Aboriginal peoples, women and the provinces resisted it at various times as the imperial imposition of a pan-Canadian culture over their distinct cultural ways" \autocite[7]{Tully1995}. What Tully does not seem to envision is a situation in which the constituents of the Canadian confederation do not want \textit{any} yoke, even if it is "accommodated to the diverse necks of those who agree to it". Indeed, the metaphor of the yoke calls into question the very possibility of agreement or consent. 

Tully does not think we can do away with the yoke entirely; and presumes that at some point Indigenous peoples will consent to be bound by it. He does not consider the possibility that Indigenous peoples may reject the very idea of a constitutional yoke entirely.

Tully, then, essentially repeats in constitutional terms the liberal view of capital-labour relations: that the legally free worker consents to sell their labour-power and is free to withhold it. Workers, of course, are \textit{not} free to withhold the sale of their labour-power, separated as they are from the means of material subsistence. The same is true of the various cultural groups that make up the Canadian confederation. What does Tully think will happen if any of the various groups do not consent to the yoke? Indigenous people, for example, cannot secede without a battle over Indigenous land - the very opposite of Tully's mediated peace. Anyone who does not consent to the constitutional yoke must conform to it or emigrate. Only the persistent myth of the social contract allows Tully to think there can be consent to modern constitutionalism in any meaningful sense.

Mutual recognition implies a material equality, independence, and self-sufficiency that does not exist under settler-colonial, racial capitalism. Just as the worker is separated from the means of subsistence, Indigenous people are divided from their land and their ways of living. What Tully wants, what he sees as the \textit{Spirit of Haida Gwaii}, is for politics to proceed peacefully in accord with liberal proceduralism and his three constitutional conventions. While the passengers in Canadian confederation "vie and negotiate for recognition and power... they also never fail to heed what is said by the chief whose identity has remained a mystery until this moment. She or he is the mediator" \autocite[212]{Tully1995}.

Without real economic redistribution or social equality, it is hard not to read Tully's constitutionalism as anything other than a description of Lockean sovereignty proper to the Canadian liberalism of the 1990s. The recognition of "Indigeneity" - often reduced, as Nichols points out, to a "fixed, temporally frozen cultural substance" \autocite[110]{Nichols2020} - or Quebecois distinctness does nothing to address real historical suffering or contemporary oppression.

\section*{Conclusion}

In the end, the politics of recognition offers a way of \textit{avoiding} real social and political change by rhetorical performance \textit{in the name of} justice without concrete moves \textit{towards} justice. Libraries are part of this dynamic, reluctant to challenge dominant relationships of race, class, gender, and property-ownership because they are institutions whose role is to reproduce those relationships generation after generation. 

Taylor and Tully's politics of recognition gives libraries - especially in the context of Intellectual Freedom - a language and a set of positions from which to \textit{recognize} cultural or lifestyle diversity without addressing oppressive cultural, social, political, and economic relationships. In this sense, libraries conform to the second of two alternatives Hall identifies with respect to popular culture:

\begin{quote}
Sometimes [a person] can be constituted as a force against the power-bloc: that is the historical opening in which it is possible to construct a culture which is genuinely popular. But, in our society, if we are not constituted like that, we will be constituted into its opposite: an effective populist force, saying 'Yes' to power. Popular culture is one of the sites where this struggle for and against a culture of the powerful is engaged: it is also the stake to be one or lost \textit{in} that struggle. It is the arena of consent and resistance. \autocite[192]{Hall2002}
\end{quote}

Libraries like to see themselves in terms of the first alternative \autocite{Popowich2019}, but their commitment to "neutrality" and to the liberal conception of individual Intellectual Freedom places them squarely in the second camp. A clear example of this can be seen in the way Toronto Public Library continues to celebrate its recognition of LGBTQIA+ rights even while banned from Toronto Pride for platforming a transphobic speaker.

Essentially, what Taylor and Tully try to do, what the Canadian state and libraries try to do, is to continue to weld the irreducible plurality of individuals - the basis of liberal social ontology - into a single, unified "people" while safely recognizing a small amount of cultural difference. The universal equality of rights and freedoms risks erasing the concrete specificities of peoples' lives, not just in a cultural sense, but in concrete terms of opportunity, prosperity, marginalization, and oppression. Such erasure of specificity was the very thing the "resurgence of the people" in the late 1960s fought against.

That resurgence, perhaps paradoxically, contained within it both hyperindividualistic and radically communal tendencies (sometimes, as with the Manson family, embodied in the same social phenomena). And we can understand the two tendencies of post-Rawls liberalism as deriving from this contradiction of late-1960s social thought. Post-Rawls liberalism insisted on two kinds of pluralism, one individualistic, the other communitarian, in the face of the homogenization and universality of post-war capitalism that Herbert Marcuse described as "one-dimensionality". 

The politics of recognition sought to reconcile these two tendencies, but ended up merely pretending that it was \textit{not} engaged in the erasure of difference through universality and the undermining of communal relationships through a foundational individualism. The politics of recognition maintained the goal of a unified people. In order for the liberal ideal of the procedural republic to be achieved, society must be decomposed into isolated individuals, and those individuals must become algorithmically interchangeable, must have their individuality stripped from them. 

Under capitalism, the liberal insistence on individualism covers an erasure of individual difference. Hobbes' notion of a unified people, Rousseau's conception of a "general will" all presume this kind of homogenization, but this only becomes achievable with the development of algorithmic and computational technology under neoliberalism \autocite{Popowich2020}\footnote{A Hobbesian set of "procedural rules of justice and morality" is only one potential outcome of individualistic "artificial social contract" theory which, as Janet Coleman pointed out, opened up a large range of political possibilities \autocite[274]{Coleman2000}. However, the development of industrial and then computational capitalism foreclosed most these possibilities, leaving algorithmic capitalism as the dominant political structure of the 21st century.}.

The politics of recognition, then, ends up constructing what Hall has referred to as an "authoritarian populist" regime in contrast to a genuinely popular-democratic one formed out of the constituent power of the multitude. In chapters 6 and 7 we will pick up Virno's distinction between the people and the multitude and apply it to Hall's idea of the "popular democratic", but before that we need to look in more detail at the ways the politics of recognition operates in specific controversies within Canadian librarianship. 

The library presents itself as a neutral mediator of intellectual and cultural positions, like Tully's nameless Chief, providing a value-free platform for "all sides" of an issue to be debated. All perspectives are innocently \textit{recognized} by the library because it knows that - like the right to habeas corpus - there are very real social and political goods and commitments that remain non-negotiable and unchallenged: the goods and commitments of racial capitalism; good backed up in the final analysis by the violence of the state. 

The politics of recognition helps to absolve libraries of any responsibility for the consequences of their intellectual and discursive "neutrality", such as the consequences of "neutrally" debating trans people's right to exist. The distinction between speech and action, as we will see in the next chapter, means that the construction of a transgender moral panic - the representation by libraries of trans people as fundamentally destructive of the proper social order - remains in the eyes of many library workers separate from the real violence carried out against trans people.

\section*{NOTES from PHD meeting}

- Too much focus on the one book of Tully's \textit{Strange Multiplicity}, when his thinking has developed since then. Read the \textit{Public Philosophy} volumes, and the article on Berlin's two concepts of liberty in context. Also, incorporate some of the secondary literature on Taylor and Tully (in the form of a lit review, maybe?). Tully maybe deserves to be fleshed out in his own chapter. Richer engagement with his work.
- Tone down the Taylor prose a little (massage it). Jeannie thinks I'm right in what I'm saying but that actual students or adherents of Taylor would disagree. Tone it down just to avoid that argument. 
- Jettison the Virno chapter (in the book, Virno can be represented by that fragmentary chapter on free speech, the linked data, and compound brain articles). Definitely include Hall chapter.
- Write out the chapter outline - make it clear/crisp/explicit.
- Anarchism remarks; probably not defensible. Cut?
- Highlight the Butler bit. Be explicit about why exactly I'm using her (the pre-social).
- Flesh out the paragraph from Harootunian. Inventory - traces - make the connections between Gramsci, Hall, Said, and what I'm doing explicit. 
- Jeannie sees this chapter as including four things: Canadian politics (cover this in chapter 2); the traces of liberal individualism; Taylor and Tully reflecting the traces of individualism (these two sections are the real focus of this chapter); libraries reflecting it back (cover in chapter 5). 
- Is there any IF theoretical "smoking gun" that covers this stuff explicitly (e.g. Taylor Tully?). Schrader? Jim Turk?
- Instead of "pure political theory", describe what I mean (Jeannie suggests: "normative or analytical", as opposed to theory that "actively engages the world"). 
- Charles Mills "Rawls \& Race"; Tully and Mills dialogue to be published.
- Hall: neoliberalism running amuck; Taylor and Tully's idealism ignores what's going on while Hall can see it clearly. Spell everything out a bit more. 
- The focus on individualism is to distract from liberal structures of power. 

Notes from David:
We met to discuss a draft of chapter 3. The chapter represents some further solid progress and the thesis continues to progress well. We discussed:

* do we need separate chapters on Taylor and Tully? (the consensus seemed to be no)
* do we need some more discussion of secondary literature (e.g. more inclusion of a summary of the existing literature on Taylor/Tully)?
* clarifying the overall argument of the chapter
* the good way in which Hall's approach underpins the analysis of the chapter
* the need to expand the selection of Tully's work that is engaged (especially not to limit to Strange Multiplicity only)
* to include the chapter outline
* in bringing in thinkers - e.g.Butler -  make clear to the reader the role that their work is making in your analysis 

Overall the writing continues to proceed well. Agreed to keep these comments in mind but next step should be the next chapter.

Agreed to meet in January to discuss the first draft of the next chapter.