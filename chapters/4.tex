% politics_and_if


% (5) don't say 'simply' when it isn't simple :-)

% (6) is there an asymetry in chapter 3 in that it discusses the US much more than Canada? [I think we agreed this wasn't the case]


\section*{Introduction}

In the last chapter we explored the contrapuntal relationships between the political crisis of the late 1960s - including the rise of Indigenous sovereignty and Quebecois nationalism - and both radical and liberal political theory. In this chapter, I want to situate the development of Intellectual Freedom within the same contrapuntal process. What I will argue is that Intellectual Freedom was forced, like political theory itself, to respond to the political crisis and that moreover it responded in \textit{the same way}, by giving rise to two distinct approaches: "Intellectual Freedom" and "Social Responsibility". Just like the liberal/libertarian vs. communitarian distinction, however, Intellectual Freedom vs. Social Responsibility is a debate \textit{within} the hegemony of liberalism. The rise of Social Responsibility was the librarianship equivalent to the rise of the communitarian challenge to liberalism as such.

But just as communitarianism was not a rejection of liberalism, simply a change in emphasis or "correction" of liberalism's fundamental individualism, so Social Responsibility is not diametrically opposed to Intellectual Freedom.
What often appears to librarians to be regularly recurring conflict between two distinct positions is in fact simply the local dissonance between two interlocking voices within a single contrapuntal ensemble. The point here is that if we want to \textit{actually} to develop a concept of Intellectual Freedom that is not caught within the double bind of liberal social and political thought - the irreconcilability of individual and collective rights, identity and difference - then we will need to step outside the counterpoint of Intellectual Freedom and Social Responsibility, to open up a space for a radically new understanding of individualism, intellectual activity, and political power. 

To recap some of the developments looked at in the last chapter, I argued that the crisis of the late 1960s provoked the creation of two streams - or in the language of contrapuntal music, "voices" -  within liberalism itself. One voice attempted to shore up the individualism of classical liberalism and utilitarianism against the communal and collective demands of 1968; the other proposed a communitarian variant of liberalism which took these demands seriously and attempted to incorporate them within liberalism's hegemonic project. Rawls' \textit{Theory of Justice}, as we have seen, attempted to combine these two voices in a single political philosophy and to provide a reinvigorated liberalism for the post-1968 conjuncture\parencite[69]{Forrester2019}.

The rise of these two tendencies also affected Intellectual Freedom. Prior to the 1960s there was a single concept of Intellectual Freedom based on the liberal individualist model tempered by the demands of postwar solidarity; after 1968 there were two competing concepts. The rise of Social Responsibility corresponds to the development of a communitarian variant of liberalism. The dichotomy between Intellectual Freedom and Social Responsibility is reflected organizationally within the American Library Association, which has an Office of Intellectual Freedom (founded 1967) and a separate Social Responsibilities Round Table (SRRT, founded 1968).

In addition to this distinction, I contend that Canadian librarianship has its own particular form of Intellectual Freedom informed by the kind of communitarianism formalized by Charles Taylor, James Tully, and others. Canadian librarianship does not refer much to Social Responsibility because the Canadian (communitarian) form of Intellectual Freedom already includes many of the social responsibilities that American librarianship needs SR to support. Thus, while in the American context, IF and SR are two distinct tendencies, in Canada they are subsumed under an already-communitarian form of Intellectual Freedom. This distinction is ignored within Intellectual Freedom debates in Canada and the US, and it is one goal of this thesis to analyze it and make it explicit. 

The distinction is due to the different constitutional regimes in Canada and the US, the Canadian informed by a politics of recognition derived from the United Nations Declaration of Human Rights; the American by the  individualism enshrined in the US Constitution and the Bill of Rights. We might think of the Bill of Rights and the Canadian Charter of Rights as modeling libertarianism/liberalism on the one hand and communitarianism on the other, and this modeling has significant effects on the practice of librarianship in the two countries.

Because Social Responsibility is folded in to the Canadian communitarian version of Intellectual Freedom, it is difficult to untangle and analyze the two tendencies. As a result, in this chapter, I will explore the philosophical lineage of Intellectual Freedom and its origin in classical liberalism and the American Bill of Rights regime. Then I will explore the rise of Social Responsibility in the context of the crises of the late-1960s, setting the stage for an understanding of the particularly Canadian form of Intellectual Freedom.


\section*{The Philosophical Lineage of Intellectual Freedom}

While Social Responsibility appears to be a diametric alternative to Intellectual Freedom, prioritizing social justice over individual rights and liberties, it still relies on the individualist social ontology of classical liberalism and social contract theory. For Social Responsibility, individuals come first and must enter into social relations through voluntary association after the fact. In this, Social Responsibility mirrors the communitarian challenge to liberalism, as a kind of correction of its individualist excesses.

Indeed, Michael Walzer sees communitarianism in the same way, as a "correction" of liberalism. He nonetheless remarks that "American communitarianisms have to recognize that there is no one out there but separated, rights-bearing, voluntarily associating, freely speaking, liberal selves" \parencite[15]{Walzer1990}. This perspective dominates both Intellectual Freedom and Social Responsibility, indicating that far from being diametrically opposed, they are simply two sides of the same coin. It is also a perspective that needs to be rejected if we are to escape from librarianship's double bind, which runs more or less as follows. 

Intellectual Freedom is dominated by individualism, rational choice, and the individual's responsibility to participate in liberal democracy. Intellectual Freedom's "neutrality" is integral to this process, and while Social Responsibility sees the library as needing to take sides, it typically adheres to the same liberal notions of individuality, freedom, and democracy. Social Responsibility is not an alternative to IF - as the ALA organizational structure would suggest - but merely a difference of emphasis within a larger hegemonic project. In his discussion of the value of communitarianism, Walzer admits that "it would be a good thing... if we could teach those selves to know themselves as social beings, the historical products of, and in part the embodiments of, liberal values" \parencite[15]{Walzer1990}, and this sentiment could stand as a motto of Social Responsibility. 

In order to move beyond Social Responsibility as a "correction" of Intellectual Freedom, we have to reject Walzer's view entirely. One form this rejection could take, and I will argue has been taken in Canadian librarianship, is to construct a version of Intellectual Freedom which already includes the Social Responsibility position. But this simply reiterates the hegemony of liberal social thought and catches librarianship once more in the double bind of individual and collective rights. To go further, we need to reject individualist social ontology completely, looking to social theories that take social construction seriously. In chapter 7 I will look at Paolo Virno's transindividualist conception of the multitude and what that theory offers librarianship. To begin with, however, we need to clarify the lineage of Intellectual Freedom itself. 

This lineage has three distinct strands. First there is the professional history of librarianship, especially in debates over the relative importance of Intellectual Freedom and Social Responsibility \parencite{Samek2001} but also in the concrete policies and interventions that are part of the rhetorical self-definition of the profession\footnote{A good example of these are the many editions of the ALA's \textit{Intellectual Freedom Manual}.}. This self-definition may be described as "the democratic discourse of librarianship" \parencite{Popowich2019} and one of the major contradictions within the profession is between this discourse and the real sociopolitical effects of library work.

Second, there is the legal framework, which begins chronologically with the Library Bill of Rights of 1938/1939 (explicitly referring to the American Bill of Rights), later gaining juridical support from the 1948 UN Declaration of Human Rights. The Intellectual Freedom statements of the International Federation of Library Associations (IFLA), the Canadian Federation of Library Associations (CFLA), and the British Chartered Institute of Library and Information Professionals (CILIP), which are not part of the American Bill of Rights regime, all refer to Article 19 of the UN Declaration. In this way non-US Intellectual Freedom statements connect IF - usually uncritically - with the entire discourse of post-war human rights. 

However, the reliance on the UN Declaration tends to obscure the differences between the post-war and neoliberal period. Human Rights scholar Samuel Moyn argues that the UN Declaration needs to be reread in the light of the passing of the welfare state. "The Universal Declaration", Moyn writes, "was connected with the believable empowerment and intervention of the state, not the prestige of non-governmental action or the cautious reform of judges with which social rights became bound up in a neoliberal age\footnote{for example, the Carver and Delgamuukw Indigenous land claim decisions - see Chapter 2}... Most historians of the document, celebrating it for an internationalization of rights politics that occurred decades later, have omitted the welfare state that it canonized" \parencite[44]{Moyn2018}. In many ways, Intellectual Freedom in the Commonwealth remains bound to the welfare state period just as American Intellectual Freedom remains bound to the political thought of Jefferson and Madison.

Third, there is the philosophical discourse of free speech and freedom of expression, beginning (for librarianship) with Mill's \textit{On Liberty} (1859)\footnote{Despite the prevalence of the "marketplace of ideas" in Intellectual Freedom debates, Milton's \textit{Areopagitica} (1644) has played little role in the development of Intellectual Freedom. Nevertheless, Milton's view that "though all the windes of doctrin were let loose to play upon the earth, so Truth be in the field, we do injuriously by licensing and prohibiting to misdoubt her strength... who ever knew Truth put to the wors, in a free and open encounter?" \parencite[45]{Milton1999} is identical with Mill's view. For an analysis of the marketplace of ideas in librarianship and how it relates to Milton and Mill, see \cite{Tanner2018}.} and subsequently supplemented by a concept of communicative reason (from Habermas), a focus on minority rights and justice (from Rawls) and a weak conception of hegemony (from Gramsci) \parencite{Alfino2014, Buschman2014, Raber2014}.

Habermas' thought has played an important role in recent thinking around the politics of libraries (for example in \cite{Buschman2003} and \cite{Buschman2014}), but Habermas' critical view of the rise of bourgeois society is typically replaced by a fully liberalized theory of communicative action, in which libraries serve as a foundation for the intersubjective speech acts of individuals. In this view, the public sphere is a sphere of consensus and progress and libraries play a role in the Enlightenment of public sphere participants. John Buschman, the foremost proponent of Habermas in librarianship, writes that:

\begin{quote}
In Habermas' high-theory terms, he has articulated the concept of libraries as democratic public spheres, holding out the possibility of communicative reason, truth verification, rational argumentation, and the providing of alternatives and alternative public spaces - all essential to a democratic culture. [...] The democratic possibility of rational communication also gives a way out of the radical pessimism that the critique of positivism and instrumental rationality has sometimes engendered in librarianship.  \parencite[179]{Buschman2003}
\end{quote}

Buschman ignores, however, the critical edge of Habermas' view that "the political task of the bourgeois public sphere was the regulation of civil society" \parencite[52]{Habermas1989}. In other words, public sphere institutions such as the public library were institutions of social control and the hegemonic project of liberalism itself\footnote{The Enlightenment view continues to dominate librarianship, contributing to the "vocational awe" \parencite{Ettarh2018} that is the focus of much contemporary critique}. Buschman, like so many liberal thinkers, ignores class divisions and therefore sees libraries as serving only a single, homogeneous group of users, rather than mediating the relations between middle- and working-class constituencies.

The social control thesis sees libraries as the means to inculcate immigrants the working-class with bourgeois values, habits, and respect for institutions. Practices such as silence, hygiene, payment of fines, respect for property, and the reading of uniquely "useful reading"\footnote{Whether to include fiction in public library collections was a controversial topic in the early years of the American Library Association.} all contributed to the spread of bourgeois norms and mores in the wake of 1848. 

Gramsci's conception of hegemony or leadership is important to understanding this aspect of librarianship. Drawing on Machiavelli, Gramsci used the image of the centaur to illustrate his concept of hegemony. The centaur - half-human, half-beast - exemplifies a dual model of power: the combination of coercion with consent. The state, in Gramsci's view, does not rely solely on physical coercion to dominate the bodies of human beings, but also on consent derived from the manipulation and organization of values, knowledge, ideas, and culture \parencite[Q1, \S48]{Gramsci1992}. The class whose ideas and knowledge are dominant in a society exercises hegemony through institutions such as schools, libraries, etc, to maintain social order and control, holding physical coercion in reserve. The identification of librarians with this dominant class both explains and perpetuates the unwillingness of LIS theorists to engage with questions of class at all\footnote{A notable exception is Stephen Bales' Gramscian exploration of \textit{The Dialectic of Academic Librarianship} \parencite{Bales2015}.}. 

However, the image of the centaur reminds us that these two forms of power are not discontinuous. In addition to the manufacture of moral or ideological consent, the library always exercises \textit{physical} discipline as well, in the insistence on silence and the prohibition of eating, for example, and even in the design of library buildings themselves (see \cite{Black2009}). The use of police force in the two cases we will look at later show that consent (libraries) and coercion (police) are two moments in a single continuum of state power. Indeed, Gramsci writes that

\begin{quote}
a class is dominant in two ways, namely it is 'leading' and 'dominant'. It leads the allied classes, it dominates the opposing classes. Therefore, a class can (and must) lead even before assuming power; when it is in power it becomes dominant, but it also continues to 'lead'... political leadership becomes an aspect of domination.  \parencite[Q1, \S44]{Gramsci1992}.
\end{quote}

Just as John Buschman blunts Habermas' critical point, so librarianship weakens and recuperates Gramsci's class analysis to hegemonic liberalism, for example in Douglas Raber's reconciliation of Gramsci and Mill \parencite{Raber2014}. One way in which Raber weaksn Gramsci's concept of hegemony is by using it solely to analyze society outside libraries, while leaving libraries themselves untouched, protected by the Enlightenment thesis. Society at large may be the site of hegemonic struggle, but libraries are neutral venues of unconstrained Intellectual Freedom to foster democratic participation. In this way, the Enlightenment view refuses to recognize librarianship's role in the manufacture and maintenance of consent, its place within capitalist, patriarchal, and racist hegemonic structures. For Gramsci, consent is not something spontaneous, it has to be produced within governed subjects:

\begin{quote}
Government by consent of the governed, but an organized consent, not the vague and generic kind which is declared at the time of elections: the state has and demands consent, but it also 'educates' this consent...  \parencite[Q1, \S47]{Gramsci1992}
\end{quote}

This unacknowledged role of libraries not only as crude institutions of social control but subtle organizers of consent is the real context of both Intellectual Freedom and Social Responsibility. It adds a deeper resonance to the Office of Intellectual Freedom's insistence that "the public library is the only government agency with a core mission that encompasses both the private and the public purposes of the Bill of Rights" \parencite[6]{OfficeofIntellectualFreedom2006}. The relationship between public and private, like that between individual and society, is a vital aspect of the maintenance of class leadership.

\section*{The Library Bill of Rights}

Central to librarianship's hegemonic project in the early period was the American Bill of Rights. The United States has always dominated Anglo-American librarianship, the profession as it is practised in the UK, Canada, the US, and other Commonwealth countries. The first national tax-funded libraries were instituted in Britain in 1850, but the creation of the American Library Association in 1876 gave American librarians an organizational unity that Canadian and British librarians lacked. The ALA is the profession's longest-standing association\footnote{The Canadian Library Association (1946-2016) and subsequently the Canadian Federation of Library Associations (CFLA) are the ALA equivalent in Canada. CFLA tends to take its lead from ALA policy recommendations but has to navigate a different legal and cultural context. In the UK, the Chartered Institute of Library and Information Professionals (CILIP) plays a similar role, but is in some respects quite different from both the ALA and the CFLA.} and it makes policy recommendations for the profession in North America at large. It is also the accrediting body for library education in the US and Canada and thereby exercises considerable professional hegemony. However, as opposed to the professional associations in law, medicine, or engineering, the ALA does not certify (and cannot then "disbar") librarians. Thus, while the ALA exercises a certain cultural hegemony across the profession\footnote{Toni Samek's important history of Intellectual Freedom in libraries between 1967 and 1974 explicitly links the ALA's struggle for cultural hegemony in the profession with the question of libraries as hegemonic institutions within civil society \parencite{Samek2001}. \cite{Bales2015} and \cite{Popowich2019} also deal with the question of libraries as hegemonic institutions.}, it does not hold legal or sanctioning power over libraries or librarians.

One area in which the ALA's hegemony is immediately apparent is Intellectual Freedom. The ALA's Office of Intellectual Freedom (OIF) was created in 1967, that is, in the context of civil disobedience and anti-war protest, and indeed the OIF has always been an explicitly first-amendment body \parencite{LaRue2017}. Judith Krug, the first director of the OIF made the connection clear in 1972, writing that "in this entire country, the library is the only institution where the First Amendment is - or can be - fulfilled" \parencite[811]{Krug1972}. However, American dominance and reliance on the concept of "free speech" enshrined in the First Amendment often overlooks the differences between "free speech" and "free expression" in other constitutional regimes.

The First Amendment underwrites the definition of Intellectual Freedom for the ALA. Krug put it succinctly when she described the Library Bill of Rights as "the library profession's interpretation of the First Amendment to the United States Constitution and our embodiment of the 'intellectual freedom' concept" \parencite[810]{Krug1972} arguing that the Library Bill of Rights made American librarianship unique. 

In Canada the juridical support for free expression comes from the Canadian Charter of Rights and Freedoms of 1982. The main difference between the American and Canadian contexts is that while the American concept of free speech is unlimited except in some very specific and legally limited exceptional cases (obscenity, child pornography, fighting words, etc.), the Canadian Charter begins with a sense of balancing individual rights against social and collective good: 

\begin{quote}
The Canadian Charter of Rights and Freedoms guarantees the rights and freedoms set out in it subject only to such reasonable limits prescribed by law as can be demonstrably justified in a free and democratic society. (\S1)
\end{quote}

My claim here is that while the formulation of Intellectual Freedom based on the First Amendment required a "correction" in the form of Social Responsibility, the Canadian Charter of Rights and Freedoms already had such a communitarian response built into it. Canada is thus a "communitarian" regime of rights including free expression compared with the "liberal" American regime. The use of the term Intellectual Freedom for both the Canadian and American contexts can therefore be misleading, as we will see in our discussion of the Toronto Public Library room-rental case\footnote{It is instructive to note that the reverse process informed the naming of Canadian political parties. Frank Underhill remarks that Canada and the US developed very similar two-party political systems, and "the fact that the British names for the [Canadian] parties were preserved and that the parties operated within a British constitutional framework made little difference to their essentially North American quality" \parencite[22]{Underhill1961}.}.

The American definition of free speech, which pre-dated the late-1960s resurgence of the people and the communitarian response to classical liberalism, supports a definition of free speech that corresponds to Mill's view in \textit{On Liberty}. However, the primary authorities for American freedom of speech and the Bill of Rights from the ALA perspective are James Madison and Thomas Jefferson, as the Seventh Edition of the ALA's \textit{Intellectual Freedom Manual} makes clear:

\begin{quote}
Freedom is the value most cherished by Americans. Constitutional provisions to protect freedom from the threats of usurpation, incompetence, and tyranny of the majority have created a constitutional legacy. [...] Both Thomas Jefferson and James Madison felt strongly that an informed and educated citizenry is the best defense against a despotic or tyrannous government. They believed that freedom of inquiry and speech are essential to the search for truth. For truth to emerge, erroneous ideas must also be available for the people to examine and discuss. The marketplace of ideas must be free and unfettered and, most of all, not restricted by government action. The founders also believed that an education citizenry is essential to the preservation of freedom and democracy.  \parencite[7]{OfficeofIntellectualFreedom2006}
\end{quote}

The authors of the ALA Manual go on to explicitly describe Intellectual Freedom as arising out of liberal individualism: "the marketplace of ideas must be free and unfettered and, most of all, not restricted by government action. [...] The Bill of Rights serves both to protect individual liberties... and to support an informed, educated citizenry with access to the open dialogue necessary for democracy" \parencite[7]{OfficeofIntellectualFreedom2006}. However, the reality of librarianship was never as clear-cut as this description makes it sound. The profession has always been marked by gender inequality \parencite{Garrison1972} and racial oppression \parencite{Hathcock2015, Schlesselman-Tarango2017}. However, the built-in liberal individualism of the ALA has made it difficult to address structural questions of gender, race, sexuality, and disability. As a result, the connection between Intellectual Freedom and American democracy is more tenuous than it is presented.

In practical terms, then, despite finding legal justification and grounding in the Bill of Rights and various Charters, Intellectual Freedom developed organically through the various political practices of early librarians. While these practices were informed by hegemonic American liberalism, Intellectual Freedom was not formalized until the eve of the Second World War. We will look at this process next.

\section*{1850-1938: Enlightenment vs. Social Control}

Intellectual Freedom was not part of librarianship's worldview in the early days of the state-supported library, generally dated to around 1848-1850 with the establishment of the Boston Public Library and the passing of the Public Libraries Act in Britain. In the wake of Chartism\footnote{For the significance of the decline of Chartism in the history of early public libraries, see \cite[37]{Black1998}.}, the 1848 Revolution, and the rise of labour militancy in the US\footnote{In the 1840s, labour rights gained significant legal recognition in the US, for example in the landmark \textit{Commonwealth v. Hunt} case of 1842 which ruled that labour unions were legal organizations with the right to organize and strike.}, the Anglo-American librarians were concerned to consolidate and defend the power of the triumphant middle-class. In the first of his prison notebooks, Gramsci wrote that "after 1848, the struggle against the 'religious' conception of life ends in the victory of liberalism (understood as a conception of life as well as positive political action)" \parencite[Q1, \S38]{Gramsci1992}\footnote{In the Canadian context, the triumph of the bourgeoisie is marked by the granting of "responsible government" in 1849.}.

The early library, then, must be seen as an institution of class conflict. It is unsurprising, therefore, that the historiography of librarianship in this period reflects class tensions. On the one hand, there is an "Enlightenment" thesis \parencite{Bivens-Tatum2012}, which sees public libraries as institutions of progress and democracy; on the other hand there is a "social control" thesis \parencite[220-224]{Black1998} which understands early libraries as concerned to promote bourgeois values and disseminate liberal ideology in order to defuse working-class agitation. Libraries in this view, play a de-radicalizing role in liberal society.

The Enlightenment thesis is the dominant one in library history, expressed in many of the most authoritative texts. Sidney Ditzion, for example, in his \textit{Arsenals of a Democratic Culture}, writes that

\begin{quote}
The major ideological currents of this period were directed towards producing a unified nation based on the free informed choice of individuals rather than on measures of indoctrination in behalf of any particular group. As it happened there was a fairly close identity among the requirements of national prosperity, the needs of the new dominant industrial middle class, and the tenets of flourishing individualistic philosophies. Divisive tendencies, having their origins in prejudices of race, section, nationality, creed, and class, were present indeed. It was hoped that these could be eased, or perhaps erased, by establishing agencies of enlightenment for adult and youth alike.  \parencite[75-76]{Ditzion1947}
\end{quote}

We have seen in the previous chapter how Pierre Elliot Trudeau's liberalism sought to assimilate difference to a universal Canadian citizenship. Similarly, Ditzion makes it clear that "the library had an Americanizing function" among new immigrants which does not strike him as problematic. In addition, the library also served the public sphere purpose identified by Habermas: public libraries, as opposed to the private libraries of the feudal nobility, can be seen as part of "the process in which the state-governed public sphere was appropriated by the public of private people making use of their reason" \parencite[51]{Habermas1989}. 

In contrast to the "Enlightenment thesis", the social control thesis sees early libraries as instruments by which to inculcate immigrants and the working-class with bourgeois values and respect for institutions. The ambiguous relationship between libraries and social order is underscored by the fact that Andrew Carnegie, whose name has become a byword for libraries funded and built by his philanthropy, dedicated a new public library to his workers after he had broken the largest strike in American history \parencite[237]{Krause1992} \parencite[23-24]{Popowich2019}.

Michael Harris, in his "revisionist" library history of the 1970s, has argued that what Ditzion describes as "Americanization" was an explicit attempt to "socialize the unruly immigrants" \parencite[61]{Rubin2016} and other associated rabble. Harris writes that

\begin{quote}
Boston's Brahmin's were especially unhappy about the flood of ignorant and rough immigrants into this country. The Standing Committee of the Boston Public Library noted that the people of Boston spent large sums of money on education each year, and their reasons were quite explicit: "We educate to restrain from vise[sic], as much as to inculcate sentiments of virtue; we educate to enable men to resist the temptation to evil, as well as to encourage and strengthen the incentive to do good." \parencite[14]{Harris1973}
\end{quote}

The public library was "one means of restraining the 'dangerous classes'\footnote{"Dangerous classes" is the translation of \textit{lumpenproletarian} used by Samuel Moore and Edward Aveling in the first English edition of Marx's \textit{Capital} (1887).} and inhibiting the chances of unscrupulous politicians who would lead the ignorant astray' \parencite[15]{Harris1973}. Harris and Spiegler note that what Ditzion calls "individualistic philosophies" (i.e liberalism) were recognized as a threat to the republic which relied on a conservative sense of belonging and social order: 

\begin{quote}
it was not so easy to discipline the increasingly unruly masses of the country. This was especially disconcerting in the face of the evident disintegration of the family and the church as the primary institutions of socialization and civilization. How was the Republic to be preserved if the people were given free reign, and were allowed to run unchecked by custom or moral precept from the cradle to the grave? \parencite[255]{Harris1974}.
\end{quote}

Public libraries, along with the modern school, were attempts to respond to the disintegrating, corrosive effects of the new liberal social order after 1848\footnote{In the Canadian context, similar fears were raised in the wake of the 1837-1838 rebellions. The Durham Report (1839) recommended the abolition of the conservative "family compact", liberalism, and responsible government. In his criticism of the Durham Report, Sir Francis Bond Head wrote that "the 'Family Compact' which [Lord Durham] deems is so advisable that the Queen should destroy, is nothing more nor less than the 'social fabric' which characterizes every civilized community in the world... The 'family compact' of Upper Canads is composed of those members of its society who... have amassed wealth" cited in \parencite[3-4]{Underhill1961}.}. But the need for institutions of social order - or what Louis Althusser has called "ideological state apparatuses" \parencite{Althusser2014} - was not limited to the 19th century. They are echoed in the fears surrounding civil disobedience and the decline of political participation in the late-1960s. But at that time, Intellectual Freedom was not an explicit concern of libraries, neither as a core component of a drive to enlightenment or of necessary social control.

\section*{1938-1968: The Origins and Development of Intellectual Freedom}

The origins of Intellectual Freedom are often ascribed to the Enlightenment thesis - the neutral protection of the individual right to access information to foster democratic participation. But as we will see, the idea of "neutrality" hides particular social and political commitments, primarily to individualism and to negative liberty. Intellectual Freedom as such arose as a response to calls for censorship in the politically turbulent 1930s. The central event, but by no means the only one, is often taken to be the banning of John Steinbeck's \textit{The Grapes of Wrath} in various states in 1939. While Samek ascribes the calls to ban the book to its supposed immorality and Steinbeck's (left-wing) "social views" \parencite[7]{Samek2001} it is important to remember that the exploitation Steinbeck described was not in the distant past, but was still going on in California \parencite{Lingo2003}. The attempt to ban the book was not merely an intellectual objection to Steinbeck's views, but a concrete social and political intervention\footnote{Douglas Campbell has challenged the idea that Intellectual Freedom only became a concern for librarians in 1939, and that \textit{The Grapes of Wrath} was the impetus for the adoption of the Library Bill of Rights. Campbell argues that the period between 1876 (the founding of the ALA) and 1939 saw three distinct periods with respect to libraries' attitudes towards censorship. After 1923, libraries became attuned to the increasing power of the state to censor materials, exemplified in the Smoot-Hawley Tariff of 1929, which allowed customs officers to cut imported books due to their content\autocite{Campbell2014}.}.

The resistance to banning \textit{The Grapes of Wrath} is often presented in librarianship as a defense of Intellectual Freedom and the neutrality of librarianship. But on the eve of the Second World War, and in the context of the virulent American anti-communism in the 1930s it is perhaps more useful to recognize the defense of Intellectual Freedom as a commitment to American individualism and negative liberty, both cornerstones of 20th century liberalism. Far from being "neutral", Intellectual Freedom was actively committed to the idea of individual choice dead set against the idea of positive liberty which lay "at the heart of many of the nationalist, communist, authoritarian, and totalitarian creeds of our day" \parencite[144]{Berlin1969}.

These commitments were part of the fabric of American librarianship. As Samek points out, while the ALA tends to take the credit for institutionalizing Intellectual Freedom, it was in fact grassroots librarians and activists who led the charge \parencite[7]{Samek2001}. The Library Bill of RIghts, almost always seen as a product of the ALA, was first instituted in Des Moines, Iowa in 1938, the year before its adoption by the ALA \parencite[5-7]{OfficeofIntellectualFreedom1996} in response to a challenge to the inclusion of Hitler's \textit{Mein Kampf}. James LaRue, former director of the Office of Intellectual Freedom argues that the Des Moines Bill of Rights "establish[ed] for the first time the library's endorsement of intellectual freedom" which he characterizes as "the right to access even uncomfortable or offensive content" \parencite{LaRue2018}.

But this focus on the \textit{content} of objectionable literature obscures the \textit{form} of Intellectual Freedom which was more important. Intellectual Freedom can be neutral with respect to the content of \textit{The Grapes of Wrath} or \textit{Mein Kampf} because it is committed to (i.e. not neutral with respect to) negative liberty. What is presented as neutrality is in fact a commitment to a particularly liberal political theory. Negative liberty takes the form, in fact, of \textit{not} evaluating the content of books or the social class or opinions of library users. This marked a shift from the library as either instiller of Enlightenment values or as mechanism of social control. This shift will become important when we look at the Toronto Public Library room-rental case, but we can see its effects in the changes made by the ALA to the Library Bill of Rights.

Both the Des Moines and ALA versions of the Bill of Rights contain two clauses relating to book selection and a clause related to room-booking. The book selection clauses enjoin librarians to collect material regardless of the views of the writer and representing "all sides" of a given issue "as far as material permits". The room-booking clause refers to making space available for public use. 

However, the ALA version removes a clause pertaining to politically unorthodox material. The De Moines statement reads:

\begin{quote}
3. Official publications and/or propaganda of organized political, fraternal, class, or religious sects, societies, or similar groups, and of institutions controlled by such, are solicited as gifts and will be made available to library users without discrimination. This policy is made necessary because of the megre funds available for the purchase of books and reading matter. It is obviously impossible to purchase the publications of all such groups and it would be unjust discrimination to purchase those of some and not others.
\end{quote}

Even though the Des Moines statement attempts to represent the view of "all such groups", the fact that "organized political, fraternal, class or religious... groups" were likely to uphold a version of positive liberty made this position unacceptable to the ALA. The ALA reiterated this "democratic" neutrality by inserting a political statement into the room-booking clause. The Des Moines clause reads:

\begin{quote}
4. Library meeting rooms shall be available on equal terms to all organized nonprofit groups for open meetings to which no admission fee is charged and from which no one is excluded.
\end{quote}

The ALA version reads:

\begin{quote}
3. The library as an institution to educate for democratic living shoudl especially welcome the user of its meeting rooms for socially useful and cultural activities and the discussion of current public questions. Library meeting rooms should be available on equal terms to all groups in the community regardless of their beliefs or affiliations.
\end{quote}

We can see here the beginnings of the neutral proceduralism committed to negative liberty that will flourish with the Cold War and the military-industrial complex. Indeed, by 1948, a new and quite different version of the Bill of Rights was developed \parencite[9-10]{OfficeofIntellectualFreedom1996}. The "all sides" clause now includes a requirement for truth, aligning with the technical and scientific focus of the postwar period\footnote{Norbert Wiener, the "father of cybernetics" states that "one of the few things gained from the great conflict was the rapid development of invention, under the stimulus of necessity and the unlimited employment of money; and above all, the new blood called in to industrial research" \parencite[148]{Wiener1954}. For the centrality of cybernetics to broader technological changes in global political economy see \cite[39-48]{Dyer-Witheford2015}. Libraries were necessary to this development because of the increasing complexity of information management. They thus aligned themselves ideologically and politically with the new technocratic regime. See \cite[219-220, 233-234]{Popowich2019}.}: "books or other reading matter of sound factual authority should not be proscribed or removed from library shelves because of partisan or doctrinal approval". 

Most importantly, however, were two clauses added to the Bill of Rights which makes reference to what the ALA considered the ethical exceptionalism of the United States.

\begin{quote}
3. Censorship of books, urged or practiced by volunteer arbiters of morals or political opinion or by organizations that would establish a coercive concept of Americanism, must be challenged by libraries in maintenance of their responsibility to provide public information and enlightenment through the printed word.

4. Libraries should enlist the cooperation of allied groups in the fields of science, of education, and of book publishing in resisting all abridgement of the free access to ideas and full freedom of expression that are the tradition and heritage of Americans.
\end{quote}

In addition to the technological developments, political events determined the shape of post-war Intellectual Freedom from the end of the Second World War. Eric Hobsbawm writes that the period between 1947 and 1951 were "probably the most explosive period" of the Cold War, when "the American fear of social disintegration or revolution within the non-Soviet parts of Eurasia were not wholly fantastic" \parencite[229]{Hobsbawm1994}. Fears of communist expansion abroad were mirrored by fears of infiltration, social disintegration, and state tyranny at home, all of which was reflected in the new Library Bill of Rights.

From the perspective of political theory, we can read the 1948 Bill of Rights, as we have seen, as exemplifying the negative liberty that would be described by Isaiah Berlin in 1958. In this sense, a "coercive concept of Americanism" would be the kind of "positive liberty" Berlin was afraid of. In his critique of Berlin's prioritization of negative over positive liberty, Charles Taylor writes that

\begin{quote}
When people attack positive theories of freedom, they generally have some Left totalitarian theory in mind, according to which freedom resides exclusively in exercising collective control over one's destiny in a classless society, the kind of theory which underlines, for instance, official communism. This view, in its characteristically extreme form, refused to recognize the freedoms guaranteed in other societies as genuine. The destruction of 'bourgeois freedoms' is no real loss of freedom, and coercion can be justified in the name of freedom if it is needed to bring into existence the classless society in which alone men are properly free. Men can, in short, be forced to be free.  \parencite[211]{Taylor1985a}
\end{quote}

If pre-war librarianship was committed to bourgeois values either in the Enlightenment or the social control sense, post-war librarianship hid its commitments beneath the veil of "neutrality". In fact, post-war librarianship expressed a particular set of social and political perspectives just as much as the pre-war library: a valorization of negative liberty, individual freedom, the utility of scientifically tested knowledge, and anti-communism. As such, Intellectual Freedom took a form consistent with the growing scientific, technical, and military power of the US in the early years of the Cold War, itself hiding behind the cover of technological, scientific, and managerial objectivity. As US power grew, American librarianship entered into a paradoxical relationship with it: on the one hand becoming more and more entrenched within the scientific-military-industrial complex, while on the other hand defending individual access to information against government control and censorship. The Bill of Rights underlined librarianship's commitment to liberal principles against their erosion by the American state while fully participating in the transition to mass technological society described by Marcuse in 1964\footnote{"It is my purpose to demonstrate the \textit{internal} instrumentalist character of this scientific rationality by virtue of which it is \textit{a priori} technology, and the \textit{a priori} of a \textit{specific} technology - namely, technology as a form of social control and domination. [...] Modern scientific thought, inasmuch as it is pure, does not project particular practical goals nor particular forms of domination" \parencite[161]{Marcuse1991}.}. The ambiguous positioning of North American libraries - like a chord which can function differently in different harmonic progressions - positioned them well both organizationally and ideologically for the transition to neoliberalism after 1968. They were able to claim to serve both the public and the private, the individual and the collective, the welfar state and the neoliberal state, through neutral proceduralism and the priority of form over content.


\section*{1968: Neoliberalism and Social Responsibility}

Intellectual Freedom changed, along with much else, in response to the crisis of the late-1960s. As we have seen, that period was marked by social and political unrest in the context of demands for collective rights - including civil rights and decolonization - as well as individual expression. The contradiction between countercultural individualism and the expression of group rights opened the door (culturally) for the neoliberal project. This is not to argue that civil rights activists, anti-colonialists, or gender and sexual minorities brought neoliberalism on themselves, but that this cultural shift combined with the late-1960s crisis of profitability to give capital a justification for neoliberal restructuring\footnote{As Quinn Slobodian remarks, "decolonization... was central to the emergence of the neoliberal model of world governance" \parencite[5]{Slobodian2018}.}. In his discussion of the transition "from Fordism to flexible accumulation", David Harvey writes that "the period from 1965 to 1973 was one in which the inability of Fordism and Keynesianism to contain the inherent contradictions of capitalism became more and more apparent" \parencite[141-142]{Harvey1990}. These contradictions forced a split within Intellectual Freedom, resulting in the creation of what has become known as Social Responsibility, as we will see below.

When the "resurgence of the people" challenged the liberal orthodoxies on which post-war Intellectual Freedom was based - individualism, negative liberty, etc. - librarianship found it hard to respond. Rather than forcing the profession to really enunciate its political commitments, the crisis drove the profession to double-down on its "neutral" and absolute conception of Intellectual Freedom. Rather than moderating Intellectual Freedom, "correcting" it with a communitarian adjustment (which, as we will see, constitutes Intellectual Freedom in Canada), a separate tendency arose within the profession which more or less adopted the communitarian approach.

In December 1967, in the midst of various countercultural, decolonial, and social justice demands after the "Summer of Love", the ALA consecrated Intellectual Freedom with the creation of the Office of Intellectual Freedom (OIF) \parencite{LaRue2017}. Planning for the OIF had begun in 1965 and was explicitly positioned as a First Amendment activist group\footnote{1965 was a significant year for First Amendment challenges and decisions. In \textit{Freedman v. Maryland} the court struck down the "prior restraint" of motion picture production enshrined in Maryland's censorship legislation since 1916; the \textit{Tinker v. Des Moines} case was not heard by the Supreme Court until 1969, when it upheld students' First Amendment right to protest in a landmark decision. However, the court also denied a First Amendment challenge to the Draft Cart Mutilation Act of 1965, arguing that the burning of draft cards was not protected expression and that its restraint was necessary for the good of the nation.}. Over the previous decade, anti-war sentiment had been increasing along with American activity in Vietnam, and anti-war protest was a key site of First Amendment debates. Perhaps more significant for the formation of the OIF, however, was the Civil Rights movement and the contradictions it exposed between librarianship's avowed democratic values and the reality of racism and segregated libraries \parencite{Graham2001}.

As a case in point, in 1959 segregationists in Alabama challenged a children's book called \textit{The Rabbits' Wedding} for depicting marriage between a black rabbit and a white rabbit. The Alabama Public Library Services Division (PLSD), headed by Emily Wheelock Reed, came under attack by State legislators for promoting anti-segration material. Eventually, the stance against \textit{The Rabbits' Wedding} proved unpopular and the legislators backed down, but Reed's defense of Intellectual Freedom and her collection policy made her personally a target of the attack. In a bid to have Reed removed from her position, the Segregation Steering Committee sought to change the qualifications to head the PLSD, dropping the requirement for an ALA-accredited library degree and requiring that the position be held by an Alabama native, which Reed was not.

Public librarians in Alabama had remained mostly quiet throughout the controvery. Graham writes that "Reed does not recall receiving any meaningful support from the librarians of the state during her controversial stand against censorship" \parencite[11]{Graham2001}. However, when the state was seen to be undermining librarians' professional autonomy by changing the PLSD requirements, \textit{then} public librarians were to moved to action - organized mainly by the Alabama Library Association - and the Segregation Committee ended up approving only a watered down version of the amendment. Nevertheless the damage was done and Reed resigned two months later.

Throughout this process, and others throughout the South, "the ALA Intellectual Freedom Committee failed to support Reed during the censorship controversy", but such events "did help shape the opinions of committee members on the issue of libraries and segregation" \parencite[13]{Graham2001}. In this period Intellectual Freedom was far from the primary ethical focus that it is today, but the attitude of apolitical aloofness that remains prevalent in the debates over library neutrality today was already clearly present\footnote{The curious combination of "neutrality" with a \textit{commitment} to Intellectual Freedom is one of the many contradictions inherent in the dominant conception of IF.}.

The commitment to a "neutral" (procedural) conception of Intellectual Freedom was challenged by the rise of the civil rights movement, and the attitudes of librarians reflected the complex race relations of American society itself. Patterson Graham has argued that 

\begin{quote}
Most white librarians... were moderates and largely apolitical; they were neither fervent segregationists nor vocal supporters of civil rights. Their attitudes towards blacks were as varied and as difficult to explain as the complex relationships between white moderates and blacks in the general population. Their racial attitudes incorporated notions of paternalism but also of professional responsibility. The Civil Rights movement was a time of anxiety for them, but for some the fear of changing racial mores was acute.  \parencite[24]{Graham2001}
\end{quote}

It is important to recognize, then, that structural features of American society, rather than simply the attitudes of individual librarians, played a large role in librarianship's reproduction of inequality and oppression. For example, "Black librarians were excluded from the decision-making process, and they could not exert measurable influence on 'white' libraries" irrespective of the individual feelings of white librarians. As Graham rightly states, "professional ethics stood little chance against a tradition that overpowered white consciences, democratic values, and even Christian teachings. Most who believed segregation was wrong could not imagine actively opposing it" \parencite[24]{Graham2001}

It was this inability to imagine opposing an unjust social system that shored up the neutral conception of Intellectual Freedom. The crisis of the late 1960s changed this attitude, with the example of civil rights, women's rights, gay rights, and anti-war activism showing how real resistance was possible. Such movements contributed to a heightened politicality among some librarians which ran counter to the dominant apolitical neutrality the profession espoused. This polarization - between apolitical neutrality and politicized commitment - has never been resolved (see, for example, the essays in \textit{Questioning Library Neutrality} \parencite{Lewis2008} or the 2018 ALA debate on library neutrality \parencite{Carlton2018}). This ambiguity was reflected in the creation of the OIF itself, appearing to enshrine a new, highly political conception of Intellectual Freedom within the ALA, while at the same time reinforcing the idea of neutrality and apolitical proceduralism.

For example, in 1974 the ALA endorsed the Equal Rights Amendment, and over the next few years passed motions to not hold its annual conferences in states where the ERA was not in effect\footnote{Such boycotts have been echoed in recent years with refusal to travel to states which enact transphobic "Bathroom Bills", see \cite{Fay2016}, \cite{Alford2016}.}. But even this positive and progressive commitment was held to violate the tenets of neutrality and Intellectual Freedom:

\begin{quote}
Opponents of the amendment and pro-ERA advocates of ALA neutrality, however, were quick to argue that library users 'have a right to expect the library to furnish them with uncensored information on both sides of this and all other issues. Adoptions of advocacy positions and participating in boycotts cannot help but strike a blow at the public's confidence in the fair-mindedness and even-handedness of librarians'. \parencite[2828]{Krug2003}
\end{quote}

By the late 1960s, social and political developments had led to a polarization in librarianship between a neutrality perspective (which included Berlin's negative liberty, Rawls' thin theory of the good and principle of equality of opportunity) and an advocacy perspective (positive liberty, a thick theory of the good, and the difference principle). However, while the two poles of Intellectual Freedom and Social Responsibility are often conceived as antagonistic or mutually exclusive, they are - like liberalism and communitarianism - but two aspects of a single outlook.

Despite the presence of the SRRT, Intellectual Freedom and the OIF have remained dominant within librarianship, with Intellectual Freedom as a guiding professional value over others (as we will see in Chapter 5). As Bocchicchio-Chaudry asks in her analysis of Critical Race Theory and libraries, "why is the Office of Intellectual Freedom privileged over the Office for Diversity, Inclusion, and Outreach services? Why does the field of LIS feel compelled to put intellectual freedom and racial justice on opposit sides of the debate?" \parencite[139]{Bocchicchio-Chaudry2019}. Regarding the SRRT, she remarks that "SRRT has failed to convince the ALA as a whole that libraries should be agents for social change for over forty years" \parencite[140]{Bocchicchio-Chaudry2019}.

The claim I am making here is that the answer to this question lies in the primacy of negative liberty, a thin theory of the good, individualism, and formal equality in American political culture more broadly. The hegemony of Intellectual Freedom and the OIF mirrors the hegemony of liberalism. Intellectual Freedom in its American form, with its "neutral" proceduralism, sense of universal equality, and its ability to point to rights enshrined in the First Amendment, is the perspective of individual rights and freedom derived from Social Contract Theory, J.S. Mill, and Rawls' principle of equality from which any collective sense of social justice is excluded. 

Because political culture in Canada is different in some significant ways from that of the US, the Canadian version of Intellectual Freedom takes on some of the aspects of Social Responsibility. This helps explain why Social Responsibility as such is less frequently mentioned in Canadian LIS discourse, but it also allows a certain amount of slippage between the American and Canadian definitions for particular political purposes, as when Toronto City Librarian called upon Intellectual Freedom in the defence of "freedom of speech", which is not a legal concept in the Canadian Charter of Rights from which the Canadian definition of Intellectual Freedom derives.


\section*{The 1970s: Conflict and Neutrality}

The rise of Social Responsibility as a communitarian or social-justice response to Intellectual Freedom set off a series of recurring debates. The Berninghausen Debate in the early 1970s was the first open conflict between Intellectual Freedom and Social Responsibility, liberalism and communitarianism, or Rawls' two principles, within the profession.

The debate was over the question of neutrality in library collections and was summed up by David Berninghausen's pronouncement that, far from social responsibility being committed to social justice, it must be understood in formal and universal terms.

\begin{quote}
Is is the social responsibility of librarians to select library materials from \textit{all} producers, from the whole world of published media (not from any approved list), to build balanced collections representing \textit{all} points of view on controversial issues, regardless of their personal convictions or moral beliefs. \parencite[emphasis added]{Berninghausen1972}
\end{quote}

Liberalism's formal and procedural insistence on negative liberty and a thin theory of the good as well as its universal pretensions are on full display here\footnote{We can see a premonition of today's "all sides" discourse in Berninghausen's position.}. From this perspective, Berninghausen saw the Social Responsibility movement as expressly challenging (even contravening) the Library Bill of Rights. The editors of \textit{Library Journal} dedicated a special issue to rebutting Berninghausen's thesis.

Like other forms of procedural liberalism, however, Intellectual Freedom's "thin theory of the good" ignores or represses positive commitments to particular goods. For example, by focusing on government or professional censorship ("approved lists"), Berninghausen ignores the influence of the market on collections. A library's collection, like individuals, is conceived as self-determining, free in the negative sense that it is unconstrained by explicitly political forces. But this ignores a whole spectrum of influencing factors, not the least of which are the profit-driven concerns of capital. Sanford Berman, one of the main critics of neutrality in the 1970s, summed up this problem in his preface to Samek's \textit{Intellectual Freedom and American Librarianship}:

\begin{quote}
Becoming increasingly dominant within librarianship - albeit never recognized by the Intellectual Freedom junta - is what might be termed the Techno-Blockbuster philosophy, which views digital technology as the overriding fact of the future, making traditional formats like books, magazines, CDs, and videos ultimately superfluous, yet which emphasizes - for the time being - conglomerate-published, Madison Avenue-hyped bestsellers, which may be bought in massive quantities to satisfy artificially created demand. And they aren't just acquired. They're prominently displayed and publicized by libraries as though there were some special, intrinsic, compelling worth to them. They are consciously pushed in ways that most midlist or small and alternative press materials are not, reflecting a bias in favour [of] bigness and big money. \parencite[xv-xvi]{Samek2001}
\end{quote}

Berman here makes explicit the connection between Intellectual Freedom's "neutrality" and what is valued in a capitalist society: the circulation of value and the privatization of public money. The dominance of mass-market material in library collections is a not a neutral choice according to a thin theory of the good, but a commitment to the concrete goods of consumer society and the promotion and circulation of its commodities and ideologies.

These positive commitments are obscured by a belief in neutrality. Thus, the recurring debates over neutrality in librarianship tend to ask whether neutrality is a good thing or not, rather than whether "neutrality" is even a valid category. Debates in the 1990s over whether the ALA should take a position on the 1991 Gulf War was framed as whether the ALA should "remain neutral" \parencite{Rosenzweig2008}; the idea that \textit{not} taking a stand constitutes a non-neutral stance remains marginal within the profession, precisely due to the hegemony of the political concepts already identified (negative liberty, etc.)

Librarianship's mystification of choices, values, and commitments under the mask of neutrality has some important consequences for our understanding of Intellectual Freedom. Intellectual Freedom is supposed to be a right (either natural or conventional) that attaches to an individual; neutrality is therefore the recognition and respect of that right. But this right can only exist in and through this recognition, it requires the recognition and legal support of either the First Amendment or the Canadian Charter. In this sense, Intellectual Freedom, like individuality itself is both sui generis and self-determining \textit{and} requires the intersubjective or legal recognition of some kind of sovereign power.

This ties Intellectual Freedom to the political developments we looked at in the last chapter, where Indigenous sovereignty and Quebecois nationalism required the recognition of Canadian courts in order to be seen to exist. In this way, individual freedom, Indigenous sovereignty, and Quebecois nationalism are - paradoxically - both prior to intersubjective and juridical recognition (indeed, prior to the social contract and the settler state itself) and yet require that recognition in order to exist. As a result, individual and collective freedom and self-determination are maintained as legal fictions while being undermined by and recuperated to capitalism, colonialism, and the bourgeois state. The question then becomes what social formation can honestly and adequately deal with questions of freedom and self-determination (even if that means dispensing with them) and what form of "Intellectual Freedom" would be appropriate to that social form. This is the question we will investigate in chapter 6. 

\section*{Conclusion}

In Chapter 2, we saw how the split between liberalism and communitarianism arose out of the combined effects of the "resurgence of the people" and the beginnings of the neoliberal turn. In this chapter, we have looked at how this process played out within librarianship. A commitment to post-1848 liberal values and social improvement gave way to Intellectual Freedom  formalized in the Library Bill of Rights and embodied in the ALA Office of Intellectual Freedom. This form of Intellectual Freedom represented the liberal ideals of negative liberty, a thin theory of the good, and universal egalitarianism through individual rights. A second tendency, representing communitarianism, developed in the late 1960s and was given the name Social Responsibility.

I have spent a considerable amount of time on the connections between Intellectual Freedom and Social Responsibility because, while the two positions are distinct within the American context - due to the reliance on the US Constitution and the First Amendment - they are combined into a particularly Canadian form of Intellectual Freedom. The differences between American and Canadian forms of Intellectual Freedom, roughly equating to liberalism and communitarianism, respectively, have been largely overlooked within LIS theoretical debates. The differences, however, are significant in that they have determined how Canadian librarianship responded to recent controversies over freedom of expression and civil rights.

In the next chapter, I want to explore in more detail the particularly Canadian "politics of recognition" and how that theory informs the Canadian conception of Intellectual Freedom. The politics of recognition as formulated by Charles Taylor and James Tully attempts to resolve some of Canadian society's historical and political problems through a specific communitarian approach based on mutual recognition of individuals and cultures. I will continue to use the contrapuntal approach to follow the multiplicity of voices through their tensions, dissonance, and moments of harmony, in order to provide the appropriate background for the controversies to be analyzed in chapter 5. 


