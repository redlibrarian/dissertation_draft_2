%NOTE Should this be the first substantive chapter?

\begin{quote}
The first form which the 'experience of social crisis' assumes a public consciousness, then, is the \textit{moral panic}... a society famous for its tenacious grasp on certain well-earned rights of personal liberty and freedom, enshrined in the liberal state, screws itself up to the distasteful task of going through a period of 'iron times'. \autocite[36-37]{Hall1988}
\end{quote}

\section*{Introduction}



As Stuart Hall pointed out in his analysis of the rise of "authoritarian populism" in the context of the social and economic crisis of the 1970s, in conjunctures of crisis the centrist, liberal neutrality of the thin theory of the good and negative liberty is gradually (and incrementally) swayed towards the right. The right becomes not an instrument of conservatism but of radical social restructuring, requiring the legitimacy of popular consent for the restructuring project. This happened during the neoliberal turn after the resurgence of the people (between 1968 and 1973) and we can see it now in the ways intellectual freedom is used to selectively legitimize certain perspectives and values at the expense of others, under the auspices of recognition, tolerance, pluralism and benefiting from the trusted authority of the library itself. Liberalism, and here we can use intellectual freedom as a concrete example, is won over to "creeping authoritarianism" under the veneer of the populism. What is often framed as a debate between competing values\footnote{As Deb Thomas wrote in the aftermath of the TPL event, "within the library community, we discussed and debated how to reconcile the intersection between intellectual freedom and inclusion - two values fundamental to our work" \autocite[7]{Thomas2020}.} or between gradual reform and sudden realignment, is in fact a new articulation of social forces and conflicting agendas for change.

There are many examples of recent controversies arising from these conflicts and contradictions, made possible by the social forces thrown into crisis by (and still not recovered from) the global meltdown of 2007-9: the protest against CIA recruitment at the American Library Association conference in 2019 by members of the Library Freedom Project \autocite{Germanos2019}, for example, or the cancellation (also in 2019) of the Radical Imagination Film and Discussion Series by Halifax Public Library when the series organizer refused to let Halifax Police speak at the event \autocite{Khasnabish2019}. The two controversies I want to focus on in this chapter are the platforming of transphobic speaker Meghan Murphy at Toronto Public Library in October 2019, and the imposition of "airport-style" security at Winnipeg Public Library, justified as protecting staff- and public-safety. Taken together, these two cases provide a clear illustration of the tensions and contradictions around intellectual freedom and social responsibility in Canadian librarianship. 

In Chapter 4, we looked at the lineage of intellectual history and mapped it to conjunctural changes in the twentieth century. I argued there that while in the US context intellectual freedom is firmly based on the classical liberalism of the First Amendment, in Canada intellectual freedom is based on a communitarian form of liberalism, the politics of recognition. The adherence to classical liberalism in the United States required the development of Social Responsibility as a complement or an alternative, while in Canada Social Responsibility is - theoretically at least - included within the concept of intellectual freedom.

In Chapter 3, I discussed the development of the politics of recognition both as a communitarian response to radically individualist liberalism and as a practical political strategy on the part of the Canadian state. The politics of recognition became an ideological articulation of various problems of Canadian society and their proposed resolutions. If we accept Hall's definition of a conjuncture as "the coming together of often distinct though related contradictions, moving according to different tempos, but condensed in the same historical moment" \autocite[41]{Hall1988}, then we can trace one conjunctural period from social and economic crises of 1968-73 to the post-Cold-War "end of history" in the mid-90s. In this view, the resurgence of the people in the late 1960s passed through the adoption of recognition as a political strategy to the attempt to enshrine recognition in the constitution and the two failed rounds of constitutional amendments, and finally the formalization of recognition in the political theory of Charles Taylor and James Tully, bookended by the Kanehsatà:ke resistance and the second Quebec referendum.

The current conjuncture begins with the Global Financial Crisis of 2007-2009, with the collapse of prior social and economic certainties (e.g the perpetual inflation of house prices) and the consequent shift to the right which issued in the "strongman" populist politics of Donald Trump others. It is a mistake to imagine that a later conjuncture will play out in identical fashion to an earlier one, to think that the rise of populism represented nothing but "the permanent, unchanging shape of reactionary ideas" \autocite[40]{Hall1988}. And yet, a crisis is a crisis, whether it is the oil crisis of the early 1970s or the toxic debt crisis of 2008, and the recession that followed each crisis was bound to have certain social and political effects.

In both cases, conspiracy theories developed in the wake of the crisis. Shadowy groups were intent on destroying the liberal order, the republic, or Western civilisation. We can trace the vagaries of these right-wing conspiracies from the (anti-semitic) "cultural Marxist" trope, through the deep-state/QAnon conspiracy, to the anti-woke panic over Critical Race Theory under Trump. While these conspiracies have their sharp wedges, Hall has described how they appealed to the silent majority of right-thinking people in the 1970s, oriented - then as now - primarily around race:

\begin{quote}
The real problem was the great liberal conspiracy, inside government and the media, which held ordinary people up to ransom, making them fearful to speak the truth for fear of being called "racialist," and "literally made to say black is white." It was race - but now as the pivot of this "process of brain-washing by repetition of manifest absurdities"; race as a "secret weapon," "depriving them of their wits and convincing them that what they thought right was wrong"; in short, race as the conspiracy of silence and blackmail against the silent and long-suffering "majority". \autocite[65-66]{Hall2021i}
\end{quote}

While it is important to draw fine distinctions between two historically specific conjunctures, it is just as important to draw parallels. This quote from "Race and 'Moral Panics' in Postwar Britain'" perfectly describes the anti-woke conspiracy that encompasses the "War on Christmas", cultural Marxism, \#BlackLivesMatter and Critical Race Theory, and anti-trans discrimination.

Transphobia is perhaps the most telling of the current round of "end of Western Civilization" conspiracy thinking. The "trans agenda", like the discredited gay agenda, in the 1980s, is seeking to destroy all the truths of liberal society by actively recruiting and violently transforming the nation's children. Trans people have such control over things like language that "we can't say 'women' anymore". While race continues to be a powerful organizing tool for populist, law-and-order ideology, trans people have become the easiest target around which to structure and articulate the fears of a right-thinking silent majority. 

While anti-Black racism has always been present within Canadian society, race-based animosity is primarily organized first around Indigenous people, then immigrants, and lastly (in right-wing parts of anglophone Canada at least), Quebec. Each of these groups challenges the universal egalitarian order on which Canada's self-image is founded, and can be held up as threats to Canadians' identity, structured around rural hard-work, rugged individualism, and a belief in their own tolerance and enlightened political and social culture. 

Attempts to resolve the challenges posed by Indigenous resurgence, Quebecois nationalism, and immigration were made via the constitutional process (including the Charter of Rights and Freedoms) and the passing of the Multiculturalism Act of 1985. These allowed Canadians to feel good about their commitment to diversity and inclusion (through the politics of recognition) without addressing the structural nature of Canadian racism. This issues have never been fully resolved, and we continue to deal with their presence in Canadian society and politics today \autocite{Popowich2021d}.

In the wake of the Global Financial Crisis, as the social and economic contradictions bit deeper and deeper, anti-Indigenous sentiment developed alongside a new round of Indigenous self-confidence. The Truth and Reconciliation Report of 2005, Idle No More movement in 2012, and protests around Murdered and Missing Indigenous Women and Girls (like the REDdress project), residential schools (including the toppling of colonial statues in 2021), and against miscarriages of justice in the case of anti-Indigenous crime (for which the Colten Boushie murder in 2016 stands as a significant touchstone) provoked a backlash on the part of the settler-colonial majority of "right thinking people". When the RCMP was too slow in clearing Indigenous anti-pipeline protesters from public highways in 2020, private citizens reacted against Indigenous "special treatment"\footnote{Anti-Indigenous sentiment, like anti-Quebecois sentiment, often revolves around perceived unequal or unfair special treatment, perpetuated in myths about Indigenous people getting free goods and services, not paying taxes, or benefiting from affirmative-action (as a form of "reverse racism"). See \autocite{Vowel2016}.}, seeing it as a violation of Canadian individual egalitarianism.

The structural nature of Canadian racism and transphobia is not accidental or superficial, to be dealt with through "unconscious bias" seminars or Human Libraries. They are not a failure of knowledge or insight, to be procedurally corrected through the discovery of new facts. Rather, as Hall writes,

\begin{quote}
How things are represented and the "machineries" and regimes of representation in a culture do play a \textit{constitutive} and not merely a reflexive, after-the-event role. This gives questions of culture and ideology, and the scenarios of representation - subjectivity, identity, politics - a formative, not merely an expressive, place in the constitution of social and political life. \autocite[248]{Hall2021b}
\end{quote}

The argument I have been making is that ideological constructions like the politics of recognition and intellectual freedom are the theoretical supports and justifications for particular regimes of representation. Intellectual freedom in particular is the mechanism by which libraries negotiate and represent particular cultural and social values while at the same time disavowing them and denying any responsibility for them. Intellectual freedom based on the politics of recognition allows libraries to pay lip-service to communitarian social goods while placing individualist limits around them (as in Taylor), allows libraries to insist on their democratic role while supporting centralist, colonial conceptions of consitutional sovereignty (as in Tully). 

In this chapter, we will look at two case studies in order to analyze the representational role libraries play in a society in crisis, under the auspices of the supposed neutrality ("all sides") of intellectual freedom. Before we get to the case studies, however, I want to take a brief detour through the speech/action distinction.

\section*{Speech vs. Action}

Hall argues, following Gramsci's image of the centaur combining consent with coercion, that two processes are operational in a given conjuncture, especially one marked by crisis: strong-state authoritarianism and "the construction of popular authoritarian ideologies" \autocite[67]{Hall2021i}. It is important to bear in mind, however, that these two processes are not independent from each other, but merge and reinforce each other. It is tempting to see them as distinct and mutually exclusive, especially if we subscribe to the distinction between speech and action that is constitutive of the liberal view of free speech/intellectual freedom.

The limits places around cultural difference in Taylor and Tully, the line between expressions of difference that can be recognized and tolerated, and expressions which must be dismissed or repressed in order to ensure the stability and legitimacy of the liberal order, maps to this speech/action distinction. The distinction is enshrined in the First Amendment, and there is thus an integral connection between the limits of communitarian thinking and free expression/intellectual freedom. 

Frederick Schauer argues that the First Amendment protection of "freedom of speech" implicitly draws a distinction between free speech and other kinds of freedom. As a result,

\begin{quote}
it appears, initially and obviously, that the protections of the First Amendment extend to some acts or events or behaviours but not to others. Indeed, not only the First Amendment but also any coherent principle of freedom of speech presupposes a meaningful distinction between the activities that are encompassed by the principle and those that are not. Loosely and preliminarily, we can label the former as "speech" and the latter as "action". \autocite[1-2]{Schauer2014}\footnote{Shauer actually considers this preliminary labeling inadequate as he explores the question of whether the speech/action distinction is a mistake.}
\end{quote}

Speech, therefore, is cultural and can therefore be recognized and tolerated within the limits set by the politics of recognition, but only by excluding action - specifically, but not limited to, violent action\footnote{Of course, violence perpetrated by the state is exempt from this exclusion.} - as a violation of liberal norms, values, and procedures. Free speech as a human or civil right depends on this distinction, this exclusion of a violent opposite or Other. 

It is impossible to maintain this distinction in practice. Even if we take the speech/action distinction for granted (Schauer notes that "\textit{all} speech is action (or conduct) in some sense"), Stanley Fish points out that "speech always seems to be crossing the line into action, where it becomes, at least potentially consequential" \autocite[105]{Fish1994}. The sacrosanct nature of this distinction, however, explains the liberal insistence that \textit{more speech} is the only way to come to truth and social harmony, the only legitimate way to challenge incorrect, harmful or hateful views. Prioritizing speech, insisting on more and more speech, excludes action from the political realm.

From the liberal and communitarian perspectives, as we have seen, it is safe to recognize cultural difference because cultural difference is immaterial (speech rather than action). With speech, the liberal status quo is defended because the proper procedures (free speech, recognition) have been respected; no violent threat to the liberal order took place. Recognition of different conceptions of the good framed in terms of culture or lifestyle does nothing in and of itself to address real material inequality, oppression, and violence. In 2021, racist, Islamophobic, anti-Asian, and anti-Indigenous violences continues to mark Canadian society \textit{despite} the politics of recognition and liberal tolerance and pluralism. The discovery of over a thousand unmarked graves of Indigenous children on residential school sites across Canada does not find an adequate response in the recognition of Indigenous cultural difference or of a plurality of conceptions of the good. 

In many ways the politics of recognition allows for a kind of idealist hypocrisy to take the place of real change. A politician or organization recognizing that "trans rights are human rights" is a safe alternative to real action to protect trans lives. Statements of library values - community partnerships, for example - take the place of actually partnering with communities. The Canadian government "recognizes" the pain and suffering caused by residential schools - enshrined in a photo of Prime Minister Justin Trudeau kneeling sadly at one of the residential grave sites in 2021 - even as it continues to fight a Canadian Human Rights Tribunal order to compensate residential school survivors and their families. Recognition becomes a performance, a cultural representation, a piece of theatre, which actively blocks progress and social justice.

But more insidiously, and more importantly in the context of this thesis, recognition provides a cover for actual harm. Stuart Hall has written about the construction of "moral panics" in order to justify the rightward shift in the UK in the 1970s. The broad, amorphous spread of conspiracy theories is harnessed to specific "enemies within", distruptive and destablizing Others whom it is safe to hate, because that hate both distracts from the real causes of the crisis and allows the quiet imposition of specific political conditions (the "drift into the law and order state", as Hall puts it). Hall writes that the ideological shift to a law and order state is, in its early phases,

\begin{quote}
sustained by what we call a \textit{displacement effect}: the connection between the crisis and the way it is appropriated in the social experience of the majority - social anxiety passes through a series of false 'resolutions', primarily taking the shape of a succession of \textit{moral panics}. It is as if each surge of social anxiety finds a temporary respite in the projection of fears on to and into certain compellingly anxiety-laden themes: in the discovery of demons, the identification of folk devils, the mountain of moral campaigns, the expiation of prosecution and control - in \textit{the moral-panic cycle}. \autocite[36]{Hall1988}
\end{quote}

What I will explore in this chapter is the role played by Canadian libraries in this moral panic cycle and in the process of demonizing trans and Indigenous people. This process takes place under the auspices of intellectual freedom and the politics of recognition, but not in a straightforward way. In the first of two case studies we will look at in this chapter, intellectual freedom is \textit{explicitly} used to defend the legitimation of a transphobic moral panic by the authority of the library's liberal bona fides; in the second, intellectual freedom is notably absent from the discourse, despite intellectual freedom issues actually being central to the case. I will argue that the demonization of trans and Indigenous people takes priority over intellectual freedom, reducing intellectual freedom to a technique in the legitimation of a transphobic moral panic. The reason intellectual freedom can be deployed in the TPL case is that that case gravitated around "speech" while the WPL case gravitated around "action". We can best understand the relationship between the two cases if we reject the speech/action distinction, though we must understand that the distinction played a role in determining library policy and response to criticism.

\section*{Challenges to Individual Essentialism}

While Indigenous peoples have been a traditional and historical "folk devil" of settler-colonial society, trans people are uniquely positioned in the 21st century because they disrupt one of the last naturalistic or essentialist myths of Western culture, and thereby one of the last supports of the idea of a natural, pre-social individual. Once the last defence of the "natural man" is gone, then the social contract project and its idea of consent must also disappear. It is for this reason that transphobia has inspired such virulent responses in defence of the "objective truths" of Western civilization.

We have seen how Taylor and Tully both moderate post-Rawlsian liberalism but do not reject it. Both believe that collective relations must be taken seriously, fostered, and supported, but must remain subservient to individual rights and freedoms founded on a pre-social individual identity. Taylor's formulation of an intersubjective politics, for example, still understands social bonds as external relations of pre-existing individual subjects, rather than constituting or producing a fully socially-constructed, transindividual personality. 

The insistence on individual subjectivity is political important because it supports the idea of the consent of the governed. It is a truism of social contract theory that only individuals who ontologically pre-exist social relations can consent to be bound by them. Rousseau, for example, writes:

\begin{quote}
There is only one law which by its very nature requires unanimous consent. That is the social compact: since civil association is the most voluntary act in the world, every man having been born equal and master of himself, no-one can under any pretext whatsoever, make him a subject without his consent (\textit{aveu})\footnote{The word \textit{aveu} is usually translated as "consent" in this sentence, however a more accurate translation would be "avowal" and describes the recognition of another's sovereignty.}. \autocite[440]{Rousseau1964a}
\end{quote}

The full social production of identity puts paid to the liberal notion of consent just as it does the liberal notion of freedom: if consent is socially constructed, then it becomes meaningless\footnote{This critique of consent has been an important focus of feminist scholarship, see \autocite{White1990} and \autocite{McGuinness1993}.}. If individual freedom and agency is rejected, then Rousseau's claim that "from universal silence we must presume the consent of the people" \autocite[369]{Rousseau1964a} becomes unsupportable. 

Judith Butler clearly identifies the importance of pre-social naturalism or essentialism to liberal politics:

\begin{quote}
Perhaps the subject, as well as the invocation of a temporal "before", is constituted by the law as the fictive foundation of its own claim to legitimacy. The prevailing assumption of the ontological integrity of the subject before the law might be understood as the contemporary trace of the state of nature hypothesis, the foundationalist fable constitutive of the juridical structures of classical liberalism. The formative invocation of a nonhistorical "before" becomes the foundational premise that guarantees a presocial ontology of persons who freely consent to be governed and, thereby, constitute the legitimacy of the social contract. \autocite[5]{Butler1990}
\end{quote}

The threat of destabilizing "natural" categories, such as binary sex, becomes apparent. If liberal rights and freedoms - and consent to liberal government - are built on a fundamental natural individualism, then any adherence to a shared or collective identity, and any identity which rejects a pre-social, natural foundation for politics, is anathema, destructive of the (properly liberal) social order. Collective identity is marked by the shadow of collectivism, the communal/communist Other of liberal possessive individualism. Anti-foundational theories of social construction challenge the naturalism of pre-social state of nature. 

Indigenous belonging and transgender identity fit these requirements in Canada, which is why they are excluded from accommodation within the liberal regime (even as they are officially "recognized", for example in the Canadian criminal code\footnote{Note about the inclusion in the Canadian criminal code. Bill C-60?}. It remains safe to discriminate against these two groups in Canada. Indeed, even as they grant "recognition" to Indigenous and trans rights, libraries represent Indigenous and trans people as disruptive and dangerous, corrosive to the liberal order. This is only a problem, of course, to those who continue to subscribe to the hegemony of liberalism itself. 

Collective and anti-foundationalist identities become easy targets for the ideological construction of "enemies within", demonized Others, and moral panics for the management of political crises. In the more libertarian US regime, this antagonism between individual and collective, naturalism and social construction, adopts the black-and-white view of classical liberalism and utilitarianism, while the Canadian context - through the politics of recognition - adopts what appears to be a more nuanced and tolerant approach that nevertheless remains committed to the "common sense" implicit goods of liberal society, capitalism, and settler-colonial structures of domination.

\section*{Description of the Events}

The basic outlines of the two controversies are straightforward. In August 2019, a group called "Radical Feminists Unite" rented a room in Toronto Public Library's Palmerston branch for "A Discussion and Q \& A with Canadian Feminist and Journalist Meghan Murphy". The stated purpose of the October 29 event was "to have an educational and open discussion on the concept of gender identity and its legislation ramifications on women in Canada" (TPL email, August 20, 2019). Murphy, a "gender critical feminist"\footnote{"Gender critical feminism" is the name adopted by feminists critical of what they see as a "gender ideology" which recognizes and accepts trans women. The term "gender critical" is more often used in the UK, while "trans-exclusionary radical feminism" (TERF) is most often used in North America. "Gender critical feminists"/"TERF" often criticize policies designed to protect trans rights and increase trans inclusions, such as the reformed UK Gender Recognition Act (2018), and the inclusion of gender identity as a protected category in Canada's hate crime laws. For the UK context, see \autocite{Zanghellini2020}. For transphobia as a global phenomenon with links to the right, see \autocite{Butler2021}.} had spoken at Vancouver Public Library in January of that year, which sparked protests among community members and led to VPL being banned from Vancouver Pride in the summer. Murphy herself had been banned from Twitter for repeated misgendring and deadnaming trans people\footnote{Murphy sued Twitter, but the case was thrown out by the First District Appellate Court of California, who found that Twitter did not violate free-speech laws in banning Murphy (Murphy v. Twitter, Inc [2021]).}.

The announcement of the room-rental sparked protest in Toronto among trans people, allies, and LGBTQ+ activisits. One point of contention was over the library's insistence that the topic of discussion would not violate TPL's space-rental policy, a policy which had been strengthened in 2017 in the wake of protests against the library renting space to white supremacists. A TPL board meeting held on October 22 was seen as an opportunity to express opposition to the rental and many community members and library workers attended. Despite weighty testimony from trans people as to the violence they face in Toronto and the harmful effects of transphobic speech on their lives, the TPL board did not rescind the room-rental. I will argue below that the board meeting constituted a perfect example of the \textit{performance} of recognition without restribution which, as we have seen, marks Canadian politics.

The evening of October 29, approximately 1000 protesters gathered outside the Palmerston branch. They protested peacefully, sometimes chanting slogans, at one point holding a seated "read-in" in front of the library doors. TPL staff reported "lots of police presence" the branch, but evidence conflicts as to whether they were asked to be there by TPL or were deployed on their own initiative. At one point, some of the protesters entered the library, at which point the police locked the library doors. Protesters were told they could leave via a rear entrance, but this was not clearly communicated or understood, and protesters reported feeling trapped and kettled by police in the library. At 8:30 when the library officially closed, the crowd left peacefully. 

Before and after the event, debates about intellectual freedom and free speech played out in Canadian librarianship and the media. Toronto Public Library solicited\footnote{Source: email from Toronto City Librarian Vickery Bowles to James Turk dated October 13, 2019 (see Appendix A). Turk's blog post does not state that the post was written at Bowles' request.}. a blog post supporting their position from James Turk, Director of the Centre for Free Expression at Ryerson University and a frequent commentator on intellectual freedom in libraries \autocite{Turk2019}. Blog posts\footnote{Not sure if I need to disclose this, but I was one of the participants in this debate through my blog}, newspaper stories, and scholarly articles staked out both sides in the debate, but often appeared to be speaking past each other rather than seriously engaging. It appeared as though intellectual freedom in Canadian libraries constituted, the phrase Hugh Maclennan used to describe the split between English and French Canada in the 1940s, "two solitudes". 

The TPL event was only one controversy surrounding libraries and Meghan Murphy. Earlier in 2019, Vancouver Public Library (VPL) had rented a room to a gender-critical feminist group for Murphy to speak, and VPL was subsequently banned from the city's annual Pride parade \autocite{CBC2019}. Until that time, VPL had proded itself on valuing "community-led planning" and "community partnerships", including real efforts to make connections with the LGBTQ+ community. But the individualistic orientation of intellectual freedom and exchange/contract (i.e. the renting of space) took precedence over VPL and TPL's communitarian aspirations. The rental of library space to TERF groups hosting Meghan Murphy was considered by many trans people in the city and the organizers of Pride to have ruptured their community connections\footnote{Libraries are selective in terms of which communities they recognize and serve. For example, in 2020 Seattle Public Library (SPL) rented space to another TERF group for Murphy to speak. When in June 2020, SPL won the \textit{Library Journal} \textit{Library of the Year} aware, the Executive Director of the library said that the award was "a well-deserved recognition of their commitment to involving community voices in programs, reducing barriers to access and actively pursuing key partnerships" \autocite{SPL2020}. \textit{Library Journal} gives out annual "Mover \& Shaker" awards to individual librarians, and in the wake of the SPL award, many of these librarians returned their own awards. The TPL event is one moment in a longer and broader debate around trans rights, individualism/communitarianism and intellectual freedom.}. 

Because (individualistic) intellectual freedom is named in library values statements alongside communitarian ones, these debates are often framed within the profession as merely a conflict between values, a conflict to be solved by the liberal procedures of recognition, compromise, and more discussion. The issue is seen as one of a failure of knowledge, as if - in Lukacs' words - the contradiction that arises "only proves that our knowledge is as yet imperfect" and "must therefore be transformed and subsumed under even wider theories in which the contradictions finally disappear" \autocite[3]{Lukacs1971}. We will see in the two following chapters how these contradictions are not failures of knowledge or understanding, but reflections of real material problems in North American social relations. 

The point that the contradiction between intellectual freedom and social responsibility can be overcome simply by recognizing their larger affinities is explicitly made by Alvin Schrader, a longtime advocate for intellectual freedom:

\begin{quote}
What the library profession needs is better informed critics and advocates about the core values of intellectual freedom and social responsibility and their interconnectedness. We must start with the view that constitutionally protected expressive freedom is fundamental to social justice; they are not opposites, not binaries, and never achievable one at the expense of the other. \autocite{Schrader2019}
\end{quote}

Critics of the dominant conception of intellectual freedom are badly informed; there is no problem with our conceptions of intellectual freedom and social responsibility, they have just been misunderstood. This attitude forecloses the very possibility of a fundamental critique of intellectual freedom: the contradictions within intellectual freedom can (and can only) be overcome with better knowledge and understanding within the terms set by the constitution and charter rights. The sanctity of Canadian liberal constitutionalism places an unquestionable limit on the kind and depth of critique permissible with respect to intellectual freedom. 

Intellectual freedom, enshrined in the Charter of Rights and Freedoms, takes priority over community relationships and other communitarian policies and orientations. We saw in a previous chapter how communitarianism is limited by a set of fundamental liberal rights (such as habeas corpus) which are indispensable to a liberal order, no matter how communitarian its orientation. The way intellectual freedom trumps communitarianism or social responsibility in librarianship is evidence of that individualistic limit. 

\section*{Section?}

If we accept the primacy of intellectual freedom over other library values - even in a "first among equals" sense - then it appears surprising that it was never raised in the context of "airport-style security" at Winnipeg Public Library (WPL) in 2019. Winnipeg is a small city in the middle of the country whose history is full of contradictions. It was founded as a colony in 1811 by Thomas Douglas, 5th Earl Selkirk, on a grant of land from the Hudson's Bay Company (founded by Royal Charter in 1670). By 1870, it had become the birthplace of the Métis Nation through the resistance of the Métis to the imposition of federal Canadian power in the Red River Resistance (1869-1870), led by Louis Riel. Deeply colonial in its foundations, Winnipeg encompasses many settler-colonial contradictions. Louis Riel, for example, was a Member of Parliament executed for treason in 1885. 

In 2016, Winnipeg had the largest Indigenous population of any city in Canada, but also an endemic problem of anti-Indigenous racism. Named "Canada's most racist city" in 2015 \autocite{Macdonald2015}, Winnipeg was being hailed as a "capital of reconciliation" a year later \autocite{Macdonald2016}. Despite this purported turnaround, the settler-colonialism remains deeply rooted in Winnipeg society, not least in the extrajudicial execution of Indigenous people by police. In a recent study of colonial violence on the Canadian prairies, Dorries et. al (2019) argue that cities like Winnipeg, longstanding sites of Indigenous organization and resistance, are also

\begin{quote}
sites of specific forms of settler colonial violence and settler colonial governmentality, including displacement through gentrification and development, institutionalized physical violence and police brutality, and systemic discrimination in housing and employment. With policing resembling neighbourhood occupation, community survival is criminalized while racist discourse continues to rationalize incredible rates of police violence as normal and acceptable. \autocite[11]{Dorries2019}
\end{quote}

Downtown Winnipeg, the heart of the province's governmental and financial administration, brings suburban white middle-class Winnipeggers in contact with poor and often homeless marginalized communities, particularly Indigenous people, but also racialized immigrant populations. Relations between these demographics, mediated by "Business Improvement Zone" patrols and city police (i.e. capital and the state) have often been strained, and recent years have seen a broad securitization of the downtown area, especially in businesses such as the (state-run) liquor stores \autocite{Gowriluk2019}. The increase of public and private security downtown, as well as infrastructural decisions like secure underground parking, are designed to ensure that white, middle-class settler Winnipeggers feel as safe as possible among a population of dangerous, demonized Others, an image constructed by the media, the state and, as we will see, the library itself. 

The deep connection of libraries to this colonial dynamic was brought to light in early 2019 when WPL implemented metal-detectors, pat-downs, and bag-searchers in its main downtown branch ("Millennium"). This was ostensibly in response to safety concerns on the part of library staff\footnote{Main/downtown branches are always on the front lines of social problems in North American cities, and with ongoing cuts to social services, pick up more and more of the responsibility to provide social services without adequate funding or support. Many library policies - such as "no sleeping" policies - are misguided attempts to deal with these issues on a piecemeal basis.}. WPL took advice from city police on how best to deal with safety concern, rather than listening to local community groups or other libraries \autocite[55-56]{Selman2019}.

The recent expansion of security is ostensibly in response to a meth crisis in the city (in the next chapter we will look at Hall's theory that a panic around criminalization is motivated by an increase in police funding, arming, and mobilization). But the dynamic of the security response is not new: historically, collaboration between racial capitalism, the city police, and fearful white settlers has been a mechanism of oppression and marginalization \autocite{Toews2018}. Dr Bronwyn Dobchuk-Land, a researcher at University of Winnipeg specializing in settler-colonial violence and the carceral logic of liberal states, was one of the members of Millennium4All, a community group that arose in response to the WPL security measures. Dobchuk-Land connects security measures at the library with anti-poor and anti-Indigenous policies more broadly: "Being familiar with Millennium Library and the kind of space it is and who uses it, and being familiar with the way that politics work in Winnipeg with public discourse and security 0 you just know immediately who it's intending to keep out" (quoted in \autocite[4]{Wilt2019}). The added security at WPL led to a large drop in usage and the exclusion and increased marginalization of already vulnerable populations for whom, in the face of drastic cuts to social services, the library was often the only public place they could go during the day \autocite{Selman2019}. 

A year of sustained community-led pressure followed the new security implementation, including testimony at City Hall, read-ins, drag queen storytimes outside the library, and the presentation of a report on the effects of increased policing on the library as well as alternative policy recommendations \autocite{Selman2019}. This report "outline[d] ways to pursue real safety at the library through a combination of de-escalation, harm-reduction, mental health care, and access to basic needs - not enhanced securitization measures that target the most vulnerable" \autocite{Toews2021} in the broader context of defunding the police\footnote{In 2018, the Canadian Centre for Policy Alternatives put forward an alternative Winnipeg budget which recommended reducing the police budget and increasing funding to social and educational services, including the library \autocite{CanadianCentreforPolicyAlternatives2018}. Post-2020 we can contexualize all of this as part of a broader police defunding/abolition and anti-carceral movement across North America \autocite{Wilt2020}.}. Amid the backlash, the library consulted with community members and, in October 2020, when COVID cases in the city fell sufficiently to allow branches to reopen, the security gates remained down \autocite{Bettens2020}. The library decided instead to fund a Community Resource Space rather than rely on security and policing \autocite{Pursaga2021}. 

Both the TPL and WPL examples indicate a closer relationship between libraries and state power than conventional narratives of enlightenment, inclusion, and universality would suggest. It is this connection to institutions to state power, mediated through the representational systems of libraries, values, and individual rights, that links these two cases, despite the fact that intellectual freedom was front and centre in the TPL case, but was conspicuous by its absence in Winnipeg. This suggests that libraries play a deeper political role in Canadian society than surface analysis would indicate. TPL's absolutist adherence to intellectual freedom resulted in the support and reproduction of transphobia in a context of a worldwide anti-trans moral panic, while WPL's implementation of strict security measures in an iniquitous settler-colonial context reproduced the structures of racial-capitalist policing and state violence. In both cases, the outcome was the same: the ideological reproduction of a constructed demonized "Other" and the increased policing and exclusion of oppressed subaltern populations. This outcome was both made possible and required by an overarching liberal universal and individidual egalitarianism, specifically the politics of recognition. 

\section*{Section?}

When TPL defended its room rental in the name of intellectual freedom, the professional leadership rallied behind it. The Canadian Federation of Library Associations \autocite{CFLA2019}, the Canadian Urban Libraries Council \autocite{CULC2019}, and various other associations offered letters of support, public library leadership closed ranks against their own staff, and dissenting voices - York University's Faculty Association open letter challenging the TPL decision \autocite{Joseph2019} and the BC Library Association's letter to the CFLA board \autocite{BCLA2019} are two examples - were dismissed (in the terms used by Schrader above) as not understanding the nuances or importance of intellectual freedom to liberal democracy. However, WPL's exclusion of marginalized people from the main library space in a historic site of settler-colonial oppression did \textit{not} draw similar responses. Library leadership and the professional associations did not call out WPL's violation of intellectual freedom in the name of security, exclusion, and marginalization. Such challenges only came from library workers, users, and community groups (see \autocite{Schmidt2019}).

Why was intellectual freedom deployed in one case but ignored in the other? What did the discourse of IF intend to signify in Toronto but not called upon to signify in Winnipeg? In both cases the values of equity, diversity, inclusion, and community, as well as the rights enshrined in the Charter were \textit{recognized} but trumped by another logic, a logic of state power and oppression operating at the levels of representation and actual police force. It is this contradiction,  the notion that signification/representation/ideology on the one hand and force on the other are not opposites but points on a spectrum, that I want to investigate in this chapter. 

\section*{Authoritarianism and Populism}

While there are many ways to analyze the events at TPL and WPL, what I will focus on here is the way that a commitment to a liberal individualist intellectual freedom is backed up by the coercive power of the state. We have seen in previous chapters how the communitarian form of liberalism, more specifically the politics of recognition, developed as an ideological response to real challenges to Canadian liberalism, eventually to be entrenched in Canadian constitutionalism. As the various critics of recognition have demonstrated, there is always a real material limit on the consequences of recognition: recognition does not mean decolonization or redistribution of wealth or life chances. My own critique of Taylor and Tully's politics of recognition showed how their communitarianism was limited by a sacrosanct view of individualism and certain inalienable individual rights. 

The lens through which I want to analyze contradiction in librarianship between freedom and coercion is the distinction Stuart Hall draws between an authentically popular politics and an authoritarian populist one. I will argue that while ideologically libraries see themselves as authentically popular, in the two cases under investigation, they in fact participate in and reproduce authoritarian populism through the representations of moral panics and the construction of demonized Others. Intellectual freedom becomes, not an overarching value of librarianship, but a tool to be selected or discarded based on the pragmatic political requirements of the moment. As I showed in the chapter on the genealogy of intellectual freedom, far from being a pure or disinterested value, intellectual freedom has always been a political concept in this sense, used for particular purposes at particular times. Understanding IF in this way allows us to contextualize it, making space for a non-absolutist view of IF and the potential development of alternative IF conceptualizations and policies.

The social contract origins of intellectual freedom and the role it plays in the democratic discourse of librarianship makes IF an integral part of the genuinely popular self-image of the profession. In this way, libraries appear to uphold liberal pluralism, neutrality, the thin theory of the good, and individual rights. In Canada, as we have seen, individualism is moderated (up to a point) by a communitarian recognition of collective identity. But in general, librarianship presents itself as a bastion of individual rights and freedoms as guarantors of democratic legitimacy. In this way, libraries are represented in society as genuinely popular institutions. In his book, \textit{Palaces for the People}\footnote{Klinenberg's title is taken from Andrew Carnegie's descriptions of libraries. Carnegie's philanthropic reputation rests in no small part on the endowment and construction of public libraries. However, Carnegie's first library was presented to his steelmill workers as a gesture of reconcilation after the largest strikebreaking operation in American labour history. This is a good example of the way the popular image of libraries serves to hide their authoritarian function. See \autocite[Chapter 16]{Krause1992}.}, for example, sociologist Eric Klinenberg expresses a commonly-held view of the role and purpose of libraries:

\begin{quote}
Why have so many public officials and civic leaders failed to recognize the value of libraries and their role in our social infrastructure? Perhaps it's because the founding principle behind the library - that all people deserve free, open access to our shared culture and heritage, which they can use to any end they see fit - is out of sync with the market logic that dominates our time... Their core mission is to help people elevate themselves and improve their situation. \autocite[84]{Klinenberg2018}\footnote{This argument is perennial within the profession. John Buschman made it in 2003's \textit{Dismantling the Public Sphere} and Ed D'Angelo's made it even more strikingly in \textit{Barbarians at the Gates of the Public Library} \autocite{DAngelo2006}.}.
\end{quote}

This common view of the role of libraries is not the only one. The "social control thesis" \autocite{Black1998} argues that libraries play a role in the maintenance of social order and the political status quo. Through late fees, for example, children are introduced to ideas of contract and penalty; the genuinely popular goal of teaching them about sharing and promise-keeping\footnote{Cf. "To breed an animal with the prerogative to \textit{promise} - is that not precisely the paradoxical task which nature has set herself with regard to humankind?" \autocite[35]{Nietzche2006}. For the relationship of Nietzche's "prerogative to promise" to late capitalism, see \autocite{Lazzarato2012}.} is operationalized through the "cash nexus" of capitalist social relations.

While the Enlightenment (or democratic) thesis and the social control thesis are often seen as contradictory, mutually exclusive views of the role and function of libraries in the West, I want to propose that in fact \textit{both} the Enlightenment and social control roles function concurrently within the context of what Hall has called "authoritarian populism". Put another way, since liberal Enlightenment itself is an ideological construction for the maintenance of capitalist hegemony, what it legitimately values (individual rights, intellectual freedom, thin theory of the good, etc.) are inextricable from the necessary social control of the capitalist order itself. This is a different argument than simply recognizing the continuum of coercion and consent, as in Gramsci's image of the centaur, and more like the nuanced Marxist recognition of ideology not as "false consciousness" but as the intellectual and cultural reflection of a \textit{true} state of affairs \footnote{This insight derives mainly from Althusser but is an important element in the Marxist cultural studies of both Hall and Jameson.}.

If libraries are not genuinely popular but rather authoritarian populist institutions, then we can explain presence or absence of intellectual freedom in the TPL and WPL cases, at least as a first approximation, as a difference in emphasis between the "authoritarian" and the "populist" elements. Derived from the social contract and filtered through the politics of recognition, intellectual freedom is a populist value; Toronto Public Library chose to emphasize the populist element up front because it was defending Meghan Murphy's intellectual freedom against trans-rights activists (the "authoritarian" element - the Toronto Police Service - was held in reserve and called upon later). In Winnipeg's case, the authoritarian element was front and centre because what was at issue was law and order. In each case, the secondary term backed up the primary one: authoritarian police force backed up TPL's populist intellectual freedom defense; populist conceptions of law and order and public safety backed up WPL's authoritarian response. In both cases, however, the trigger for the library's response was a constructed moral panic: in TPL's case around trans people; in WPL's case around Indigenous people (presented as homeless drug users). 

For Hall, authoritarian populism and moral panics are both responses to crises of hegemony and social/state legitimacy. The resurgence of the people in 1968 coupled with the profitability crisis marked a legitimacy crisis that led to the transformation we know in hindsight as neoliberalism. The moral panics of the period, primarily around immigration, race, and crime, were used by the only party to fully understand and embrace the shift - the Thatcherite wing of the Conservative party - to institute a new legitimacy. Hall calls this "The Great Moving Right Show". 

In "Race and 'Moral Panics' in Postwar Britain" (1978), Hall describes Enoch Powell's anti-immigrant racism as a response to the crisis of hegemony brought about by the wider social crisis (we have taken 1968 as the turning point here, the same year as Powell's "rivers of blood" speech). For Hall, "it is this whole crisis - not race alone - which is the subject and object of law and order crusades, the appeals to 'tough measures', the whispers of conspiracy" \autocite[63]{Hall2021c}. Hall describes how "when the 'silent' and beleaguered majorities - the great underclasses, the great silent, 'British Public' - are made to 'speak', through the ventriloquism of its public articulators, it is not surprising, then, that it 'speaks' with the unmistakable accent of a thoroughly homegrown racism" \autocite[63]{Hall2021c}.

Since the 2008 global financial crisis, the capitalist metropoles have been undergoing a new legitimacy crisis and are dealing with it in the same way. Existential threats to the social order - trans people worldwide, and Indigenous people in Canada - act as focal points or centres of gravity for new moral panics which can be capitalized on in the same ways: an extension of the power of the police and a reengagement with liberal "common sense". The library plays a role in both these aspects, contributing to the imposition of a new authoritarian populism adequate to a new round of hegemonic crisis. We will look at Hall's theory in more detail in the next chapter.

\section*{Canadian Moral Panics}

In Canada, homegrown racism has primarily (though by no means exclusively) been directed against Indigenous peoples. In the current conjuncture, a worldwide more panic against trans people supplements anti-Indigenous racism as the main Other of Canadian society. Both Indigenous and trans people represent a threat to the liberal settler-colonial social order: Indigenous people challenge the primacy of private ownership (especialy of land), resource extraction, and the narrative of Canada as a fundamentally "just society" (Pierre Trudeau's slogan). The continuing inability to reckon with its colonial past - quintessentially embodied in the intractable image of the residential school - let alone to achieve real redistribution rather than performative recognition, continues to provide a justification for anti-Indigenous racism couched in the language of liberal egalitarianism and universality, individual rights, and cultural recogntition.

Trans people, on the other hand, not only challenge the comfortable binaries of Canadian culture, but one of the last foundational certainties of bourgeois social life: the essential fixity of biological sex. Perhaps even more than that, however, they challenge the "rational" and "scientific" Enlightenment of Western culture itself. The fixed categories of scientific classifications - around sex and gender in this case - are exposed as the ideological constructions of a fully politicized science. Trans people, simply by existing, require that Canadians recognize that science is always social and political, thus removing any transcendental or independent ground for "rational" and objective politics and social policy. This rationality was always political, intended to circumvent political struggle and achieve Tully's utopian "mediated peace" by reference to objective scientific truths which could be made into procedures and algorithms and could provide the necessary conditions for a thin theory of the good and negative liberty, because what is scientifically objective is not, by definition, a moral commitment. By removing the scientific basis for biological sex, trans people expose sex and not just gender as a social construct backed by particular moral commitments.

The undermining of the certainties of liberal Canadian life goes hand in hand with the economic and political crisis of legitimacy/hegemony. Canada is riven by housing crisis, opioid crisis, climate crisis, reconciliation crisis, and the COVID pandemic, in the face of which liberalism typically leverages ideological and cultural certainties liberalism to maintain the status quo As these certainties are more and more challenged, liberal settler-capitalism loses its grip on the social order.

In moments of crisis, as we have seen, the authoritarianism necessary for the maintenance of the capitalist social order and the reassertion of legitimacy achieves a veneer of popular support by constructing and reflecting back the views of ordinary people, presumed to be natural or self-selected, the products of intellectual freedom. Hall notes that

\begin{quote}
When crime is mapped into the wider scenarios of 'moral degeneration' and the crisis of authority and social values, there is no mystery as to why some ordinary people should be actively recruited into crusades for the restoration of 'normal times' - if necessary through a more-than-normal imposition of moral-legal force. \autocite[143]{Hall1980}
\end{quote}

The categorization of trans people - like gay people before them - as "morally degenerate" thus dovetails with the fear of the violence of poor people, drug addicts, or Indigenous people. The role they play in the institution of "authoritarian populism" is the same. Libraries participate in the construction and demonization of these Others in order to play a structural role in authoritarian populism, by representing - through room rentals or security policies, for example - the demonized Other that needs to be the scapegoat for social and economic crisis. Intellectual freedom is one mechanism by which "ordinary people" come to think of the library as not having a social agenda, and by which libraries can shore up the idea of "common sense" as natural or innate; in reality these ordinary people are made to speak - thanks to the library - in the language of transphobia and anti-Indigenous racism.

\section*{Conclusion}

The Enlightenment discourse of libraries, democracy, and intellectual freedom has tended to see libraries as objective ("neutral") institutions which enable individual choice and self-development through unfettered access to information. In this sense, negative liberty and a thin theory of the good are built into intellectual freedom and other library values and procedures. Many of these other values - around community partnerships, for example - attempt to mitigate pure procedural liberalism by introducing an element of communitarianism or social responsibility. The presumption of individualism and free intellectual choice - like the voice of the silent majority - is presumed to be original to library users (i.e. in a contractarian state of nature, innate) and so the library's role is to enable this pre-existing individual agency and to stay out of the way in order not to influence individual users with the library's or the state's conception of the good. 

However, the very idea of a thin theory of the good is spurious, suggesting as it does that there can ever be an absence of moral commitment. In reality, there are only positive commitments to competing (or at least alternative) good. Every appeal to neutrality or agnosticism with respect to the good simply masks a commitment to some other, obscured, good. The appeal to the Bill of Rights or to the Declarations and Charters of Rights and Freedoms affirm at the very least a commitment to an individualist social ontology (and more likely to the whole set of tacit good required of capitalist society: private property, sanctity of contract, etc.).

By relying on a particular lineage of intellectual freedom - Madison and Jefferson, or Mill, or Habermas - the library asserts the primacy of liberalism and the bourgeois social order. It sets the terms of the debate in such a way that challengers of the social order are forced to speak in its language. In discussing the way in which the terms of discourse were set by the British media, Hall notes that

\begin{quote}
What is significant is not that they produce a racist ideology, from some single-minded and unified conception of the world, but that they are so powerfully constrained - 'spoken by' - a particular set of ideological discourses. The power of this discourse is its capacity to constrain a very great variety of individuals: racist, antiracist, liberals, radicals, conservatives, know-nothings and silent majoritarians. \autocite[115]{Hall2021e}
\end{quote}

In fact, the very pluralism of the debate ends up reinforcing the dominant or hegemonic perspective: "Liberals, antiracists, indeed raging revolutionaries can contibure 'freely' to this debate, and indeed are often obliged to do so, so as not to let the case go by default: without breaking for a moment the chain of assumptions which holds the racist proposition in place. However, changing the terms of the argument, questioning the assumptions and starting points, breaking the logic - this is a quite different, longer, more difficult task" \autocite[115]{Hall2021e}\footnote{This thesis is a contribution to breaking this logic in libraries: to question the assumptions around individuality, freedom, democracy, and the Charter of Rights and Freedoms in Canada.}.

This is the problem with the liberal insistence that open debate representing "all sides" is the proper response to the increased polarization of crisis society. Racists and transphobes are not interested in finding the truth through the open testing of opinions (as per Mill), but of reasserting and confirming the (racist or transphobic) certainties of "normal" society. Libraries, especially under the aegis of intellectual freedom, are absolutely implicated in this process. The granting of a platform to transphobic speakers allows transphobia to set the terms of the debate in exactly the same way the British media allowed Enoch Powell to do in the 1970s. 

Indeed, the recuperation of pluralism is a hallmark of hegemony itself. In a series of lectures from 1983, Hall remarked that when the dominant culture is challenged, hegemomy proves itself

\begin{quote}
by the fact that the dominant culture need not destroy the apparent resistance. It simply needs to include it within its own spaces, along with all the other alternatives and possibilities. In fact, the more of them that are allowed in, and the more diverse they are, the more they contribute to the sense of the rich open-ended variety of life, of mutual tolerance and respect, and of apparent freedom. The notion of incorporation points to the extremely important idea that the dominant ideology responds to opposition, not by attempting to stamp it out, but rather by allowing it to exist within the places that it assigns, by slowly allowing it to be recognized, but only within the terms of a process which deprives it of any real or effective oppositional force. \autocite[50]{Hall2016}
\end{quote}

This quote could stand for the entire discourse of intellectual freedom in libraries. And while Hall is thinking specifically of oppositional ideologies like Marxism, Critical Race Theory, or radical feminism, it applies just as much to right-wing ideologies of transphobia and racism. Filtered through libraries' intellectual freedom policies, racist and transphobic discourses are legitimized precisely under the guise of a pluralist "all sides" liberalism\footnote{The banrkuptcy of this pluralism reached its apotheosis in Texas in October 2021, when a school administrator suggested that books condemning the holocaust should be balanced by books with "opposing views" \autocite{Prose2021}.}. We must also note how recognition plays an important role here, as a theoretical mechanism whereby the plurality of voices is given access to the spaces afforded to them.

However, the development of authoritarian populism and the right-wing project of reconstruction of legitimacy ends up using the hegemonic pluralism of intellectual freedom to construct a demonized Other to be the focus of "law and order" while reflecting and drawing upon "common sense" popular support for these positions. The social order that the right seeks to construct relies upon a notion of norms that be definition excludes those who do not conform. Discussing Durkheim, Hall writes that

\begin{quote}
Society has at its core a set of collective ideas and norms which
hold the whole thing together; and if a society is to reproduce itself,
it must also reproduce those collective representations and normative
structures. That is the nature and the source of social order,
which is only maintained insofar as the normative structure of a society
controls or defines the limits of acceptable behaviour within
the society. Consequently, social order is dependent on constraint.
Those who are not within the normative order are subject to control,
preferably being induced back into the structure. It is within this
conception of social order that Durkheim talks about crime as more
than an infringement of the social or normative order, for it takes on
a symbolic importance within every society by creating the opportunity
for the ritual act of punishing those who are the exception to
the rule. Only through punishment does a society reaffirm its normative
integration and the power of its normative structure. \autocite[58]{Hall2016}
\end{quote}

Thus, what connects the TPL and WPL cases - unrelated on the surface - is the construction of an excluded, abnormal population: poor people, meth addicts, Indigenous people, trans people - in order to reaffirm a social order that is mystfied and hidden by the very adherence to intellectual freedom. IF becomes the mechanism by which settler-colonial, racial, transphobic capitalism retains its hegemony \textit{at the same time} as the concept of IF denies it. In order to lay bare this mystifying, ideological role, we have looked at the liberal political theory that underpins IF and its own history in 20th century librarianship. It remains to to answer two questions: how precisely does librarianship represent the demonized Other and construct an authoritarian populist "common sense", and what socio-political ontology can be offered as an alternative to the contractarian ontology of recognition? We will look at each of these in turn in the next two chapters. 




