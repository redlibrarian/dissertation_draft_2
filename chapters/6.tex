% Stuart Hall chapter - outline.

What I have called the "pure" political theory of liberalism, communitarianism, and the politics of recognition understands itself as a movement of ideas, untouched and unencumbered by worldly considerations. In \textit{Sources of the Self}, Charles Taylor explicitly repudiates a "causal-historical" account of the development of modern individuality. This form of idealism runs counter to a tendency, beginning with Hegel, to recognize precisely what Edward Said has called the "worldliness" of ideas and representations of reality. Marx's historical materialism, itself influenced by Hegel and Hegelians like Feuerbach, was only one of a number of "impure" philosophies that took seriously the fact that a philosophy is 

\begin{quote}
 already and necessarily contaminated by its involvement with power, position, and interests, whether it was a victim of them or not. Worldliness - by which I mean at a more precise cultural level that all texts and all representations were \textit{in} the world and subject to its numerous heterogeneous realities - assured contamination and involvement, since in all cases the history and presence of various other groups and individuals made it impossible for anyone to be free of the conditions of material existence. \autocite[49]{Said2004}
 \end{quote}
 
Thus, while Rawls, Taylor, and Tully were debating the right balance of individual and collective rights on on the idealistic terrain, capitalism was moving forward regardless. Stuart Hall's writings on the causes, character, and consequences of Thatcherism are an example of Said's worldliness: a theorizing that takes the concrete movements of the real social and political world seriously and attempts to grasp the conjunctural details as they really exist. Hall's project throughout the 1970s and early 1980s was to fully understand and explain the politics of the transition to neoliberalism, that is the Thatcherite project. Rawls - and to a certain extent Taylor and Tully - by ignoring the concrete political realities, continued to write as if the Welfare State still existed. Katrina Forrester points out that "\textit{A Theory of Justice} must be understood as a book of the postwar, not as a response to the years of the Great Society"\autocite[3]{Forrester2019}, that is the mid-1960s; equally, what followed Rawls' work in the 1970s, 1980s, and 1990s is not really a response to the neoliberal transition, except implicitly, in the Hegelian-Marxist sense that a given historical moment produces a particular outlook and set of ideas. The "pure philosophy" of liberalism must be understood precisely as insulating the hegemonic ideology of capitalism from the worldly realities of exploitation and oppression. It is for this reason that intellectual freedom is unable (not just unwilling) to deal adequately with questions of trans rights and Indigenous oppression.

It is to Hall's theory that I will turn now. If Taylor and Tully expressed, despite themselves, a liberal response to the twin challenges of Indigenous sovereignty and Quebecois nationalism, they did so without tackling the concrete details of these challenges. Hall, on the other hand, not only dealt in great sociological detail with moral panics (centred on crime, racism, and immigration), but with the ways the media constructed representations of moral panic and threats to the liberal social order, the way those representation fed into a build-up of law and order and a cultural shift to the right, and the way that shift provided a new legitimacy for the right-wing hegemony of the neoliberal period. 

Hall's theory, more than Taylor and Tully's, can help us answer the following question: in what sense can government institutions like public libraries defend spaces of individual freedom against state power, while simultaneously bringing state power \textit{into} the libraries themselves? What the Toronto and Winnipeg Public Libraries have in common is a "drift" into what we can think of as the "law and order library". The mechanism by which both libraries accomplished that involved their self-description as institutions that support classical liberal individualist rights (intellectual freedom and bodily security respectively), the simultaneous representation of uniqely dangerous and disruptive social elements (trans people and Indigenous people) and the library as defender of a liberal social order; and the leveragine of this "authoritarian populist" position to legitimate the deployment of state and quasi-state power in library spaces.

What I want to do in this chapter is to analyze the two recent controversies in Canadian librarianship to tease out the ways libraries, far from being "neutral" guarantors of quasi-natural rights, in fact work to construct particular social and political representations within Canadian society. I will demonstrate that libraries participate in what Stuart Hall has called "authoritarian populism", a "new combination of coercion/consent" in which state power is normalized within a context of anti-state protection of civil liberties. For Hall, authoritarian populism is not truly anti-state but rather attempts "to represent itself as anti-statist for the purposes of populist mobilization". It is this mystified anti-state position that allows the library to bring the police into spaces defined as areas of intellectual freedom, of liberal toleration, and of excluding oppressed and marginalized people in the name of democratic freedom and security. 

Over the course of a long career, Stuart Hall's interests encompassed "New Left" Marxist theory, the politics of media and communications, ideology, race, and post-colonialism. One major concern was the development of Thatcherite politics a tool for the neoliberal restructuring of capitalism in and after the late-1960s/1970s crisis, what Hall referred to as "the great moving right show". Hall argued that in the management of crisis, the socio-economic restructuring necessary for capitalism to survive requires a level of discipline that runs counter to liberal political thinking hegemonic in Western democracies. The imposition of authoritarianism under the guise of populist democracy begins as a dual movement: on the one hand the representation of a threat to liberal values (including intellectual freedom), with a scapegoat safely to hand (in Hall's case, Black people and immigrants; in our case trans people and Indigenous people); on the other hand an increased reliance on "law and order". These two movements interlock and reinforce each other, leading toa regime Hall described as "authoritarian populist".

In this chapter, I will trace the following line through Hall's work. First, Hall's understanding of representation and the encoding of ideological content in a broad definition of "texts", encompassing non-textual structures like the production conditions of television programs or (as I will argue) the policies and performances of public libraries. Second, the use of this representational mechanism to construct and reproduce moral panics around which support for the dominant social formation can be constructed. Hall's interest was primarily around racism as the focus of moral panic, and I will extend this analysis to include trans and Indigenous people in the Canadian context. The rallying of support for the status quo against disruptive "anti-social" elements leads to the third aspect, that of the right-wing deployment of populism to shore up an authoritarian social and political culture in order to bring capitalism successfully through a period of crisis. 

In my analysis of the TPL and WPL events, I will show how libraries participate in this authoritarian-populist process, and how intellectual freedom - far from being an overriding principle or value of libraries - is in fact pragmatically deployed or dropped depending on the particular needs of the moment. At TPL, intellectual freedom was used as a justification first for the representation of an anti-trans moral panic, then for the deployment of police force against citizens. At WPL, by contrast, intellectual freedom - which \textit{should} have been at the forefront of the debate - was conspicuously absent from the library's "law and order" narrative. 

Taking as our starting point the libraries' reliance on police force to back up their social claims (protecting intellectual freedom and security/safety), this chapter will unfold as follows: First, we will investigate Hall's theory of encoding/decoding and of representation more broadly, allowing us to see how libraries function as representational systems in their own right [encoding/decoding, representation]. Then we will focus on the actual construction of demonized Others and moral panics, drawing on Hall's writings on the media and racism [Whites of their Eyes, Race and Moral Panics, Policing the Crisis, Poulantzas], and the role these moral panics play in the legitimation of law-and-order discourse [Drifting into the Law and Order Society, The Great Moving Right Show] Finally, Hall's theories of representation, moral panics, and law and order, will be integrated into his view of authoritarian populism, which he contrasted with a genuinely popular democratic politics [Poulantzas intro, AP chapter, reply to Jessop]. 

\section*{Representation: Encoding and Decoding}

The question of the representation of reality has plagued philosophy at least since the Greeks. For much of its history, philosophy considered representation - in a work of art, or in scientific discourse - either in rationalist or empiricist terms: representation was either a matter of "clear and distinct ideas" about the world (but distinct from it) or it was a simple reading off of the material world by senses unmediated by thought or perceptual faculties. The scientific revolution of the 17th century threatened, in Hume's work, the entire rationalist possibility of knowing the world, setting in motion various attempts to reconcile idealist and empiricist epistemologies. Kant's critique of pure reason sought to restore the primacy of rational thought as a way of getting at objective truth in the face of Hume's skepticism; Hegel attempted to overcome limitations of Kant's critique, etc.

In the postwar period, and particularly in the 1970s, in response to the determinism of structuralism, new epistemologies of representation developed which departed from the individualism of both rationalism and empiricism and focused on the \textit{social} productions of representations. Erich Auerbach's \textit{Mimesis} (1946; English edition 1953) demonstrated the sociality of representations in Western literature, while Thomas Kuhn's \textit{Structure of Scientific Revolutions} (1962) did the same for the natural sciences.

Roy Bhaskar, in his \textit{Realist Theory of Science} (1975) developed the idea of what he called the "intransitive" and "transitive" objects of science, in contrast to both Hume and Kant. The intransitive, in Bhaskar's theory, is the real, physical world that is the object of scientific study; the transitive are all the documents, theories, paradigms, notes, and discourses of scientists, which are socially produced. For Bhaskar, we can only truly know these transitive objects because (following Vico) they are things we have made (see \autocite[90-92]{Said2004}).

Hall takes a similar approach to cultural representation, arguing that language is neither \textit{reflective} (i.e. fundamentally related to the world it represents) nor \textit{intentional} (in which individual speakers produce their own idiosyncratic representations), but socially constructed; meaning is therefore neither out there in the world (empiricism) nor in the speaker's individual mind (rationalism) but is a product of our social relations; language is therefore "worldly" in Said's sense\footnote{Said quotes Richard Poirier's view that literature demonstrates "what can be made, what can be done with something shared by everyone, used by everyone in the daily conduct of life, and something, besides, which carries most subtly and yet measurably within itself, its vocabulary and syntax, the governing assumptions of society's social, political, and economic arrangement... literature depends for its principle or essential resource on materials that it must share in an utterly gregarious way with the society at large and with its history" \autocite[59-60]{Said2004}. We can add here Paolo Virno's description of the mother tongue as "belong[ing] to everyone and no one; it is a public and collective dimension. It shows with great clarity the preliminary sociality of the speaker" \autocite[65]{Virno2015}.} Hall writes that the constructionist approach

\begin{quote}
recognizes this public, social character of language. It acknowledges that neither things in themselves nor the individual users of language can fix meaning in language. Things don't \textit{mean}: we \textit{construct} meaning, using representational systems - concepts and signs. [...] According to this approach, we must not confuse the \textit{material} world, where things and people exist [Bhaskar's intransitive] and the \textit{symbolic} practices and processes through which representation, meaning and language operate [Bhaskar's transitive]. \autocite[11]{Hall2013}.
\end{quote}

It follows that "individual" speakers - individuals who engage with \textit{any} symbolic practice (readers, musicians, photographers, etc.) never do so out of some pre-social individualism, but are always-already socialized \textit{into} the structures of representation of their society. This is Judith Butler's point about gender, and it applies to intellectual engagement with any literary, scientific, or cultural object.

A constructionist approach to representation and meaning therefore has grave consequences for librarianship. Many areas of library work - cataloguing and classification, for example - have traditionally been bound to an empiricist, reflective epistemology (though this is starting to change). Intellectual Freedom, on the other hand, is bound to an intentional epistemology, by which self-determining individuals freely impose their own meanings on the world - we might even call this the definition of intellectual freedom. A constructionist approach like Hall's challenges not only the liberal conception of individual \textit{freedom} but of individualism itself. The "individual" who possesses intellectual freedom cannot be conceived as unsullied by social relationships, culture, and history until they choose to enter into them, always maintaining a distance and objectivity, a proud and independent \textit{self}. Rather, this individual is a product of society, history, and culture, and can only engage intellectually with cultural objects from inside them, as part of the world. There is no foundational, or external, or privileged safe position from which to engage any kind of "freedom", intellectual or otherwise. There is only history and social construction.

While Hall's theory derives from a particular view of language, it was one of his major insights to apply that view to other representational or symbolic systems, other kinds of cultural structures. In this way, he developed a theory of representation and ideology that pushed forward Marxist views of ideology, rejecting both the empiricist conception of ideology as "false consciousness" (that is, empirical error) and the structural determinations of Althusserian ideology. Ideology - as a representational system - is socially constructed as a language, as a way of articulating empirically true social realities in particular ways. 

Hall also applied this view to concrete cultural manifestations like television programs. In "Encoding and Decoding in the Television Discourse" (1973), Hall... 