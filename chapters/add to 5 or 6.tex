\section*{Libraries as Signifying Systems}

Emphasis here on the ideological role played by libraries should not obscure their material role played in the establishment of authoritarianism in the name of "law and order". The Toronto Police Service was involved early on in TPL's plans to manage the protest against Radical Feminists Unite, and the Winnipeg Police were consulted on the new security plans at WPL. Since the financial crisis of 2007-2009, police presence in libraries has become ubiquitous, as it has in schools, and other areas where hegemony risks being challenged. Coercion and consent are not two opposing poles, but are rather terminal moments on a spectrum of domination.

However, I want to dig more deeply into the ways libraries perform their ideological role. To this end, I want to offer a view of libraries as signifying systems in their own right. Libraries have always engaged in the practice of representation: classification systems are perhaps the clearest example, but libraries also perform cultural representation through book displays, childrens' programming, and library policies. A recent trend towards allowing children to "read away" their late-fees has been met with the response that such a policy makes reading a punishment, reinforces what Marx and Engels called the "cash nexus" (i.e. reducing all social interaction to cash exchange) and reinforces social inequality, because equal fines have unequal effects on different socio-economic strata. Similarly, the whole concept of late fees inculcates library users into the idea of contract, respect for property, and the following of rules. These material requirements of library participation create a particular \textit{subjective} orientation to library services and to capitalist society as a whole, as library historians like Alistair Black have argued\footnote{More broadly, E.P. Thompsons's essay "Time, Work-Disciplie, and Industrial Capitalism" (1967) makes the point that time-discipline had to be consciously ingrained in the working classes through concrete procedures and policies.}.

But the positivism of librarianship, originating in the prestige of technical efficiency of the last quarter of the 19th century, and achieving real dominance with the rise of "scientific information" and the military-industrial complex, requires that librarians subscribe to a "reflective" view of representation: the idea that our representations have a one-to-one relationship with reality which can simply (and neutrally) be read off. Description and classification are then nothing but objective reflections of the reality they describe. While this may have been true to a certain point from the limited perspective of bibliographic reality, as libraries become occupied more and more with the description of a non-bibliographic world, the reflection theory becomes less and less tenable. Challenges to Library of Congress Subject Headings ("illegal aliens" for example) have been the subject of fierce controversy in recent years. 

However, recognition of classifications as \textit{constructed} has begun to make inroads into the profession. Attempts to deal with outdated, racist, and offensive descriptions have led to initiatives such as the "Decolonizing Description" project at the University of Alberta, which seeks to create a vocabulary for the culturally-aware description of Indigenous materials. Whether this project turns out to be more than just an exercise in the politics of recognition remains to be seen, but it has certainly helped in decentring the reflection theory of representation. 

However, elsewhere in the profession, the reflection model remains dominant. In terms of space rentals and platforming, there is an idea that libraries simply find attitudes and opinions in civil society. This goes along with the idea of the individual and the individual's agency in intellectual activity which forms the basis of Intellectual Freedom. The library in this view does not \textit{construct} any kind of intellectual message, it simply reflects what is already present in society. The responsibility is on people with other points of view to leverage the library's supposed neutrality to challenge opinions openly. However, as we have seen, libraries are very much part of the construction of a social and ideological message, the definition of who is included and excluded from "proper" social norms, and the deployment of state violence against the Others in order to clearly draw the line between inside and outside and thus reaffirm and bolster the social order itself. 

In this sense, libraries and librarians do not merely curate, preserve, and provide access to meaning, but are active participants in its construction. This has serious consequences for the liberal ideology which dominates librarianship, as it means that there is no such thing as neutrality, that negative liberty and a thin theory of the good are not adequate ways of conceptualizing the role of libraries in society. As signifying systems operating through a plethora of modes of representations, libraries must be understood as politically contested sites of meaning-making and interpretation. Library values like Intellectual Freedom, and their reference to legal structures like the Charter of Rights and Freedoms, while they are put forward as ways to \textit{stop} the struggle over meaning and social interpretations, are in reality elements in the representational chain itself. Attempting to ground the contestation and struggle over meaning - politics itself - in ostensibly solid and sacrosanct documents, concepts, or procedures is an authoritarian move, even when it is cloaked in the pluralism of recognition.

%Back to the concrete case studies... 

\section*{Transphobia and the Ambiguity of Authority}

We are now in a position to analyze more deeply what went on at TPL. The construction of a demonized trans Other is part of a global tendency often called (especially in the UK) "gender critical feminism". Hall's point about the silent majority feeling as if they are made to say black is white fits with the discourse of gender critical feminism, for whom biological sex is an unquestionable scientific fact, and who feel they are being made to say that trans women are women (for example) against common sense and scientific truth. As Hall, following Gramsci, demonstrates, common sense is not a natural artifact that autonomously arises in a society, but is a socially meaningful construction which "presents itself to popular experience as transhistorical - the bedrock, universal wisdom of the ages"\autocite[142]{Hall1980}\footnote{This conceptualization of common sense is itself a product of the individualism of modern capitalist society. Gadamer, for instance, writes of an earlier \textit{sensus communis} which "obviously does not mean only that general faculty in all men but the sense that founds community"\autocite[19]{Gadamer2013}.}. Trans rights become a site of struggle in the crisis of epistemological authority (of scientific truth, for instance) as well as social struggles around misogyny, patriarchy, and gender-based violence.

Libraries can take this unobjectionable common sense as a natural artifact, arising from the intellectual agency of individuals, and lend it their legitimacy by platforming transphobic views. While this implies an \textit{authority} provided by the library as platform (i.e. citizens trust the library because they assume that all selections - of library stock or of platformed speakers - is made on solid, trustworthy principles), the library itself rejects such authority. Time and again the Toronto City Librarian, Vickery Bowles, stated that the principle of Intellectual Freedom meant that the library did not endorse any speaker it platformed, just as the library does not stand by every opinion in its book collection. But this obscures the very real decisions made by librarians as they select and deselect library stock. Librarians remove items from the collection based on out-of-date, erroneous, or dangerous information all the time. Citizens themselves see libraries as authorities on trustworthy knowledge, looking to libraries to give an imprematur on the basis of truth and worthiness. 

This ambiguity with respect to authority sits at the heart of librarianship. State institutions that are against state censorship, libraries have to try to navigate this ambiguous situation every day. Taking responsibility for authority would mean rejecting negative liberty and the thin theory of the good, which is a difficult prospect, and which is why libraries - like so much else in liberalism - seeks to avoid responsibility (for success just as much as for failure) by outsourcing human decisionmaking to procedures. Collection decisions get outsourced first to library approval plans, then to click-to-purchase "demand-driven acquisition". Libraries subscribe to the Dworkin's proceduralism in order to refuse responsibility for the social authority they hold. 

And it is this, I think, that makes debates over Intellectual Freedom so contentious. Any challenge to the dominant way of understanding Intellectual Freedom is automatically equated to censorship. So often, those who seek a more sophisticated or nuanced understanding of IF are forced to debate things on the terms set by censorship and individual agency, meaning that the terms of the debate can only be shifted with great effort and difficulty, as Hall noted.

One concrete way libraries refuse to accept the mantle of responsibility is through doubling down on individualism and individual agency. They adopt a populist approach, opting for a view of opinions and perspectives and values as naturally occurring in a population, which entails suppressing the library's own role as signifying system and ideological apparatus. Accepting authority would mean committing libraries to an active role in the construction of social representation, which would mean choosing a real conception of the good to promote, and adopting positive liberty as an orientation towards justice. But this would involve the exercise of state power in the construction of a social order. Libraries do not (currently) have it in them to accept this state of affairs. Thus states go about the business of constructing a social order, allowing libraries to play a role in that construction while denying that it is doing so. 

In many ways, this ambiguity arises out of real contradictions in libraries. They are community institutions, publicly funded, with a professional discourse of public service and social betterment. And yet they are also state institutions and (as we have seen) signifying systems in their own right. 

The agency of public libraries in selecting what voices can benefit from the authority and trustworthiness of the library's image is significant. Far from being neutral, far from representing "all sides" of a particular discourse - either through room rentals or collection development - libraries are necessarily selective in which discourses get represented within the library as as a signifying system. The following example from TPL illustrates this tension.

In terms of Canadian constitutionalism, the politics of recognition moderates any absolute conception of rights and freedoms in \S1 of the \textit{Charter of Rights and Freedoms}. "Whereas Canada is founded upon principles that recognize the supremacy of God and the rule of law", the \textit{Charter} reads, "The \textit{Canadian Charter of Rights and Freedoms} guarantees the rights and freedoms set out in it subject only to such reasonable limits prescribed by law as can be demonstrably justified in a free and democratic society" \parencite[\S1]. It is these "reasonable limits" that enshrine the difference principle in a liberal rights regime, defining a politics of recognition that Taylor formalized a decade later.

With respect to Intellectual Freedom in Canada, this regime is illustrated by changes made to the Toronto Public Library (TPL) Room Booking policy in 2018. In July 2017, a TPL room was booked to hold a memorial for Barbara Kulaszka, a lawyer who had defended holocaust-denier Ernst Zundel, Marc Lemire, leader of the white supremacist Heritage Front, and others. Community members challenged the Library's rental of the room on the grounds that it would legitimate far-right, anti-semitic, or white supremacist views. Initial legal advice provided by the City suggested that

\begin{quote}
TPL does not have the grounds to deny the booking for the memorial service based on the \textit{Canadian Charter of Rights and Freedom} [sic], the Library's \textit{Rules of Conduct} or other policies. From the Library's perspective, values enshrined in the Charter and in particular, the principles of freedom of expression, are core to the Library's mission and values. \parencite[3]{TPL2017a}.
\end{quote}

This legal advice was clearly based on an American understanding of unfettered individual rights, which ignored the balancing clause included in \S1 of the Charter. After further consultation and legal advice, TPL revised its Room Booking Policy to balance the individual right to free expression with the "reasonable limits" described in the Charter: 

\begin{quote}
the revisions have been set out to balance the interests of a welcoming supportive environment within the provisions of freedom of speech and expression. The revised language in TPL's Community and Event Space Rental Policy communicates to those wanting to book library spaces, library customers in general, and all stakeholders throughout Toronto that hate activity is not permitted on library premises. \parencite[4]{TPL2017a}
\end{quote}

Specifically, the Denial of Use sections of the policy depart from neutralit, a "thin concept of the good", and negative liberty, stating that rentals will be denied or cancelled if TPL

\begin{quote}
reasonably believes the purpose of the booking is likely to promote, or would have the effect of promoting, discrimination, contempt or hatred of any group, hatred for any person on the basis of race, ethnic origin, place of origin, citizenship, colour, ancestry, language, creed (religion), age, sex, gender identity, gender expression, marital status, family status, sexual orientation, disability, political affiliation, membership in a union or staff association, receipt of public assistance, level of literacy or any other similar factor. \parencite[5]{TPL2017a}
\end{quote}

This limitation would be unheard of in the American Intellectual Freedom context. Indeed, among various defenders of absolute Intellectual Freedom in librarianship, director of the ALA's Office of Intellectual Freedom James LaRue wrote in support of TPL's original policy. In his letter, after quoting from the Library Bill of Rights and reiterating the the democratic discourse of librarianship and the liberal Intellectual Freedom position derived from Madison, Jefferson and Mill, LaRue writes

\begin{quote}
In both our countries, free speech is the brand of the public library. It's what we stand for. Sometimes it is our uncomfortable duty to remind our communities that free speech means more than the right to say or believe what no one objects to. Even when the speech is false or egregious, the best defense is exposure, not suppression. \parencite{LaRue2017a}.
\end{quote}

LaRue concludes by urging TPL "to retain the policy structure that has made Toronto Public the highly respected institution it is".

On the one hand, then, "recognition" is inadequate to achieving real social justice and as we have seen is easily recuperated to structures of oppression surrounding gender and race in a white, heteropatriarchal, settler-colonial context. But the broader suggestion I want to make is that the necessary selectivity of the library, with respect to platforming and Intellectual Freedom as much as anything else, is a question of politics and meaning more than a simple question of legislation and juridical procedure. Thus, when Alvin Schrader writes that "We must recognize that those who try to restrict free expression must show it is harmful; that the legal test of harm reduction constitutes the legal constraints and limits on state intervention"\autocite{Schrader2019}, not only does this attempt to offload professional decision-making onto the (ostensibly) transcendental power of the law, but it ignores the very real scope of action for libraries \textit{within} the legal framework. For example, when Ottawa Public Library denied a room rental to a group which sought to show a white supremacist film in a library space, the group sued, but an Ottawa court found that, because of the commercial nature of the rental, that function of the library was not covered by the \textit{Public Libraries Act}, and therefore not subject to the \textit{Charter}. Essentially, once the transaction was an exchange relationship, the library stopped acting as a public organization, and could make decisions about whom it rents space to on other lines than those strictly laid out by the \textit{Charter}. Defenders of the dominant conception of Intellectual Freedom do not talk about this case\footnote{I have been very careful throughout this thesis to describe the Radical Feminists Unite transaction as a \textit{rental} rather than the term preferred by the library, a \textit{booking}. It is important to distinguish between the private exchange-relationship of a renatl and the public nature of a booking/}.


\section*{Indigeneity, Land, and Dispossession}

\section*{The Rejection of Individualism}

We have focused throughout this theses on the individualist social ontology of liberalism and its consequences for Intellectual Freedom. How far are we justified in rejecting the presupposition of individualism? Regarding the work of Giambattista Vico, Gadamer writes that "what gives the human will its direction is not the abstract universality of reason but the concrete universality represented by the community of a group, a people, a nation, or the whole human race. Hence developing this communal sense is of decisive importance for living"\autocite[20]{Gadamer2013}.

This communal sense balances - in a way that individualistic liberalism cannot - the individual and the collective. In discussing Althusser's structural Marxism, Hall argues that Marx's work seeks to make this balance visible through multiple levels of abstraction: 

\begin{quote}
at the higher levels, Marx is working more with the notion of men and women as bearers of relations; but at the lower, more concrete levels, he works more with the notion of men and woman as making their own history. But this second notion does not return us to the the humanist subject, for in no sense does this second notion conceive of human beings as agents who can see through to the end of their practices. \autocite[103]{Hall2016}
\end{quote}

Marx's view that men and women "make their own history, but they do not make it just as they please"\autocite[15]{Marx1963} opens up not only the question of necessity and freedom, but the question of ideology and hegemony. For if people cannot "see through to the end of their practices", then symbolic representations and common sense become necessary tools to situate individuals within the social world. This allows the state - including libraries - to intervene in the process of signification and the making of meaning. Hall writes that "Gramsci paid close attention to the 'ethical' function of the state, by which he meant the 'work' which the state performs on behalf of capital in establishing a new level of civilisation, creating a new kind of social individual appropriate to the new levels of material existence accomplished by the development of capitalism's base"\autocite[83]{Hall2021a}.

% discourse and the construction of individualism.



